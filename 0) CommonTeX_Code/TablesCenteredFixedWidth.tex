\documentclass{article}
\usepackage{amsmath}
\usepackage{array}
\usepackage{tabulary}
\newcolumntype{K}[1]{>{\centering\arraybackslash}p{#1}}
\begin{document}
\begin{center}
table with centred columns of specified width
\bigskip

\begin{tabular}{K{2cm}K{2cm}}
\hline
foo & bar \\
foobar & barfoo \\
\hline
\end{tabular}
\end{center}

\bigskip
\begin{table}[htp]
 \caption{Hamming code properties.}
\begin{center}
\begin{tabular}{||K{1cm}|K{3cm}|K{3cm}|K{3cm}||}
\hline\hline 
$d$ &Parity &Detected Bits &Corrected Bits             \\ \hline
       &Even  & $\frac{d}{2}$ & $\frac{d}{2}$                \\
       &Odd   & $\frac{d-1}{2}$ & $\frac{d-1}{2}$          \\\hline\hline
  1   &Odd.  &  0                     &      0                           \\
  2   &Even &  1                     &       0                          \\
  3   &Odd  &  1                     &       1                          \\
  4   &Even &  2                     &       2                          \\
  5   &Odd  &  2                     &       2                          \\
  6   &Even &  3                     &       3                          \\
  7   &Odd  &  3                     &       3                          \\
$\cdots$&$\cdots$&$\cdots$&$\cdots$ \\ \hline\hline
\end{tabular}
\end{center}
 \label{tab:HamCodeProps}
\end{table}%

\bigskip

as we make the total width wider, the columns adjust \vskip0.5ex
%\begin{equation*}
%A = 
%\begin{pmatrix}
%1 & 2 & 3 \\
%4 & 5 & 6 \\
%7 & 8 & 9
%\end{pmatrix}
%\end{equation*}
\begin{equation*}
\mathbf {G^{T}} :=
 \begin{pmatrix}
 1&1&0&1\\
 1&0&1&1\\
 1&0&0&0\\
 0&1&1&1\\
 0&1&0&0\\
 0&0&1&0\\
 0&0&0&1\\
 \end{pmatrix},\qquad \mathbf {H} :={\begin{pmatrix}1&0&1&0&1&0&1\\0&1&1&0&0&1&1\\0&0&0&1&1&1&1\\\end{pmatrix}}
\end{equation*}

\end{document}
