\mode<presentation>
{
   % \usetheme{CambridgeUS}
    %\usetheme{Madrid}
    %  \usetheme{Warburg}
    \setbeamercovered{transparent}
    \usecolortheme{beaver}
    \definecolor{darkred}{rgb}{0.8,0,0}
    \definecolor{unime}{rgb}{0.77,0.12,0.23}

    \setbeamercolor{alerted text}{fg=darkred!40!white}
    \setbeamercolor*{palette primary}{fg=darkred!40!black,bg=red!85!white}
    \setbeamercolor*{palette secondary}{fg=darkred!30!black,bg=red!60!white}
    \setbeamercolor*{palette tertiary}{bg=darkred!60!black,fg=orange!20!white}
    \setbeamercolor*{palette quaternary}{fg=darkred,bg=red!20!white}

    \setbeamercolor*{sidebar}{fg=darkred,bg=orange!15!white}

    \setbeamercolor*{palette sidebar primary}{fg=darkred!10!black}
    \setbeamercolor*{palette sidebar secondary}{fg=white}
    \setbeamercolor*{palette sidebar tertiary}{fg=darkred!50!black}
    \setbeamercolor*{palette sidebar quaternary}{fg=red!10!white}

    \setbeamercolor*{titlelike}{parent=palette primary}
    \setbeamercolor{frametitle}{bg=darkred!40!white}
    \setbeamercolor{frametitle right}{bg=darkred!30!white}

    \setbeamercolor*{separation line}{}
    \setbeamercolor*{fine separation line}{}
}

%\usepackage{beamerthemesplit}
\usepackage[orientation=landscape,size=custom,width=16,height=9,scale=0.5,debug]{beamerposter}

\setbeamerfont{footnote mark}{size=\tiny}

%\setbeamertemplate{frametitle}[default][left]
%\setbeamercovered{transparent}
%

%\makeatletter
%\let\save@measuring@true\measuring@true
%\def\measuring@true{%
%  \save@measuring@true
%  \def\beamer@sortzero##1{\beamer@ifnextcharospec{\beamer@sortzeroread{##1}}{}}%
%  \def\beamer@sortzeroread##1<##2>{}%
%  \def\beamer@finalnospec{}%
%}
%\makeatother
\usepackage{etoolbox}

% Definition %%%%%%%%%%%%%%%%%%%%%%%%%%%%%%%%%%%%%%%%%%%%%%%%%%%%%%%%
\definecolor{studentbrown}{RGB}{124,71,50}
\definecolor{lightblue}{RGB}{28,71,93}

\BeforeBeginEnvironment{block}{%
    \setbeamercolor{block title}{fg=white,bg=studentbrown}
    \setbeamercolor{block body}{fg=black, bg=studentbrown!20!white}
}
\AfterEndEnvironment{block}{
        \setbeamercolor{block title}{use=structure,fg=white,bg=structure.fg!75!black}
        \setbeamercolor{block body}{parent=normal text,use=block title,bg=block title.bg!10!bg}
}

\BeforeBeginEnvironment{exampleblock}{%
    \setbeamercolor{block title}{fg=white,bg=lightblue}
    \setbeamercolor{block body}{fg=black, bg=lightblue!20!white}
}
\AfterEndEnvironment{exampleblock}{
        \setbeamercolor{block title}{use=structure,fg=white,bg=structure.fg!75!black}
        \setbeamercolor{block body}{parent=normal text,use=block title,bg=block title.bg!10!bg}
}

\BeforeBeginEnvironment{definition}{%
    \setbeamercolor{block title}{fg=white,bg=studentbrown}
    \setbeamercolor{block body}{fg=black, bg=studentbrown!20!white}
}
\AfterEndEnvironment{definition}{
        \setbeamercolor{block title}{use=structure,fg=white,bg=structure.fg!75!black}
        \setbeamercolor{block body}{parent=normal text,use=block title,bg=block title.bg!10!bg}
}

% Theorem %%%%%%%%%%%%%%%%%%%%%%%%%%%%%%%%%%%%%%%%%%%%%%%%%%%%%%%%%%%%
\BeforeBeginEnvironment{theorem}{
    \setbeamercolor{block title}{use=example text,fg=white,bg=example text.fg!75!black}
    \setbeamercolor{block body}{parent=normal text,use=block title example,bg=block title example.bg!10!bg}
}
\AfterEndEnvironment{theorem}{
        \setbeamercolor{block title}{use=structure,fg=white,bg=structure.fg!75!black}
        \setbeamercolor{block body}{parent=normal text,use=block title,bg=block title.bg!10!bg}
}

% Proposition %%%%%%%%%%%%%%%%%%%%%%%%%%%%%%%%%%%%%%%%%%%%%%%%%%%%%%%%
\newtheorem{proposition}{Proposition}
\BeforeBeginEnvironment{proposition}{
        \setbeamercolor{block title}{use=alerted text,fg=white,bg=alerted text.fg!75!black}
        \setbeamercolor{block body}{parent=normal text,use=block title alerted,bg=block title alerted.bg!10!bg}
}
\AfterEndEnvironment{proposition}{
        \setbeamercolor{block title}{use=structure,fg=white,bg=structure.fg!75!black}
        \setbeamercolor{block body}{parent=normal text,use=block title,bg=block title.bg!10!bg}
}

%\newtheorem{ans}[theorem]{Answer}%%%%%%%%%%%%%%%
%\newtheorem{ans}{Answer}
%\BeforeBeginEnvironment{lemma}{
%        \setbeamercolor{block title}{use=alerted text,fg=white,bg=alerted text.fg!75!black}
%        \setbeamercolor{block body}{parent=normal text,use=block title alerted,bg=block title alerted.bg!10!bg}
%}
%\AfterEndEnvironment{ans}{
%        \setbeamercolor{block title}{use=structure,fg=white,bg=structure.fg!75!black}
%        \setbeamercolor{block body}{parent=normal text,use=block title,bg=block title.bg!10!bg}
%}

%\newtheorem{example}[theorem]{Example}%%%%%%%%%
%\newtheorem{example}{Example}
%\BeforeBeginEnvironment{example}{
%        \setbeamercolor{block title}{use=alerted text,fg=white,bg=alerted text.fg!75!black}
%        \setbeamercolor{block body}{parent=normal text,use=block title alerted,bg=block title alerted.bg!10!bg}
%}
%\AfterEndEnvironment{example}{
%        \setbeamercolor{block title}{use=structure,fg=white,bg=structure.fg!75!black}
%        \setbeamercolor{block body}{parent=normal text,use=block title,bg=block title.bg!10!bg}
%}

%\newtheorem{exercise}[theorem]{Exercise}%%%%%%%%%%%
\newtheorem{exercise}{Exercise}
\BeforeBeginEnvironment{exercise}{
        \setbeamercolor{block title}{use=alerted text,fg=white,bg=alerted text.fg!75!black}
        \setbeamercolor{block body}{parent=normal text,use=block title alerted,bg=block title alerted.bg!10!bg}
}
\AfterEndEnvironment{exercise}{
        \setbeamercolor{block title}{use=structure,fg=white,bg=structure.fg!75!black}
        \setbeamercolor{block body}{parent=normal text,use=block title,bg=block title.bg!10!bg}
}

%\newtheorem{convention}[theorem]{Convention}%%%%%%%%%%%%%%
\newtheorem{convention}{Convention}
\BeforeBeginEnvironment{convention}{
        \setbeamercolor{block title}{use=alerted text,fg=white,bg=alerted text.fg!75!black}
        \setbeamercolor{block body}{parent=normal text,use=block title alerted,bg=block title alerted.bg!10!bg}
}
\AfterEndEnvironment{convention}{
        \setbeamercolor{block title}{use=structure,fg=white,bg=structure.fg!75!black}
        \setbeamercolor{block body}{parent=normal text,use=block title,bg=block title.bg!10!bg}
}

%\newtheorem{statement}[theorem]{Statement}%%%%%%%%%%%%%%
\newtheorem{statement}{Statement}
\BeforeBeginEnvironment{statement}{
        \setbeamercolor{block title}{use=alerted text,fg=white,bg=alerted text.fg!75!black}
        \setbeamercolor{block body}{parent=normal text,use=block title alerted,bg=block title alerted.bg!10!bg}
}
\AfterEndEnvironment{statement}{
        \setbeamercolor{block title}{use=structure,fg=white,bg=structure.fg!75!black}
        \setbeamercolor{block body}{parent=normal text,use=block title,bg=block title.bg!10!bg}
}

%\newtheorem{fact}[theorem]{Fact}%%%%%%%%%%%%
%\newtheorem{fact}{Fact}
%\BeforeBeginEnvironment{fact}{
%        \setbeamercolor{block title}{use=alerted text,fg=white,bg=alerted text.fg!75!black}
%        \setbeamercolor{block body}{parent=normal text,use=block title alerted,bg=block title alerted.bg!10!bg}
%}
%\AfterEndEnvironment{fact}{
%        \setbeamercolor{block title}{use=structure,fg=white,bg=structure.fg!75!black}
%        \setbeamercolor{block body}{parent=normal text,use=block title,bg=block title.bg!10!bg}
%}

%\newtheorem{axiom}[theorem]{Axiom}%%%%%%%%%%%%%
\newtheorem{axiom}{Axiom}
\BeforeBeginEnvironment{axiom}{
        \setbeamercolor{block title}{use=alerted text,fg=white,bg=alerted text.fg!75!black}
        \setbeamercolor{block body}{parent=normal text,use=block title alerted,bg=block title alerted.bg!10!bg}
}
\AfterEndEnvironment{axiom}{
        \setbeamercolor{block title}{use=structure,fg=white,bg=structure.fg!75!black}
        \setbeamercolor{block body}{parent=normal text,use=block title,bg=block title.bg!10!bg}
}

%\newtheorem{problem}[theorem]{Problem}%%%%%%%%%%%%%%%%
%\newtheorem{problem}{Problem}
%\BeforeBeginEnvironment{problem}{
%        \setbeamercolor{block title}{use=alerted text,fg=white,bg=alerted text.fg!75!black}
%        \setbeamercolor{block body}{parent=normal text,use=block title alerted,bg=block title alerted.bg!10!bg}
%}
%\AfterEndEnvironment{problem}{
%        \setbeamercolor{block title}{use=structure,fg=white,bg=structure.fg!75!black}
%        \setbeamercolor{block body}{parent=normal text,use=block title,bg=block title.bg!10!bg}
%}

%\newtheorem{question}[theorem]{Q}%%%%%%%%%%%%%%%%%
\newtheorem{question}{Q}
\BeforeBeginEnvironment{question}{
        \setbeamercolor{block title}{use=alerted text,fg=white,bg=alerted text.fg!75!black}
        \setbeamercolor{block body}{parent=normal text,use=block title alerted,bg=block title alerted.bg!10!bg}
}
\AfterEndEnvironment{question}{
        \setbeamercolor{block title}{use=structure,fg=white,bg=structure.fg!75!black}
        \setbeamercolor{block body}{parent=normal text,use=block title,bg=block title.bg!10!bg}
}

\definecolor{cyanBlue}{rgb}{0.0, 1.0, 1.0}


\usepackage{longtable}
\usepackage[append]{beamersubframe}

\newcommand*{\mydprime}{^{\prime\prime}\mkern-1.2mu}
\newcommand*{\mytprime}{^{\prime\prime\prime}\mkern-1.2mu}

\usepackage{array}

\usepackage{setspace}


\usepackage{tabulary}
\newcolumntype{K}[1]{>{\centering\arraybackslash}p{#1}}

\usepackage{amsmath,lualatex-math}

\usepackage{fontspec}
%\usepackage{unicode-math}
%\usepackage{tgpagella}
\usepackage{multirow}

%\setmainfont{stix}
%\setmathfont{STIX}%[version=Stix]
% Any of the following work, and probably many more
%\setmathfont{TeX Gyre Pagella Math}
%\setmathfont{TeX Gyre Termes Math}
%\setmathfont{STIX}
%\setmathfont{Asana Math}

\usepackage{fbb}
\setmainfont{fbb}[%
UprightFeatures = {StylisticSet=01},
BoldFeatures = {StylisticSet=01}
]
\usepackage{bbm}

\usefonttheme{serif}
\usefonttheme{professionalfonts}
\usepackage[document]{ragged2e}
%\usepackage[scaled=0.85]{beramono}
%\usetheme[
%%% option passed to the outer theme
%    progressstyle=fixedCircCnt,   % fixedCircCnt, movingCircCnt (moving is deault)
%  ]{Feather}
  
% If you want to change the colors of the various elements in the theme, edit and uncomment the following lines
\setbeamercolor{item projected}{bg=red}
\setbeamertemplate{enumerate items}[default]
\setbeamertemplate{navigation symbols}{}

% Change the bar colors:
%\setbeamercolor{Feather}{fg=red!60!black,bg=red!80!black}

% Change the color of the structural elements:
%\setbeamercolor{structure}{fg=red!80!black}

% Change the frame title text color:
\setbeamercolor{frametitle}{fg=white}

% Change the normal text color background:
%\setbeamercolor{normal text}{fg=black,bg=gray!10}

% Have beamer use Figure numbers
\setbeamertemplate{caption}[numbered]

% Change the color of a block title
\setbeamercolor{block title}{fg=black,bg=red!50!white} 


%\setbeamertemplate{part page}
%{
%  \begin{centering}
%    {\usebeamerfont{part name}\usebeamercolor[fg]{part name}\partname~\insertpartnumber}
%    \vskip1em\par
%    \begin{beamercolorbox}[sep=16pt,center]{part title}
%      \usebeamerfont{part title}\insertpart\par
%    \end{beamercolorbox}
%  \end{centering}
%}


%-------------------------------------------------------
% INCLUDE PACKAGES
%-------------------------------------------------------

\usepackage[rflt]{floatflt}
\usepackage{tikz-cd}
\usepackage{caption}
\usepackage{pdfpc-commands}

\usepackage{biblatex}

%\theoremstyle{plain}
%\newtheorem{theorem}{Theorem}[section]
%\newtheorem{lemma}[theorem]{Lemma}
%\newtheorem{definition}[theorem]{Definition}
%\newtheorem{example}[theorem]{Example}
%\newtheorem{exercise}[theorem]{Exercise}
%\newtheorem{convention}[theorem]{Convention}
%\newtheorem{statement}[theorem]{Statement}
%\newtheorem{fact}[theorem]{Fact}
%\newtheorem{axiom}[theorem]{Axiom}
%\newtheorem{problem}[theorem]{Problem}
\theoremstyle{definition}
%\newtheorem{question}[theorem]{Q}
\newcommand{\im}{\operatorname{im}}
\newcommand{\lcm}{\operatorname{lcm}}
%\newcommand{\rank}{\operatorname{rank}}
\newcommand{\Var}{\operatorname{var}}

\usepackage{mathtools}

\usepackage{physics}
\usepackage{braket}
\usepackage{wrapfig}
\usepackage{pdfpages}

\usepackage{tikz}

\usetikzlibrary{external,automata,trees,positioning,shadows,arrows,shapes.geometric,shapes.misc,calc}
\usepackage{geometry}
%\usepackage[latin1]{inputenc}

%\usepackage{bigfoot}
%\DeclareNewFootnote{default}
%\DeclareNewFootnote{A}[alph]

\usepackage{commath}
\usepackage[step]{animate}

%\graphicspath{./pics/}

\usepackage{amsthm} % Required for theorem environments
\usepackage{bm} % Required for bold math symbols (used in the footer of the slides)
\usepackage{graphicx} % Required for including images in figures

\usepackage{booktabs} % Required for horizontal rules in tables
\usepackage{multicol} % Required for creating multiple columns in slides
\usepackage{lastpage} % For printing the total number of pages at the bottom of each slide
%\usepackage[english]{babel} % Document language - required for customizing section titles
\usepackage{microtype} % Better typography
%\usepackage{tocstyle} % Required for customizing the table of contents
\usepackage{gensymb} 

%\setbeameroption{show notes on second screen}

%\setbeamertemplate{note page}{\insertnote}


%\usepackage{hanging}% http://ctan.org/pkg/hanging
%\setbeamertemplate{footnote}{%
%  \hangpara{2em}{1}%
%  \makebox[2em][l]{\insertfootnotemark}\footnotesize\insertfootnotetext\par%
%}

%------------------------------------------------
\usepackage[utf8]{inputenc}
\usepackage[english]{babel}
\usepackage[T1]{fontenc}
\usepackage{helvet}
\usepackage{textcomp}%
%------------------------------------------------

\usepackage{cancel}
\newcommand\Ccancel[2][black]{\renewcommand\CancelColor{\color{#1}}\cancel{#2}}

\usepackage{csquotes}
\usepackage{url}
\usepackage{hyperref}
\hypersetup{
    colorlinks=true,
    linkcolor=blue,
    filecolor=magenta,      
    urlcolor=cyan,
}

\urlstyle{same}


\usepackage[framemethod=TikZ]{mdframed}
\mdfdefinestyle{MyFrame}{%
    linecolor=blue,
    outerlinewidth=1pt,
    roundcorner=3pt,
    innertopmargin=\baselineskip,
    innerbottommargin=\baselineskip,
    innerrightmargin=15pt,
    innerleftmargin=15pt,
    backgroundcolor=gray!50!white}


% Colors
\usepackage{color}
%\usetheme{default}
%\usetheme[secheader]{Boadilla}

\usepackage{xcolor}	 % Required for custom colors
% Define a few colors for making text stand out within the presentation
\definecolor{mygreen}{RGB}{44,85,17}
\definecolor{myblue}{RGB}{34,31,217}
\definecolor{mybrown}{RGB}{194,164,113}
\definecolor{myred}{RGB}{255,66,56}
\definecolor{babyblue}{RGB}{54,81,94}
%-------------------------------------------------------
% DEFFINING AND REDEFINING COMMANDS
%-------------------------------------------------------

%\newtheorem{thm}{Theorem}[section]
\newtheorem{define}{Definition}[theorem]
\newtheorem{pos}{Postulate}[theorem]

\raggedcolumns
\flushcolumns

\newcommand\setItemnumber[1]{\setcounter{enumi}{\numexpr#1-1\relax}}


%\setbeamertemplate{footline}[frame number]
%\setbeamerfont{footline}{size=\normalsize, series=\bfseries}
%\setbeamercolor{footline}{fg = black, bg = black}
%\setbeamercolor{page number in head/foot}{fg = black}


% colored hyperlinks
\newcommand{\chref}[2]{
  \href{#1}{{\usebeamercolor[bg]{Feather}#2}}
}
% Use these colors within the presentation by enclosing text in the commands below
\newcommand*{\mygreen}[1]{\textcolor{mygreen}{#1}}
\newcommand*{\myblue}[1]{\textcolor{myblue}{#1}}
\newcommand*{\mybrown}[1]{\textcolor{mybrown}{#1}}
\newcommand*{\myred}[1]{\textcolor{myred}{#1}}

\newcommand{\mystar}{{\fontfamily{lmr}\selectfont$\star$}}

%\setbeamertemplate{enumerate subitem}[circle]%
\renewcommand{\insertsubenumlabel}{\alph{enumii}}

\newcommand{\makepart}[1]{ % For convenience
\part{Title of part #1} \frame{\partpage}
\section{Section} \begin{frame} Section \end{frame}
\subsection{Subsection} \begin{frame} Subsection \end{frame}
\subsection{Subsection} \begin{frame} Subsection \end{frame}
\section{Section} \begin{frame} Section \end{frame}
}

\newcommand*{\img}[1]{%
    \raisebox{-.3\baselineskip}{%
        \includegraphics[
        height=\baselineskip,
        width=\baselineskip,
        keepaspectratio,
        ]{#1}%
    }%
}

\PassOptionsToPackage{unicode}{hyperref}
\PassOptionsToPackage{naturalnames}{hyperref}

\makeatletter
\newcommand\immaddtocontents[1]{{%
   \let\protect\@unexpandable@protect
   \immediate\write\@auxout{\noexpand\@writefile{toc}{#1}}%
}}
\makeatother
\newcounter{nameOfYourChoice}


