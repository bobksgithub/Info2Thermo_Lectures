\documentclass[12pt,a4paper]{article}
\usepackage{amssymb,amsmath,amsthm}
\theoremstyle{plain}
\newtheorem{theorem}{Theorem}[section]
\newtheorem{lemma}[theorem]{Lemma}
\newtheorem{definition}[theorem]{Definition}
\newtheorem{example}[theorem]{Example}
\newtheorem{exercise}[theorem]{Exercise}
\newtheorem{convention}[theorem]{Convention}
\newtheorem{statement}[theorem]{Statement}
\newtheorem{fact}[theorem]{Fact}
\newtheorem{axiom}[theorem]{Axiom}
\newtheorem{problem}[theorem]{Problem}
\theoremstyle{definition}
\newtheorem{question}[theorem]{Q}
\newcommand{\im}{\operatorname{im}}
\newcommand{\lcm}{\operatorname{lcm}}
\newcommand{\rank}{\operatorname{rank}}
\newcommand{\Var}{\operatorname{var}}
\begin{document}
\title{Coding and Cryptography}
\author{T.~W.~K\"{o}rner}
\maketitle
\emph{Transmitting messages is an important
practical problem. Coding theory includes the
study of compression codes which enable
us to send messages cheaply and error correcting codes 
which ensure
that messages remain legible even in the
presence of errors. Cryptography on the other
hand, makes sure that messages remain unreadable --- except
to the intended recipient. These
techniques turn out to have much in common.}

\emph{Many Part~II courses go deeply into one topic so that
you need to understand the whole course before you understand
any part of it. They often require a firm grasp 
of some preceding course. Although this course has an underlying 
theme, it splits into parts which can be understood separately
and, although it does require knowledge from various earlier
courses, it does not require mastery of that knowledge.
All that is needed is a little probability, a little
algebra and a fair amount of common sense. On the other hand,
the variety of techniques and ideas probably makes
it harder to understand \emph{everything} in the course
than in a more monolithic course.}  

\vspace{1\baselineskip}

\begin{footnotesize}
\noindent
{\bf Small print}
The syllabus for the course is defined by
the Faculty Board Schedules (which are minimal for lecturing
and maximal for examining).
I should {\bf very much} appreciate being told
of any corrections or possible improvements
{\bf however minor}. 
This document
is written in \LaTeX2e and should be available from my
home page
\begin{center}
{\sf http://www.dpmms.cam.ac.uk/\textasciitilde twk}
\end{center}
in latex, dvi, ps and pdf formats.
Supervisors can obtain 
comments on the exercises at the end of these notes
from the secretaries in DPMMS or by e-mail from me.
\end{footnotesize}

\begin{footnotesize}
My e-mail address is \verb+twk@dpmms+.
\end{footnotesize}

\begin{footnotesize}
These notes are based on notes taken in the course
of a previous lecturer Dr Pinch, on the excellent
set of notes available from Dr Carne's home page
and on Dr Fisher's collection of examples. 
Dr Parker and Dr Lawther
produced two very useful list of corrections.
Any credit
for these notes belongs to them, any discredit to me.
This is a course outline. A few proofs are included
or sketched in these notes but most are omitted.
Please note that vectors are \emph{row} vectors
unless otherwise stated.
\end{footnotesize}
\newpage
\tableofcontents
\section{Codes and alphabets}\label{S;alphabets}
Originally, a code was a device for making messages
hard to read. The study of such codes and their
successors is called cryptography and will form
the subject of the last quarter of these notes.
However, in the 19th century the
optical\footnote{See \emph{The Count of Monte Cristo}
and various Napoleonic sea stories.
A statue to the inventor of the optical telegraph
(semaphore)
was put up in Paris in 1893 but melted down during
World War II and not replaced 
(http://hamradio.nikhef.nl/tech/rtty/chappe/).
In the parallel universe of Disc World the 
\emph{clacks} is one of the wonders of the
Century of the Anchovy.} and
then the electrical telegraph made it
possible to send messages speedily, but
only after they had been translated from ordinary written
English or French into a string of symbols.

The best known of the early codes is the Morse code
used in electronic telegraphy.
We think of it as consisting of dots and 
dashes but, in fact, it had three symbols dot, dash and pause
which we write as $\bullet$, $-$ and $*$.
Morse assigned a \emph{code word} consisting of a 
sequence of symbols to each of the letters of the alphabet
and each digit. Here are some typical examples.
\begin{align*}
A\mapsto \bullet-*\qquad
&&B\mapsto -\bullet\bullet\bullet*\qquad
&&C\mapsto-\bullet-\bullet*\\
D\mapsto -\bullet\bullet*\qquad
&&E\mapsto \bullet*\qquad
&&F\mapsto\bullet\bullet-\bullet*\\
O\mapsto ---*\qquad
&&S\mapsto\bullet\bullet\bullet*\qquad
&&7\mapsto--\bullet\bullet\bullet*
\end{align*}
The symbols of the original message would be \emph{encoded}
and the code words sent in sequence, as in
\[SOS\mapsto
\bullet\bullet\bullet*---*\bullet\bullet\bullet*,\]
and then decoded in sequence at the other end
to recreate the original message.
\begin{exercise}\label{E;Morse}
Decode
$-\bullet-\bullet*---*-\bullet\bullet* \bullet*$.
\end{exercise}
{\ss}

Morse's system was intended for human beings.
Once machines took over the business of encoding, 
other systems developed. A very influential
one called ASCII was developed in the 1960s.
This uses two symbols $0$ and $1$
and all code words have seven symbols.
In principle, this would give $128$ possibilities, but $0000000$
and $1111111$ are not used, so there are $126$ code words
allowing the original message to contain a greater
variety of symbols than Morse code. 
Here are some typical examples
\begin{align*}
A\mapsto 1000001\qquad
&&B\mapsto 1000010\qquad
&&C\mapsto 1000011\\
a\mapsto 1100001\qquad
&&b\mapsto 1100010 \qquad
&&c\mapsto 1100011\\
+\mapsto 0101011\qquad
&&!\mapsto 0100001\qquad
&&7\mapsto 0110111
\end{align*}
\begin{exercise}\label{E;ASCII} 
Encode $b7!$. Decode $110001111000011100010$.
\end{exercise}

More generally, we have two alphabets ${\mathcal A}$
and ${\mathcal B}$ 
and a coding function $c:{\mathcal A}\rightarrow{\mathcal B}^{*}$
where ${\mathcal B}^{*}$ consists of all finite sequences of 
elements of  ${\mathcal B}$.
If  ${\mathcal A}^{*}$ consists of all finite sequences 
of elements of  ${\mathcal A}$,
then the encoding function 
$c^{*}:{\mathcal A}^{*}\rightarrow{\mathcal B}^{*}$
is given by
\[c^{*}(a_{1}a_{2}\ldots a_{n})=c(a_{1})c(a_{2})\dots c(a_{n}).\]
We demand that $c^{*}$ is injective, since otherwise
it is possible to produce two
messages which become indistinguishable once encoded.

We call codes for which $c^{*}$ is injective \emph{decodable}.

For many purposes, we are more interested
in the collection of code words ${\mathcal C}=c({\mathcal A})$
than the coding function $c$. If we look
at the code words of Morse code and the ASCII code, we
observe a very important difference.
All the code words in ASCII have the same length 
(so we have a \emph{fixed length} code),
but this is not true for the Morse code
(so we have a \emph{variable length} code).
\begin{exercise}\label{E;fix length} Explain why (if $c$ is injective) 
any fixed length code
is  decodable.
\end{exercise}

A variable length code need not be decodable even 
if $c$ is injective.
\begin{exercise}\label{E;decodable} 
(i) Let ${\mathcal A}={\mathcal B}=\{0,1\}$. If
$c(0)=0$, $c(1)=00$ show that $c$ is injective but $c^{*}$ is not.
 
(ii) Let ${\mathcal A}=\{1,2,3,4,5,6\}$ and ${\mathcal B}=\{0,1\}$.
Show that there is a variable length coding $c$ such that
$c$ is injective and all code words have length $2$ or less.
Show that there is no decodable coding $c$ such that
all code words have length $2$ or less.
\end{exercise}

However, there is a family of variable length codes
which are decodable in a natural way.
\begin{definition} Let ${\mathcal B}$ be an alphabet. We say that
a finite subset ${\mathcal C}$ of ${\mathcal B}^{*}$
is \emph{prefix-free} if, 
whenever $w\in{\mathcal C}$ is an initial sequence
of $w'\in{\mathcal C}$, then $w=w'$. 
If $c:{\mathcal A}\rightarrow{\mathcal B}^{*}$
is a coding function, we say that $c$ is prefix-free if $c$ is injective
and $c({\mathcal A})$
is prefix-free.
\end{definition}

If $c$ is prefix-free, then, not only is $c^{*}$ injective, but we can decode
messages on the fly. Suppose that we receive a sequence 
$b_{1}$, $b_{2}$, \dots.
The moment we have received some $c(a_{1})$, we know that 
the first message was $a_{1}$
and we can proceed to look for the second message. (For this reason
prefix-free codes
are sometimes called instantaneous codes or self punctuation codes.)
\begin{exercise} Let  ${\mathcal A}=\{0,1,2,3\}$, ${\mathcal B}=\{0,1\}$.
If $c,\tilde{c}:{\mathcal A}\rightarrow{\mathcal B}^{*}$ are given by
\begin{align*}
c(0)&=0&&\tilde{c}(0)=0\\
c(1)&=10&&\tilde{c}(1)=01\\
c(2)&=110&&\tilde{c}(2)=011\\
c(3)&=111&&\tilde{c}(3)=111
\end{align*}
show that $c$ is prefix-free, but $\tilde{c}$ is not.
By thinking about the way $\tilde{c}$ is obtained from $c$, or otherwise,
show that $\tilde{c}^{*}$ is injective.
\end{exercise}
\begin{exercise}\label{E;auto free} 
Why is every injective fixed
length code automatically prefix-free?
\end{exercise}
From now on, unless explicitly stated otherwise, $c$ will be injective
and the codes used will be prefix-free. In section~\ref{S;prefix} 
we show
that we lose nothing by confining ourselves to prefix-free codes.
\section{Huffman's algorithm} An electric telegraph is expensive 
to build and maintain. However good a telegraphist was,
he could only send or receive a limited number of dots and dashes
each minute. (This is why  Morse chose a variable length code.
The telegraphist would need to send the letter $E$ far more often than
the letter $Q$ so Morse gave $E$ the short
code $\bullet*$ and $Q$ the long code $--\bullet-*$.)
It is possible to increase the rate at which symbols are
sent and received by using machines, but the laws of
physics (backed up by results in Fourier analysis)
place limits on the number of symbols that can be
correctly transmitted over a given line.
(The slowest rates were associated with undersea cables.)

Customers were therefore charged so much
a letter or, more usually, so 
much a word\footnote{Leading to a
prose style known as telegraphese.
`Arrived Venice. Streets flooded. Advise.'}
(with a limit
on the permitted word length).
Obviously it made sense to have books
of `telegraph codes' in which one five letter
combination, say, `FTCGI'
meant `are you willing to split the difference?'
and  another `FTCSU' meant
`cannot see any difference'\footnote{If the telegraph
company insisted on ordinary words you got
codes like `FLIRT' for `quality of crop good'.
Google `telegraphic codes and message practice, 1870-1945'
for lots of examples.}.
 
Today messages are usually sent in
as binary sequences like $01110010\dots$,
but the transmission of each digit still costs
money. If we know that there are $n$ possible
messages that can be sent and that $n\leq 2^{m}$,
then we can assign each message a different
string of $m$ zeros and ones (usually called \emph{bits})
and each message will cost $mK$ cents where $K$
is the cost of sending one bit.

However, this may not be the best way of saving money.
If, as  often happens, one message 
(such as `nothing to report') is much more frequent
than any other then it may be cheaper \emph{on average}
to assign it a shorter code word even at the cost
of lengthening the other code words.
\begin{problem}\label{P;Compress one}
Given $n$ messages $M_{1}$, $M_{2}$,
\dots, $M_{n}$ such that the probability that $M_{j}$
will be chosen is $p_{j}$, find distinct code words $C_{j}$
consisting of $l_{j}$ bits so that the expected cost
\[K\sum_{j=1}^{n}p_{j}l_{j}\]
of sending the code word corresponding to the chosen
message is minimised.
\end{problem}
Of course, we suppose $K>0$.

The problem is interesting as it stands, but we
have not taken into account the fact that a variable
length code may not be decodable. To deal with this
problem we add an extra constraint.
\begin{problem}\label{P;Compress}
Given $n$ messages $M_{1}$, $M_{2}$,
\dots, $M_{n}$ such that the probability that $M_{j}$
will be chosen is $p_{j}$, find a prefix-free
collection of code words $C_{j}$
consisting of $l_{j}$ bits so that the expected cost
\[K\sum_{j=1}^{n}p_{j}l_{j}\]
of sending the code word corresponding to the chosen
message is minimised.
\end{problem}


In 1951 Huffman was asked to write an essay on this problem
as an end of term university exam. Instead of writing about the problem,
he solved it completely.
\begin{theorem}\label{T;Huffman}{\bf [Huffman's algorithm]} 
The following algorithm
solves Problem~\ref{P;Compress} with $n$ messages.
Order the messages so that $p_{1}\geq p_{2}\geq \dots\geq p_{n}$.
Solve the problem with $n-1$ messages $M_{1}'$, $M_{2}'$,
\dots, $M_{n-1}'$ such that $M_{j}'$ has probability $p_{j}$
for $1\leq j\leq n-2$, but $M_{n-1}'$ has probability
$p_{n-1}+p_{n}$. If $C_{j}'$ is the code word corresponding
to $M_{j}'$, the original problem is solved by assigning $M_{j}$
the code word $C_{j}'$ for $1\leq j\leq n-2$ and
$M_{n-1}$ the code word consisting of $C_{n-1}'$ followed
by $0$ and $M_{n}$ the code word consisting of $C_{n-1}'$ 
followed by $1$.
\end{theorem}
Since the problem is trivial when $n=2$ (give $M_{1}$ the code word $0$
and $M_{2}$ the code word $1$) this gives us what computer programmers
and logicians call a \emph{recursive solution}.

Recursive programs are often 
better adapted to machines 
than human beings, but it is very easy
to follow the steps of Huffman's algorithm `by hand'.
%(Note that the algorithm is very specific about
%the labelling of the code words so that, for example,
%message $1$.)

\begin{example}\label{E;Huffman do} 
Suppose $n=4$, $M_{j}$ has probability $j/10$ 
for $1\leq j\leq 4$. Apply Huffman's algorithm.
\end{example}
\begin{proof}[Solution]
(Note that we do not bother to reorder messages.)
Combining messages in the suggested way, we get
\begin{gather*}
1,2,3,4\\
[1,2],3,4\\
[[1,2],3],4.
\end{gather*}
Working backwards, we get
\begin{gather*}
C_{[[1,2],3]}=0\ldots,\ C_{4}=1\\
C_{[1,2]}=01\dots,\ C_{3}=00\\
C_{1}=011,\ C_{2}=010.
\end{gather*}
\end{proof}
The reader is strongly advised to do a slightly more
complicated example like the next.
\begin{exercise}\label{E;Huffman 1} 
Suppose $M_{j}$ has probability $j/45$ 
for $1\leq j\leq 9$. Apply Huffman's algorithm.
\end{exercise}

As we indicated earlier, the effects of Huffman's algorithm 
will be most marked when a few messages are highly
probable.
\begin{exercise}\label{E;Huffman 2} Suppose $n=64$,
$M_{1}$ has probability $1/2$,
$M_{2}$ has probability $1/4$ and $M_{j}$ has probability
$1/248$ for $3\leq j\leq 64$.
Explain why, if we use code words of equal length
then the length of a code word must be at least $6$.
By using the ideas of Huffman's algorithm (you should not
need to go through all the steps) obtain a set of
code words such that the \emph{expected} length of a code word
sent is not more than $3$.
\end{exercise}

Whilst doing the exercises the reader must already
have been struck by the fact that minor variations
in the algorithm produce different codes. (Note, for example
that, if we have a Huffman code, then interchanging the role of
$0$ and $1$ will produce another Huffman type code.)
In fact, although the Huffman algorithm will always
produce a best code (in the sense of Problem~\ref{P;Compress}),
there may be other equally good codes which could not
be obtained in this manner.
\begin{exercise}\label{E;Huffman 3} Suppose $n=4$,
$M_{1}$ has probability $.23$,
$M_{2}$ has probability $.24$, 
$M_{3}$ has probability $.26$
and $M_{4}$ has probability $.27$. Show that any 
assignment of the code words $00$, $01$, $10$
and $11$ produces a best code in the sense of Problem~\ref{P;Compress}.
\end{exercise}

The fact that the Huffman code may not be the unique
best solution means that we need to approach the proof of 
Theorem~\ref{T;Huffman} with caution. We observe that
reading a code word from a prefix-free
code is like climbing a tree
with $0$ telling us to take the left branch and 
$1$ the right branch. The fact that the code is prefix-free
tells us that each code word
may be represented by a leaf at the end of
a final branch. 
Thus, for example,
the code word $00101$ is represented by
the leaf found by following left branch, left branch,
right branch, left branch, right branch.
The next lemma contains the essence of our proof
of Theorem~\ref{T;Huffman}.
\begin{lemma}\label{L;pre Huffman}
(i) If we have a best code then it will split into
a left branch and right branch at every stage.

(ii) If we label every branch by the sum of the probabilities 
of all the leaves that spring from it then, if we have a best code, 
every branch belonging to a particular stage of growth will have at
least as large a number associated with it
as any branch belonging to a later stage.

(iii) If we have a best code then interchanging the 
probabilities of leaves belonging to the last stage
(ie the longest code words) still gives a best code.

(iv)  If we have a best code then two of the
leaves with the lowest probabilities will appear at the last stage.

(v) There is a best code in which two of the
leaves with the lowest probabilities are neighbours
(have code words differing only in the last place).
\end{lemma}

In order to use the Huffman algorithm we need to 
know the probabilities of the $n$ possible messages.
Suppose we do not. After we have sent $k$ messages
we will know that message $M_{j}$ has been sent
$k_{j}$ times and so will the recipient of the message.
If we decide to use a Huffman code for the next message,
it is not unreasonable (lifting our hat
in the direction of the Reverend Thomas Bayes) to take
\[p_{j}=\frac{k_{j}+1}{k+n}.\]
Provided the recipient knows the exact version of
the Huffman algorithm that we use, she can
reconstruct our Huffman code and decode our next message.
Variants of this idea are  known as `Huffman-on-the-fly'  
and form the basis of the kind of compression programs
used in your computer. Notice however, that whilst
Theorem~\ref{T;Huffman} is an examinable theorem, 
the contents of this paragraph form a non-examinable 
plausible statement.
\section{More on prefix-free codes}\label{S;prefix} 
It might be thought that
Huffman's algorithm says all that is to be said on
the problem it addresses. However, there are two important
points that need to be considered. The first is whether
we could get better results by using codes which
are not prefix-free. The object of this section is
to show that this is not the case.

As in section~\ref{S;alphabets},
we consider two alphabets ${\mathcal A}$
and ${\mathcal B}$ 
and a coding function $c:{\mathcal A}\rightarrow{\mathcal B}^{*}$
(where, as we said earlier, 
${\mathcal B}^{*}$ consists of all finite sequences of 
elements of  ${\mathcal B}$).  For most of this
course ${\mathcal B}=\{0,1\}$, but in this section
we allow ${\mathcal B}$ to have $D$ elements.
The elements of ${\mathcal B}^{*}$ are called \emph{words}.

\begin{lemma} {\bf [Kraft's inequality 1]}\label{Kraft 1}
If a prefix-free code ${\mathcal C}$ consists
of $n$ words $C_{j}$ of length $l_{j}$, then
\[\sum_{j=1}^{n}D^{-l_{j}}\leq 1.\]
\end{lemma}
\begin{lemma} {\bf [Kraft's inequality 2]}\label{L;Kraft 2}
Given strictly positive integers $l_{j}$ satisfying
\[\sum_{j=1}^{n}D^{-l_{j}}\leq 1,\]
we can find a prefix-free code ${\mathcal C}$ consisting
of $n$ words $C_{j}$ of length $l_{j}$.
\end{lemma}
\begin{proof} Take $l_{1}\leq l_{2}\leq\dots\leq l_{n}$.
We give an inductive construction for an appropriate
prefix-free code. Start by choosing $C_{1}$ to be
any  code word  of length $l_{1}$.

Suppose that we have found a collection of $r$ prefix-free
code words $C_{k}$ of length $l_{k}$ $[1\leq k\leq r]$.
If $r=n$ we are done. If not, consider all 
possible code words
of length $l_{r+1}$. Of these $D^{l_{r+1}-l_{k}}$ will
have prefix $C_{k}$ so at most (in fact, exactly)
\[\sum_{k=1}^{r}D^{l_{r+1}-l_{k}}\]
will have one of the code words already selected as prefix.
By hypothesis
\[\sum_{k=1}^{r}D^{l_{r+1}-l_{k}}
=D^{l_{r+1}}\sum_{k=1}^{r}D^{-l_{k}}<D^{l_{r+1}}.\]
Since there are $D^{l_{r+1}}$ possible code words of length 
$l_{r+1}$ there is at least one `good code word'
which does not
have one of the code words already selected as prefix.
Choose one of the good code words as $C_{r+1}$
and restart the induction.
\end{proof}
The method used in the proof is called a `greedy algorithm'
because we just try to do the best we can at each stage
without considering future consequences.


Lemma~\ref{Kraft 1} is pretty but not deep.
MacMillan showed that the same inequality
applies to all decodable codes. The proof is extremely 
elegant and (after one has thought about it
long enough) natural.
\begin{theorem}\label{T;MacMillan}{\bf [The MacMillan inequality]}
If a decodable code ${\mathcal C}$ consists
of $n$ words $C_{j}$ of length $l_{j}$, then
\[\sum_{j=1}^{n}D^{-l_{j}}\leq 1.\]
\end{theorem}
Using Lemma~\ref{L;Kraft 2} we get the immediate corollary.
\begin{lemma} If there exists a decodable code ${\mathcal C}$ 
consisting 
of $n$ words $C_{j}$ of length $l_{j}$, then
there exists a prefix-free code ${\mathcal C}'$ 
consisting 
of $n$ words $C_{j}'$ of length $l_{j}$.
\end{lemma}
Thus if we are only concerned with the length of code words
we need only consider prefix-free codes.
\section{Shannon's noiseless coding theorem}
In the previous section we indicated that there
was a second question we should ask about Huffman's
algorithm. We know that Huffman's algorithm is
best possible, but we have not discussed how good the
best possible should be.

Let us restate our problem. (In this section we allow
the coding alphabet ${\mathcal B}$ to have $D$ elements.) 
\begin{problem}\label{P;Compress two}
Given $n$ messages $M_{1}$, $M_{2}$,
\dots, $M_{n}$ such that the probability that $M_{j}$
will be chosen is $p_{j}$, find a decodable
code ${\mathcal C}$ whose code words  $C_{j}$
consist of $l_{j}$ bits so that the expected cost
\[K\sum_{j=1}^{n}p_{j}l_{j}\]
of sending the code word corresponding to the chosen
message is minimised.
\end{problem}
In view of Lemma~\ref{L;Kraft 2}
(any system of lengths
satisfying Kraft's inequality is associated  
with a prefix-free and so decodable code)
and Theorem~\ref{T;MacMillan} (any decodable
code satisfies Kraft's inequality),
Problem~\ref{P;Compress two} reduces to
an abstract minimising problem.
\begin{problem}\label{P;Compress three}
Suppose $p_{j}\geq 0$ for $1\leq j\leq n$ and $\sum_{j=1}^{n}p_{j}=1$.
Find strictly positive integers $l_{j}$  minimising
\[\sum_{j=1}^{n}p_{j}l_{j}\ \text{subject to} 
\ \sum_{j=1}^{n}D^{-l_{j}}\leq 1.\]
\end{problem}

Problem~\ref{P;Compress three} is hard because
we restrict the $l_{j}$ to be integers.
If we drop the restriction we end up with a problem
in Part~IB variational calculus.
\begin{problem}\label{P;Compress variation}
Suppose $p_{j}\geq 0$ for $1\leq j\leq n$ and $\sum_{j=1}^{n}p_{j}=1$.
Find strictly positive real numbers $x_{j}$ minimising
\[\sum_{j=1}^{n}p_{j}x_{j}\ \text{subject to} 
\ \sum_{j=1}^{n}D^{-x_{j}}\leq 1.\]
\end{problem}
\begin{proof}[Calculus solution] Observe 
that decreasing any $x_{k}$
decreases $\sum_{j=1}^{n}p_{j}x_{j}$
and increases $\sum_{j=1}^{n}D^{-x_{j}}$. Thus we may demand
\[\sum_{j=1}^{n}D^{-x_{j}}=1.\]
The Lagrangian is
\[L({\mathbf x},\lambda)=\sum_{j=1}^{n}p_{j}x_{j}
-\lambda\sum_{j=1}^{n}D^{-x_{j}}.\]
Since
\[\frac{\partial L}{\partial x_{j}}
=p_{j}+(\lambda\log D)D^{-x_{j}}\]
we know that, at any stationary point,
\[D^{-x_{j}}=K_{0}(\lambda)p_{j}\]
for some $K_{0}(\lambda)>0$. Since $\sum_{j=1}^{n}D^{-x_{j}}=1$,
our original problem will have a \emph{stationarising}
solution when
\[D^{-x_{j}}=p_{j},\ \text{that is to say}
\ x_{j}=-\frac{\log p_{j}}{\log D}\]
and
\[\sum_{j=1}^{n}p_{j}x_{j}=
-\sum_{j=1}^{n}p_{j}\frac{\log p_{j}}{\log D}.\]
\end{proof}
It is not hard to convince oneself that the  stationarising
solution just found is, in fact, maximising, but it is an
unfortunate fact that IB variational calculus is suggestive
rather than conclusive. 

The next two exercises 
(which will be done in lectures and form part of the course) 
provide a rigorous proof.
\begin{exercise}\label{E;Gibbs} (i) Show that
\[\log t\leq t-1\]
for $t>0$ with equality if and only if $t=1$.

(ii) {\bf [Gibbs' inequality]}
Suppose that $p_{j},q_{j}>0$ and 
\[\sum_{j=1}^{n}p_{j}=\sum_{j=1}^{n}q_{j}=1.\] 
By applying~(i) with $t=q_{j}/p_{j}$, show that
\[\sum_{j=1}^{n}p_{j}\log p_{j}\geq \sum_{j=1}^{n}p_{j}\log q_{j}\]
with equality if and only if $p_{j}=q_{j}$. 
\end{exercise}
\begin{exercise} We use the notation of 
Problem~\ref{P;Compress variation}.

(i) Show that, if $x_{j}^{*}=-\log p_{j}/\log D$, then
$x_{j}^{*}>0$ and
\[\sum_{j=1}^{n}D^{-x_{j}^{*}}=1.\]

(ii) Suppose that $y_{j}>0$ and
\[\sum_{j=1}^{n}D^{-y_{j}}=1.\]
Set $q_{j}=D^{-y_{j}}$. By using Gibbs' inequality
from Exercise~\ref{E;Gibbs}~(ii), show that
\[\sum_{j=1}^{n}p_{j}x_{j}^{*}\leq \sum_{j=1}^{n}p_{j}y_{j}\]
with equality if and only if $y_{j}=x_{j}^{*}$ for all $j$.
\end{exercise}

Analysts use logarithms to the base $e$, but the importance
of two-symbol alphabets means that communication theorists
often use logarithms to the base $2$.
\begin{exercise}\label{E;Memory} 
(Memory jogger.) Let $a,\,b>0$. Show that
\[\log_{a} b=\frac{\log b}{\log a}.\]
\end{exercise}
The result of Problem~\ref{P;Compress variation}
is so important that it gives rise to a definition.
\begin{definition}\label{D;Shannon entropy} 
Let ${\mathcal A}$ be a 
non-empty finite set and $A$ a random variable taking
values in ${\mathcal A}$. If $A$ takes the value $a$
with probability $p_{a}$ we say that the system has 
\emph{Shannon entropy}\footnote{It is unwise for the beginner
and may or may not be fruitless for the expert
to seek a link with entropy in physics.} 
(or \emph{information entropy})
\[H(A)=-\sum_{a\in {\mathcal A}}p_{a}\log_{2} p_{a}.\]
\end{definition}
\begin{theorem}\label{T;no better} 
Let ${\mathcal A}$ and ${\mathcal B}$
be finite alphabets and let ${\mathcal B}$ have $D$ symbols. If 
$A$ is an ${\mathcal A}$-valued random variable, 
then any decodable code $c:{\mathcal A}\rightarrow{\mathcal B}$
must satisfy 
\[{\mathbb E}|c(A)|\geq \frac{H(A)}{\log_{2} D}.\]
\end{theorem}
Here $|c(A)|$ denotes the length of $c(A)$.
Notice that the result takes a particularly
simple form when $D=2$. 

In Problem~\ref{P;Compress variation}
the $x_{j}$ are just positive real numbers but in
Problem~\ref{P;Compress three} the $d_{j}$ are integers.
Choosing $d_{j}$ as close as possible to the best $x_{j}$
may not give the best $d_{j}$, but it is certainly worth a try.
\begin{theorem}\label{T;Shannon--Fano}{\bf[Shannon--Fano encoding]} 
Let ${\mathcal A}$ and ${\mathcal B}$
be finite alphabets and let ${\mathcal B}$ have $D$ symbols. If 
$A$ is an ${\mathcal A}$-valued random variable, 
then there exists
a prefix-free (so decodable) code 
$c:{\mathcal A}\rightarrow{\mathcal B}$
which satisfies 
\[{\mathbb E}|c(A)|\leq 1+\frac{H(A)}{\log_{2} D}.\]
\end{theorem}
\begin{proof} By  Lemma~\ref{L;Kraft 2} (which states
that given lengths satisfying
Kraft's inequality we can construct an associated
prefix-free code), it suffices to find strictly
positive integers $l_{a}$ such that
\[\sum_{a\in {\mathcal A}}D^{-l_{a}}\leq 1,
\ \text{but}
\ \sum_{a\in {\mathcal A}}p_{a}l_{a}\leq 1+\frac{H(A)}{\log_{2} D}.\]
If we take
\[l_{a}=\lceil -\log_{D} p_{a}\rceil,\]
that is to say, we take $l_{a}$ to be the smallest integer
no smaller than $-\log_{D} p_{a}$, then these conditions 
are satisfied and we are done.
\end{proof}
It is very easy to use the method just indicated to
find an appropriate code. (Such codes are called Shannon--Fano
codes\footnote{Wikipedia and several other sources give
a definition of Shannon--Fano codes which is 
definitely \emph{inconsistent} with that given here.
Within a Cambridge examination context you may assume
that Shannon--Fano codes are those considered here.}.
Fano was the professor who set the homework for Huffman.
The point of view adopted here means that for some
problems there may be more than one Shannon--Fano
code.)

\begin{exercise}\label{E;Fano}
(i) Let ${\mathcal A}=\{1,2,3,4\}$. Suppose that the
probability
that letter $k$ is chosen is $k/10$.
Observing that $\log 2^{-n}=-n$
write down an appropriate Shannon--Fano code $c$.

(ii) We found a Huffman code $c_{h}$ for the system
in Example~\ref{E;Huffman do}. 
Show\footnote {Unless you are on a desert island
in which case the calculations are rather tedious.} 
that the entropy is approximately $1.85$,
that ${\mathbb E}|c(A)|=2.4$
and that  ${\mathbb E}|c_{h}(A)|=1.9$.
Check that these results are consistent with 
our previous theorems.
\end{exercise}
Putting Theorems~\ref{T;no better} and Theorem~\ref{T;Shannon--Fano}
together,
we get the following remarkable result.
\begin{theorem}\label{T;Shannon noiseless}%
{\bf[Shannon's noiseless coding theorem]} 
Let ${\mathcal A}$ and ${\mathcal B}$
be finite alphabets and let ${\mathcal B}$ have $D$ symbols. If 
$A$ is a ${\mathcal A}$-valued random variable, 
then any decodable code $c$ which minimises ${\mathbb E}|c(A)|$
satisfies
\[\frac{H(A)}{\log_{2} D}\leq
{\mathbb E}|c(A)|\leq 1+\frac{H(A)}{\log_{2} D}.\]
\end{theorem}
In particular,  Huffman's code $c_{h}$ for two symbols satisfies
 \[H(A)\leq
{\mathbb E}|c_{h}(A)|\leq 1+H(A).\]
\begin{exercise} (i) Sketch $h(t)=-t\log t$ for $0\leq t\leq 1$.
(We define $h(0)=0$.)

(ii) Let
\[\Gamma=\left\{{\mathbf p}\in{\mathbb R}^{n}
\,:\,p_{j}\geq 0,\ \sum_{j=1}^{n}p_{j}=1\right\}\]
and let $H:\Gamma\rightarrow{\mathbb R}$ be defined by
\[H({\mathbf p})=\sum_{j=1}^{n}h(p_{j}).\]
Find the maximum and minimum of $H$ and 
describe the points where these values are attained.

(iii) If $n=2^{r}+s$ with $0\leq s< 2^{r}$ and $p_{j}=1/n$,
describe the Huffman code $c_{h}$ for two symbols
and verify directly that (with notation
of Theorem~\ref{T;Shannon noiseless}) 
\[H(A)\leq
{\mathbb E}|c_{h}(A)|\leq 1+H(A).\]
\end{exercise}
Waving our hands about wildly, we may say that
`A system with low Shannon entropy is highly
organised and, knowing the system, it is
usually quite easy to identify an individual from the
system'.
\begin{exercise} The notorious Trinity gang 
has just been rounded up and
Trubshaw of the Yard wishes to identify the leader (or Master,
as he is called). Sam the Snitch makes the following offer.
Presented with any collection of members of the gang
he will (by a slight twitch of his left ear)
indicate if the Master is among them. However, in
view of the danger involved, he demands ten pounds
for each such encounter. Trubshaw believes that
the probability of the $j$th member of the gang being
the Master is $p_{j}$ $[1\leq j\leq n]$
and wishes to minimise the expected drain on the public purse.
Advise him.
\end{exercise}
\section{Non-independence}

(This section is non-examinable.)

In the previous sections we discussed codes
$c:{\mathcal A}\rightarrow{\mathcal B}^{*}$
such that, if a letter $A\in{\mathcal A}$ was chosen according to
some random law, ${\mathbb E}|c(A)|$ was about
as small as possible. If we choose $A_{1}$, $A_{2}$,
\dots independently according to the same law,
then it is not hard to convince oneself that
\[{\mathbb E}|c^{*}(A_{1}A_{2}A_{3}\dots A_{n})|
=n{\mathbb E}|c(A)|\]
will be as small as possible.

However, in re*l lif* th* let*ers a*e 
often no* i*d*p***ent. It is sometimes possible
to send messages more
efficiently using this fact.
\begin{exercise}\label{E;Markov} 
Suppose that we have a sequence
$X_{j}$ of random variables taking the values
$0$ and $1$. Suppose that $X_{1}=1$ with probability $1/2$
and
$X_{j+1}=X_{j}$ with probability
$.99$ independent of what has gone before.

(i) Suppose we wish to send ten successive bits
$X_{j}X_{j+1}\dots X_{j+9}$. Show that if we associate
the sequence of ten zeros with $0$, the sequence
of ten ones with $10$ and any other sequence
$a_{0}a_{1}\dots a_{9}$ with $11a_{0}a_{1}\dots a_{9}$
we have a decodable code which on average requires 
about $5/2$ bits to transmit the sequence.

(ii) Suppose we wish to send the bits
$X_{j}X_{j+10^{6}}X_{j+2\times 10^{6}}\dots X_{j+9\times 10^{6}}$. 
Explain why any decodable code will require
on average at least $10$ bits to transmit the sequence.
(You need not do detailed computations.)
\end{exercise}

If we transmit sequences of letters by forming
them into longer words and coding the words,
we say we have a block code. It is plausible that
the longer the blocks, the less important the
effects of non-independence. In more advanced courses
it is shown how to define entropy for systems
like the one discussed in Exercise~\ref{E;Markov}
(that is to say Markov chains) and that, provided
we take long enough blocks, we can recover
an analogue of Theorem~\ref{T;Shannon noiseless}
(the noiseless coding theorem).

In the real world, the problem lies deeper.
Presented with a photograph, we can instantly
see that it represents Lena wearing a hat.
If a machine reads the image pixel by pixel,
it will have great difficulty recognising
much, apart from the fact that the distribution
of pixels is `non-random' or has `low entropy'
(to use the appropriate hand-waving expressions).
Clearly, it ought to be possible to describe the photograph
with many fewer bits than are required to describe
each pixel separately, but, equally clearly, 
a method that works well on black and white photographs
may fail on colour photographs and a method that works well
on photographs of faces may work badly 
when applied to photographs  
of trees.

Engineers have a clever way of dealing with this problem.
Suppose we have a sequence $x_{j}$ of zeros and ones
produced by some random process. Someone who believes that
they partially understand the nature of the process
builds us a prediction machine
which, given the sequence $x_{1}$, $x_{2}$, \dots $x_{j}$ so far,
predicts the next term will be $x_{j+1}'$. Now set
\[y_{j+1}\equiv x_{j+1}-x'_{j+1}\mod 2.\]
If we are given the sequence $y_{1}$, $y_{2}$, \dots
we can recover the $x_{j}$ inductively using the
prediction machine and the formula
\[x_{j+1}\equiv y_{j+1}+x'_{j+1}\mod 2.\]

If the prediction machine is good, then the
sequence of $y_{j}$ will consist mainly of zeros
and there will be many ways of encoding the sequence 
as (on average) a much shorter code word.
(For example, if we arrange the sequence in blocks of 
fixed length, many of the possible blocks will have very low 
probability, so Huffman's algorithm will be very effective.)


Build a better mousetrap, and the world will beat a path to your door.
Build a better prediction machine and the world will beat your door down.

There is a further real world complication.
Engineers
distinguish between irreversible
`lossy compression'
and reversible `lossless compression'.
For compact discs, where bits are cheap,
the sound recorded can be reconstructed
exactly. For digital sound broadcasting, where
bits are expensive, the engineers make use
of knowledge of the human auditory system
(for example, the fact that we can not
make out very soft noise in the presence
of loud noises) to produce a result that might
sound perfect (or nearly so) to us, but which
is, in fact, not. For mobile phones, there can be
greater loss of data because users
do not demand anywhere close to 
perfection.
For digital TV, the situation is still more
striking with reduction in data content from
film to TV of anything up to a factor of  60.
However, medical and satellite pictures must
be transmitted with no loss of data.
Notice that lossless coding can be judged by
absolute criteria, but the merits of lossy
coding can only be judged subjectively.

Ideally, lossless compression should
lead to a signal indistinguishable (from a statistical
point of view) from a random signal
in which the value of each bit is independent
of the value of all the others.
In practice, this is only possible in certain
applications. As an indication of the kind of
problem involved, consider TV pictures. If
we know that what is going to be transmitted is
`head and shoulders' or `tennis matches' or
`cartoons' it is possible to obtain extraordinary
compression ratios by `tuning' the compression method
to the expected pictures, but then changes from
what is expected can be disastrous. At present,
digital TV encoders merely expect the picture to
consist of blocks which move at nearly constant
velocity remaining more or less unchanged from
frame to frame\footnote{Watch what happens when
things go wrong.}. In this, as in other applications,
we know that after compression
the signal still has non-trivial
statistical properties, 
but we do not know
enough about them to exploit them.
\section{What is an error correcting code?}
In the introductory Section~\ref{S;alphabets}, 
we discussed `telegraph codes' in which
one five letter
combination `QWADR', say, meant `please book
quiet room for two' and another `QWNDR' meant
`please book cheapest room for one'.
Obviously, also, an error of one letter
in this code could have unpleasant
consequences\footnote{This is a made up example,
since compilers of such codes understood the problem.}.

Today, we transmit and store long strings of 
binary sequences, but face the same problem
that some digits may not be transmitted or stored correctly.
We suppose that the string is the result of data compression
and so, as we said at the end of the last
section, \emph{although the string 
may have non-trivial
statistical properties, 
we do not know
enough to exploit this fact}. (If we knew how to
exploit any statistical regularity, we could build a
prediction device and compress the data still further.)
Because of this, we shall assume that
we are asked to consider a collection
of $m$ messages \emph{each of which is equally
likely}.

Our model is the following. When the `source'
produces one of the $m$ possible messages $\mu_{i}$ say,
it is fed into a `coder' which outputs
a string $\mathbf{c}_{i}$ of $n$ binary digits.
The string is then transmitted one digit
at a time along a `communication channel'.
Each digit has probability $p$ of being mistransmitted
(so that $0$ becomes $1$ or $1$ becomes $0$)
independently of what happens to the other digits $[0\leq p<1/2]$.
The transmitted message is then passed through
a `decoder' which either produces a message $\mu_{j}$
(where we hope that $j=i$) or an error message
and passes it on to the `receiver'.
The technical term for our model is the
\emph{binary symmetric channel} (binary because
we use two symbols, symmetric because
the probability of error is the same whichever 
symbol we use).
\begin{exercise}\label{Reverse} Why do we not consider
the case $1\geq p>1/2$? What
if $p=1/2$?
\end{exercise}

For most of the time we shall concentrate our attention
on a \emph{code} $C\subseteq\{0,1\}^{n}$ consisting
of the \emph{codewords} $\mathbf{c}_{i}$.
(Thus we use a fixed length code.) We say that
$C$ has \emph{size} $m=|C|$. 
If $m$ is large then
we can send a large number of possible messages
(that is to say, we can send more information) but,
as $m$ increases, it becomes harder to distinguish
between different messages when errors occur.
At one extreme, if $m=1$, errors cause us no problems
(since there is only one message) but no information
is transmitted (since there is only one message).
At the other extreme, if $m=2^{n}$, we can transmit
lots of messages but any error moves us from one
codeword to another. We are led to the following
rather natural definition.
\begin{definition}\label{information rate}
The information rate of $C$ is
$\dfrac{\log_{2} m}{n}$.
\end{definition}
Note that, since $m\leq 2^{n}$ the information rate
is never greater than $1$. Notice also that the values of
the information rate when $m=1$ and $m=2^{n}$ agree
with what we might expect.
 
How should our decoder work? We have assumed that
all messages are equally likely and that errors
are independent (this would not be true if, for
example, errors occurred in bursts\footnote{For
the purposes of this course we note
that this problem could
be tackled by permuting the `bits' of the message
so that `bursts are spread out'. In theory, we
could do better than this by using
the statistical properties of such bursts
to build a prediction machine.
In practice, this is rarely possible.
In the paradigm case of mobile phones,
the properties of the transmission channel
are constantly changing and are not well understood.
(Here the main restriction on the use of permutation
is that it introduces time delays. One way round this
is `frequency hopping' in which several users
constantly swap transmission channels `dividing
bursts among users'.) One desirable property of
codes for mobile phone users is that they should
`fail gracefully', so that as the error rate
for the channel rises the error rate for the receiver
should not suddenly explode.}).

Under these
assumptions, a reasonable strategy for our decoder
is to guess
that the codeword sent is one which differs
in the fewest places from the string of $n$
binary digits received. Here and elsewhere
the discussion can be illuminated by the
simple notion of a Hamming distance.
\begin{definition} If $\mathbf{x},\ \mathbf{y}\in \{0,1\}^{n}$,
we write
\[d(\mathbf{x},\mathbf{y})=\sum_{j=1}^{n}|x_{j}-y_{j}|\]
and call $d(\mathbf{x},\mathbf{y})$ the Hamming distance between
$\mathbf{x}$ and $\mathbf{y}$.
\end{definition}
\begin{lemma} The Hamming distance is a metric.
\end{lemma}
We now do some very simple IA probability.
\begin{lemma} We work with the coding and transmission scheme
described above.
Let ${\mathbf c}\in C$ and ${\mathbf x}\in \{0,1\}^{n}$.

(i) If $d({\mathbf c},{\mathbf x})=r$, then
\[\Pr({\mathbf x}\ \text{received given
${\mathbf c}$ sent})=p^{r}(1-p)^{n-r}.\]

(ii)  If
$d({\mathbf c},{\mathbf x})=r$, then
\[\Pr({\mathbf c}\ \text{sent given
${\mathbf x}$ received})=A({\mathbf x})p^{r}(1-p)^{n-r},\]
where $A({\mathbf x})$ does not depend on $r$ or ${\mathbf c}$.

(iii) If ${\mathbf c}'\in C$  and
$d({\mathbf c}',{\mathbf x})\geq d({\mathbf c},{\mathbf x})$,
then
\[\Pr({\mathbf c}\ \text{sent given
${\mathbf x}$ received})\geq
\Pr({\mathbf c}'\ \text{sent given
${\mathbf x}$ received}),\]
with equality if and only if
$d({\mathbf c}',{\mathbf x})=d({\mathbf c},{\mathbf x})$.
\end{lemma}
This lemma justifies our use, both explicit
and implicit, throughout what follows of the so-called
\emph{maximum likelihood} decoding rule.
\begin{definition} The maximum likelihood decoding
rule states that a string ${\mathbf x}\in \{0,1\}^{n}$
received by a decoder should be decoded as
(one of) the codeword(s) at the smallest
Hamming distance from ${\mathbf x}$.
\end{definition}
Notice that, although this decoding rule is mathematically
attractive, it may be impractical if $C$ is large
and there is often no known way of finding the codeword at
the smallest distance from a particular ${\mathbf x}$
in an acceptable number of steps.
(We can always make a complete search through all the
members of $C$ but unless there are very special circumstances
this is likely to involve an unacceptable amount of work.)


\section{Hamming's breakthrough}
Although we have used simple probabilistic arguments
to justify it, the maximum likelihood decoding rule
will often enable us to avoid probabilistic considerations
(though not in the very important part of this concerned with
Shannon's noisy coding theorem) and concentrate on
algebra and combinatorics.
The spirit of most of the course is exemplified in the
next two definitions.
\begin{definition} We say that $C$ is $d$ \emph{error detecting}
if changing up to $d$ digits in a codeword never produces
another codeword.
\end{definition}
\begin{definition} We say that $C$ is $e$
\emph{error correcting}
if knowing that a string of $n$ binary digits differs
from some codeword of $C$ in at most
$e$ places we can deduce the codeword.
\end{definition}

Here are some simple schemes. Some of them use alphabets with
more than two symbols but the principles remain the same.

\noindent\emph{Repetition coding of length $n$}.
We take codewords of the form
\[{\mathbf c}=(c,c,c,\dots,c)\]
with $c=0$ or $c=1$. The code $C$ is $n-1$ error detecting,
and $\lfloor (n-1)/2\rfloor$ error correcting.
The maximum likelihood decoder chooses the
symbol that occurs most often.
(Here and elsewhere $\lfloor \alpha\rfloor$ is the largest
integer $N\leq\alpha$ and $\lceil \alpha \rceil$ is
the smallest integer $M\geq\alpha$.) Unfortunately,
the information rate is $1/n$ which is
rather low\footnote{Compare
the chorus `Oh no John, no John, no John, no'.}.

\noindent\emph{The Cambridge examination paper code}
Each candidate is asked to write down a Candidate Identifier of the
form 
$1234A$, $1235B$, $1236C$,\ \dots 
(the eleven\footnote{My guess.} possible letters are repeated cyclically)
and a desk number. The first four numbers
in the Candidate Identifier  identify the candidate
uniquely. If the letter written by the
candidate does not correspond to
to the first four numbers the candidate is identified
by using the desk number.
\begin{exercise}\label{E;Cambridge exam} 
Show that if the candidate makes one error
in the Candidate Identifier,
then this will be detected.
Would this be true if there were $9$ possible
letters repeated cyclically? Would this be true
if there were $12$ possible
letters repeated cyclically?
Give reasons.

Show that, if we also use the Desk Number
then the combined code Candidate Number/Desk Number 
is one error correcting
\end{exercise}

\noindent\emph{The paper tape code.}
Here and elsewhere, it is convenient
to give $\{0,1\}$ the structure
of the field
${\mathbb F}_{2}={\mathbb Z}_{2}$
by using arithmetic modulo 2.
The codewords have the form
\[{\mathbf c}=(c_{1},c_{2},c_{3},\dots,c_{n})\]
with $c_{1}$, $c_{2}$, \dots, $c_{n-1}$ freely
chosen elements of ${\mathbb F}_{2}$ and
$c_{n}$ (the check digit) the element of
${\mathbb F}_{2}$ which gives
\[c_{1}+c_{2}+\dots+c_{n-1}+c_{n}=0.\]
The resulting code $C$ is $1$ error detecting
since, if ${\mathbf x}\in {\mathbb F}_{2}^{n}$
is obtained from ${\mathbf c}\in C$
by making a single error, we have
\[x_{1}+x_{2}+\dots+x_{n-1}+x_{n}=1.\]
However it is not error correcting
since, if
\[x_{1}+x_{2}+\dots+x_{n-1}+x_{n}=1,\]
there are $n$ codewords ${\mathbf y}$ with
Hamming distance $d({\mathbf x},{\mathbf y})=1$.
The information rate is $(n-1)/n$. Traditional
paper tape had 8 places per line each
of which could have a punched hole or not,
so $n=8$.

\begin{exercise}\label{ISBN}
If you look
at the inner
title page of almost any book published between
1970 and 2006
you will find its International Standard
Book Number (ISBN). The ISBN
uses single digits selected from 0, 1, \dots, 8, 9
and $X$ representing 10. Each ISBN consists
of nine such digits $a_{1}$, $a_{2}$, \dots, $a_{9}$
followed by a single check digit $a_{10}$ chosen
so that
\begin{equation*}
10a_{1}+9a_{2}+ \dots+2a_{9}+a_{10}\equiv 0\mod{11}.\tag*{(*)}
\end{equation*}
(In more sophisticated language, our code $C$ consists
of those elements ${\mathbf a}\in {\mathbb F}_{11}^{10}$
such that $\sum_{j=1}^{10}(11-j)a_{j}=0$.)

(i) Find a couple of books\footnote{In case of difficulty,
your college library may be of assistance.}
and check that $(*)$ holds for their ISBNs\footnote{In fact,
$X$ is only used in the check digit place.}.

(ii) Show that $(*)$ will not work if you make a mistake
in writing down one digit of an ISBN.

(iii) Show that
$(*)$ may fail to detect two errors.

(iv) Show that $(*)$ will not work if you interchange
two distinct adjacent digits (a transposition error).

(v) Does~(iv) remain true if we replace `adjacent'
by `different'?

\noindent Errors of type (ii) and (iv) are the most common
in typing\footnote{Thus a syllabus for an 
earlier version of this
course contained the rather charming misprint
of `snydrome' for `syndrome'.}.
In communication between publishers and booksellers,
both sides are anxious that errors should be detected
but would prefer the other side to query errors
rather than to guess what the error might have been.

(vi) After January 2007, the appropriate ISBN is a $13$ digit number
$x_{1}x_{2}\dots x_{13}$ with each digit 
selected from $0$, $1$,\,\dots, $8$, $9$ and
the check digit $x_{13}$ computed by using the formula
\[x_{13}\equiv -(x_{1}+3x_{2}+x_{3}+3x_{4}+\cdots+x_{11}+ 3x_{12}) 
\mod{10}.\]
Show that we can detect single errors. Give an example
to show that we cannot detect all transpositions.
\end{exercise}

Hamming had access to an early electronic computer
but was low down in the priority list of users.
He would submit his
programs encoded on paper tape to run over the
weekend but often he would have his tape returned
on Monday because the machine had detected an error
in the tape. `If the machine can detect an error'
he asked himself `why can the machine not correct it?'
and he came up with the following scheme.

\noindent\emph{Hamming's original code.}
We work in ${\mathbb F}_{2}^{7}$. The codewords
${\mathbf c}$ are chosen to satisfy the
three conditions
\begin{align*}
c_{1}+c_{3}+c_{5}+c_{7}&=0\\
c_{2}+c_{3}+c_{6}+c_{7}&=0\\
c_{4}+c_{5}+c_{6}+c_{7}&=0.
\end{align*}
By inspection, we may choose $c_{3}$, $c_{5}$, $c_{6}$
and $c_{7}$ freely and then $c_{1}$, $c_{2}$ and $c_{4}$
are completely determined. The information rate is
thus $4/7$.

Suppose that we receive the string
${\mathbf x}\in{\mathbb F}_{2}^{7}$.
We form the \emph{syndrome}
$(z_{1},z_{2},z_{4})\in{\mathbb F}_{2}^{3}$
given by
\begin{align*}
z_{1}&=x_{1}+x_{3}+x_{5}+x_{7}\\
z_{2}&=x_{2}+x_{3}+x_{6}+x_{7}\\
z_{4}&=x_{4}+x_{5}+x_{6}+x_{7}.
\end{align*}
If ${\mathbf x}$ is a codeword, then
$(z_{1},z_{2},z_{4})=(0,0,0)$.
If ${\mathbf c}$ is a codeword and
the Hamming distance $d({\mathbf x},{\mathbf c})=1$,
then the place in which ${\mathbf x}$ differs
from ${\mathbf c}$ is given by $z_{1}+2z_{2}+4z_{4}$
(using ordinary addition, not addition modulo $2$)
as may be easily checked using linearity
and a case by case study of the seven binary
sequences ${\mathbf x}$ containing one 1
and six 0s. The Hamming code is thus 1
error correcting.

\begin{exercise}\label{bacon}
Suppose we use eight hole tape with
the standard paper tape code
and the probability that an error occurs at a particular
place on the tape (i.e. a hole occurs where it should
not or fails to occur where it should) is $10^{-4}$.
A program requires about 10\,000 lines of tape
(each line containing eight places)
using the paper tape code. Using
the Poisson approximation, direct calculation
(possible with a hand calculator but really no
advance on the Poisson method), or otherwise,
show that the probability that the tape
will be accepted as error free by the decoder
is less than .04\%.

Suppose now that we use the Hamming scheme
(making no use of the last place in each line).
Explain why the program requires about
17\,500 lines of tape but that any
particular line will be correctly decoded
with probability about $1-(21\times 10^{-8})$
and the probability that the entire program
will be correctly decoded is better than
99.6\%.
\end{exercise}

Hamming's scheme is easy to implement. It
took a little time for his company to
realise what he had done\footnote{Experienced engineers
came away from working demonstrations
muttering `I still don't believe it'.}
but they were soon trying to patent it.
In retrospect, the idea of an error correcting
code seems obvious (Hamming's scheme had
actually been used as the basis of a Victorian
party trick) and indeed two or three other
people discovered it independently, but Hamming
and his
co-discoverers had done more than
find a clever answer to a question. They
had asked an entirely new question and
opened a new field for mathematics and engineering.

The times were propitious for the development
of the new field. Before 1940, error correcting
codes would have been luxuries, solutions
looking for problems, after 1950, with the
rise of the computer and new communication
technologies, they became necessities.
Mathematicians and engineers returning
from wartime duties in
code breaking, code making and
general communications problems were
primed to grasp and extend the ideas.
The mathematical engineer
Claude Shannon may be considered the
presiding  genius of the new field.

The reader will observe that data compression 
shortens the length of our messages by
removing redundancy and Hamming's scheme
(like all error correcting codes) lengthens
them by introducing redundancy. This is true,
but  data compression removes redundancy which we
do not control and which is not useful
to us and error correction coding then
replaces it with carefully controlled
redundancy which we can use.

The reader will also note an analogy with ordinary language.
The idea of data compression is illustrated by the
fact that many common words 
are short\footnote{Note how `horseless carriage'
becomes `car' and `telephone' becomes `phone'.}.
On the other hand the redund of ordin lang
makes it poss to understa it even if we do
no catch everyth that is said.
\section{General considerations} How good
can error correcting and error 
detecting\footnote{If the error rate is low and it is easy
to ask for the message to be retransmitted, it may be
cheaper to concentrate on error detection. If there is
no possibility of retransmission (as in long term data storage),
we have to concentrate on error correction.}
codes be? The following discussion is a natural
development of the ideas we have already
discussed. Later, in our discussion of Shannon's noisy coding
theorem we shall see another
and deeper way
of looking at the question. 
\begin{definition} The minimum distance $d$
of a code is the smallest Hamming distance
between distinct code words.
\end{definition}
We call a code of length $n$,  size $m$ and
distance $d$ an $[n,m,d]$ code. Less briefly,
a set $C\subseteq{\mathbb F}_{2}^{n}$,
with $|C|=m$ and
\[\min\{d({\mathbf x},{\mathbf y}):
{\mathbf x},{\mathbf y}\in C,\ {\mathbf x}\neq{\mathbf y}\}
=d\]
is called an $[n,m,d]$ code. By an $[n,m]$ code
we shall simply mean a code of length $n$
and size $m$.

\begin{lemma}\label{minimum distance}
A code of minimum distance
$d$ can detect $d-1$ errors\footnote{This is not
as useful as it looks when $d$ is large. If we know
that our message is likely to contain many
errors, all that an error detecting code
can do is confirm our expectations.
Error detection is only useful when errors are
unlikely.} and correct
$\lfloor\frac{d-1}{2}\rfloor$ errors.
It cannot detect all sets of $d$ errors
and cannot correct all sets of
$\lfloor\frac{d-1}{2}\rfloor+1$ errors.
\end{lemma}
It is natural, here and elsewhere, to make use
of the geometrical insight provided by the
(closed) Hamming ball
\[B({\mathbf x},r)=\{{\mathbf y}:
d({\mathbf x},{\mathbf y})\leq r\}.\]
Observe that
\[|B({\mathbf x},r)|=|B({\boldsymbol 0},r)|\]
for all ${\mathbf x}$ and so, writing
\[V(n,r)=|B({\boldsymbol 0},r)|,\]
we know that $V(n,r)$ is the number of points in any
Hamming ball of radius $r$. A simple counting
argument shows that
\[V(n,r)=\sum_{j=0}^{r}\binom{n}{j}.\]
\begin{theorem}{\bf [Hamming's bound]}\label{Hamming's bound} If a
code $C$ is $e$ error correcting, then
\[|C|\leq \frac{2^{n}}{V(n,e)}.\]
\end{theorem}

There is an obvious fascination (if not utility)
in the search for codes which attain the
exact Hamming bound.
\begin{definition} A code $C$ of length $n$
and size $m$ which can correct $e$ errors
is called \emph{perfect} if
\[m=\frac{2^{n}}{V(n,e)}.\]
\end{definition}
\begin{lemma} Hamming's original code is
a $[7,16,3]$ code. It is perfect.
\end{lemma}
It may be worth remarking in this context
that, if a code which can correct $e$ errors
is perfect (i.e. has a perfect packing
of Hamming balls of radius $e$),
then the decoder must invariably
give the wrong answer when presented with
$e+1$ errors. We note also that,
if (as will usually be the case)
$2^{n}/V(n,e)$ is not an integer, no
perfect $e$ error correcting code can exist.


\begin{exercise}\label{Not perfect}
Even if $2^{n}/V(n,e)$ is an integer,
no perfect code may exist.

(i) Verify that
\[\frac{2^{90}}{V(90,2)}=2^{78}.\]

(ii) Suppose that $C$ is a perfect 2 error correcting
code of length $90$ and size $2^{78}$. Explain
why we may suppose without loss of generality
that ${\boldsymbol 0}\in C$.

(iii) Let $C$ be as in (ii) with ${\boldsymbol 0}\in C$.
Consider the set
\[X=\{{\mathbf x}\in{\mathbb F}_{2}^{90}:
x_{1}=1,\ x_{2}=1,\ d({\boldsymbol 0},{\mathbf x})=3\}.\]
Show that, corresponding to each ${\mathbf x}\in X$,
we can find a unique ${\mathbf c}({\mathbf x})\in C$
such that $d({\mathbf c}({\mathbf x}),{\mathbf x})=2$.

(iv) Continuing with the argument of (iii), show
that
\[d({\mathbf c}({\mathbf x}),{\boldsymbol 0})=5\]
and that $c_{i}({\mathbf x})=1$ whenever $x_{i}=1$.
If  $\mathbf{y}\in X$,
find the number of solutions to the equation
${\mathbf c}({\mathbf x})={\mathbf c}({\mathbf y})$
with $\mathbf{x}\in X$
and, by considering the number of elements of $X$,
obtain a contradiction.

(v) Conclude that there is no perfect $[90,2^{78}]$ code.
\end{exercise}
The  result of Exercise~\ref{Not perfect} was obtained by
Golay. Far more importantly, he found another case
when $2^{n}/V(n,e)$ is an integer and there
does exist an associated perfect code (the Golay code).
\begin{exercise}\label{Golay perfect}
Show that $V(23,3)$ is a power of $2$.
\end{exercise}
Unfortunately the proof that the Golay code is perfect is
too long to be given in the course,



We obtained the Hamming bound, which
places an upper bound on how good
a code can be, by a \emph{packing} argument.
A \emph{covering} argument gives
us the GSV (Gilbert, Shannon, Varshamov) bound
in the opposite direction. Let us write
$A(n,d)$ for the size of the largest code
with minimum distance $d$.
\begin{theorem} {\bf [Gilbert, Shannon, Varshamov]}\label{GSV}
We have
\[A(n,d)\geq \frac{2^{n}}{V(n,d-1)}.\]
\end{theorem}
Until recently there were no general explicit
constructions for codes which achieved the
GSV bound (i.e. codes whose minimum distance
$d$ satisfied the inequality $A(n,d)V(n,d-1)\geq 2^{n}$).
Such a construction
was finally found by Garcia and Stichtenoth by
using `Goppa' codes.
\section{Some elementary probability}
Engineers are, of course, interested in `best codes'
of length $n$ for reasonably small values of $n$,
but mathematicians are particularly interested
in what happens as $n\rightarrow\infty$.

We recall some elementary probability.
\begin{lemma} {\bf [Tchebychev's inequality]}
If $X$ is a bounded real valued random variable
and $a>0$, then
\[\Pr(|X-{\mathbb E}X|\geq a)\leq\frac{\Var X}{a^{2}}.\]
\end{lemma}
\begin{theorem} {\bf [Weak law of large numbers]}
If $X_{1}$, $X_{2}$, \dots is a sequence of independent
identically distributed real valued bounded
random variables and $a>0$, then
\[\Pr\left(\left|n^{-1}\sum_{j=1}^{n}X_{j}
-{\mathbb E}X\right|\geq a\right)\rightarrow 0\]
as $N\rightarrow\infty$.
\end{theorem}
Applying the weak law of large numbers,
we obtain the following important result.
\begin{lemma} Consider the model of a noisy transmission
channel used in this course in which each digit
has probability $p$ of being wrongly transmitted
independently of what happens to the other digits.
If $\epsilon>0$, then
\begin{small}
\[\Pr\big(\text{number of errors in transmission for
message of $n$ digits}\geq (1+\epsilon)pn\big) \rightarrow 0\]
\end{small}
as $n\rightarrow\infty$.
\end{lemma}
By
Lemma~\ref{minimum distance},
a code of minimum distance
$d$ can correct
$\lfloor\frac{d-1}{2}\rfloor$ errors.
Thus, if we have an error rate $p$
and $\epsilon>0$, we know that
the probability that a
code of length $n$ with error correcting capacity
$\lceil (1+\epsilon)pn\rceil$
will fail to correct a transmitted message
falls to zero as $n\rightarrow\infty$.
By definition, the biggest code with
minimum distance $\lceil 2(1+\epsilon)pn \rceil$
has size $A(n,\lceil 2(1+\epsilon)pn \rceil)$
and so has information rate
$\log_{2}A(n,\lceil 2(1+\epsilon)pn \rceil)/n$.
Study of the behaviour of $\log_{2}A(n,n\delta)/n$
will thus tell us how large an information
rate is possible in the presence of a given error
rate.

\begin{definition} If $0<\delta<1/2$ we write
\[\alpha(\delta)=\limsup_{n\rightarrow\infty}
\frac{\log_{2} A(n,n\delta)}{n}.\]
\end{definition}
\begin{definition} We define the entropy
function $H:[0,1]\rightarrow{\mathbb R}$
by $H(0)=H(1)=0$ and
\[H(t)=-t\log_{2}(t)-(1-t)\log_{2}(1-t).\]
\end{definition}
\begin{exercise} (i) We have already met
Shannon entropy in Definition~\ref{D;Shannon entropy}.
Give a simple system such that, using the notation 
of that definition,
\[H(A)=H(t).\]
(ii) Sketch $H$. What is the value of $H(1/2)$?
\end{exercise}
\begin{theorem}\label{T;alpha}
With the definitions just given,
\[1-H(\delta)\leq\alpha(\delta)\leq 1-H(\delta/2)\]
for all $0\leq \delta<1/2$.
\end{theorem}
Using the Hamming bound (Theorem~\ref{Hamming's bound})
and the GSV bound (Theorem~\ref{GSV}), we see that
Theorem~\ref{T;alpha} follows at once from the
following result.
\begin{theorem}\label{log volume} We have
\[\frac{\log_{2}V(n,n\delta)}{n}
\rightarrow H(\delta)\]
as $n\rightarrow\infty$.
\end{theorem}
Our proof of Theorem~\ref{log volume} depends, as
one might expect, on a version of Stirling's formula.
We only need the very simplest version proved
in~IA.
\begin{lemma}[Stirling] We have
\[\log_{e} n!=n\log_{e}n-n+O(\log_{2}n).\]
\end{lemma}
We combine this with the remarks that
\[V(n,n\delta)=\sum_{0\leq j\leq n\delta}
\binom{n}{j}\]
and that very simple estimates give
\[\binom{n}{m}\leq \sum_{0\leq j\leq n\delta}
\binom{n}{j}
\leq (m+1)\binom{n}{m}\]
where $m=\lfloor n\delta\rfloor$.

Although the GSV bound is very important,
Shannon showed that
a stronger result can be obtained for the
error correcting power of the best long codes.
\section{Shannon's noisy coding theorem}
In the backstreets of Cambridge
(Massachusetts) there is a science museum devoted to
the glory of MIT. Since MIT has a great deal of glory
and since much thought has gone into the presentation
of the exhibits, it is well worth a visit. However,
for any mathematician, the highlight is a glass case
containing such things as a juggling machine,
an electronic calculator\footnote{THROBAC
the THrifty ROman numeral BAckwards-looking Computer.
Google `MIT Museum', go to `objects'  and then
search `Shannon'.}
that uses Roman numerals both externally and internally,
the remnants of a machine built to guess which
of heads and tails its opponent would choose 
next\footnote{That is to say a prediction machine.
Google `Shannon Mind-Reading Machine' for sites
giving demonstrations and descriptions of the underlying program.}
and a mechanical maze running mouse.
These objects were built by Claude Shannon.

In his 1937 master's thesis, Shannon showed how to
analyse circuits using Boolean algebra and binary arithmetic.
During the war he worked on gunnery control
and cryptography at Bell labs and in 1948 he published
\emph{A Mathematical Theory of Communication}%
\footnote{This beautiful paper is available on the web
and in his \emph{Collected Works}.}.
Shannon had several predecessors and many successors,
but it is his vision which underlies this course.

Hamming's bound together with
Theorem~\ref{T;alpha} 
gives a very strong hint that it is not possible to
have an information rate greater than 
$1-H(\delta)$ for an error rate $\delta<1/2$.
(We shall prove this explicitly in Theorem~\ref{T;Shannon converse}.)
On the other hand  the GSV bound
together with
Theorem~\ref{T;alpha} 
shows that it is always possible to
have an information rate greater than 
$1-H(2\delta)$ for an error rate $\delta<1/4$.

Although we can use repetition codes
to get a positive information rate
when $1/4\leq \delta<1/2$ it looks
very hard at first (and indeed second) glance
to improve these results. 

However, Shannon realised that 
we do not care whether errors arise because
of noise in transmission or imperfections
in our coding scheme. By allowing our coding 
scheme to be less than perfect 
(in this connection, see Question~\ref{E;random ball})
we can 
actually improve the information rate whilst
still keeping the error rate low.
\begin{theorem}{\bf [Shannon's noisy coding theorem]}\label{T;Shannon}
Suppose $0<p<1/2$ and $\eta>0$.
Then there exists an $n_{0}(p,\eta)$ such that,
for any $n>n_{0}$, we can find codes of
length $n$ which have the property
that (under our standard model
of  a symmetric binary
channel with probability of error $p$) the probability
that any codeword is mistaken is less than
$\eta$ and still have
information rate $1-H(p)-\eta$.
\end{theorem}
Shannon's theorem is a masterly display
of the power of elementary probabilistic
arguments to overcome problems
which appear insuperable by other 
means\footnote{Conway says that in order to
achieve success in a mathematical field
you must either be first or be clever.
However, as in the case of Shannon,
most of those who are first
to recognise a new mathematical field
are also clever.}.

However, it merely asserts that good codes
exist and gives no means of finding
them apart from exhaustive search.
More seriously, random codes will have
no useful structure and the only way to use them
is to `search through a large dictionary'
at the coding end and
`search through an enormous dictionary'
at the decoding end. 
It should also be noted that  $n_{0}(p,\eta)$
will be very large when $p$ is close to $1/2$.

\begin{exercise} Why, in the absence of suitable
structure, 
is the dictionary at the decoding
end much larger than the dictionary at the
coding end?
\end{exercise}

It is relatively simple to obtain a converse
to Shannon's theorem.
\begin{theorem}\label{T;Shannon converse}
Suppose $0<p<1/2$ and $\eta>0$.
Then there exists an $n_{0}(p,\eta)$ such that,
for any $n>n_{0}$, it is impossible
to find codes of
length $n$ which have the property
that (under our standard model
of  a symmetric binary
channel with probability of error $p$) the probability
that any codeword is mistaken is less than
$1/2$ and the code has
information rate $1-H(p)+\eta$.
\end{theorem}

As might be expected, Shannon's theorem 
and its converse extend
to more general noisy channels (in particular,  those 
where the noise is governed by a Markov chain $M$).
It is possible to define the entropy $H(M)$ 
associated with $M$
and to show that the information
rate cannot exceed $1-H(M)$ but that any
information rate lower than $1-H(M)$ can 
be attained with arbitrarily low error rates.
However, we must leave something for more
advanced courses, and as we said earlier, 
it is rare in practice to have very clear 
information about the nature of the noise 
we encounter.

There is one very important theorem of Shannon
which is not covered in this course. In it, he reinterprets
a result of Whittaker to show that any continuous signal 
whose Fourier transform vanishes outside a range of length
$R$ can be
reconstructed from its value at equally spaced
sampling points provided those points are less than $A/R$
apart. (The constant $A$ depends on the conventions used
in defining the Fourier transform.) This enables us
to apply the `digital' theory of information transmission
developed here to continuous signals.
\section{A holiday at the race track}\label{S;race track}
Although this section is examinable\footnote{When the
author of the present notes gives the course.
This is his interpretation of the sentence in the schedules
`Applications to gambling and the stock market.'
Other lecturers may view matters differently.},
the material is peripheral to the course.
Suppose a very rich friend makes you the following
offer. Every day, at noon, you may make a bet with her
for any amount $k$ you chose. You give her $k$
pounds which she keeps whatever happens.
She then tosses a coin
and, if it shows heads, she pays you $ku$
and, if it shows tails, she pays you nothing.
You know that the probability
of heads is $p$.
What should you do?

If $pu<1$, you should not bet, because your expected winnings are
negative. If $pu>1$, most mathematicians would
be inclined to bet, but how much? If you bet your entire fortune
and win, you will be better off than if you bet
a smaller sum, but, if you lose, then you are bankrupt
and cannot continue playing. 

Thus your problem is to discover
the proportion $w$ of your present fortune
that you should bet. Observe that your choice of $w$ 
will always be the same (since 
you expect to go on playing for ever). Only the size of your 
fortune will vary. If your fortune after $n$ goes
is $Z_{n}$, then
\[Z_{n+1}=Z_{n}Y_{n+1}\]
where $Y_{n+1}=uw+(1-w)$ if the $n+1$st throw is heads
and $Y_{n+1}=1-w$ if it is tails. 

Using the weak law of large numbers, we have the following 
result.
\begin{lemma} Suppose $Y$, $Y_{1}$, $Y_{2}$, \dots are identically
distributed 
independent random variables taking values in $[a,b]$ with
$0<a<b$. If we write $Z_{n}=Y_{1}\dots Y_{n}$, then
\[\Pr(|n^{-1}\log Z_{n}-{\mathbb E}\log Y|>\epsilon)\rightarrow 0\]
as $n\rightarrow 0$.
\end{lemma}   

Thus you should choose $w$ to maximise 
\[{\mathbb E}\log Y_{n}=p\log\big(uw+(1-w)\big)+(1-p)\log(1-w).\]
\begin{exercise}\label{E;fast Kelly}  
(i) Show that, for the situation described,
you should not bet if $up\leq 1$ and should take
\[w=\frac{up-1}{u-1}\]
if $up>1$. 

(ii) We write $q=1-p$. Show that, if $up>1$ and we choose 
the optimum $w$,
\[{\mathbb E}\log Y_{n}=p\log p+q\log q+\log u-q\log(u-1).\]
\end{exercise}

We have seen the expression $-(p\log p+q\log q)$ before
as (a multiple of)
the Shannon information entropy of a simple probabilistic system.
In a paper entitled 
\emph{A New Interpretation of Information Rate}\footnote{Available
on the web. The exposition is slightly opaque because
the Bell company which employed Kelly was anxious
not draw attention to the use of telephones for
betting fraud.}
Kelly  showed how to interpret this and similar
situations using communication theory. In his model
a gambler receives information over a noisy channel
about which horse is going to win. Just as Shannon's theorem
shows that information can be transmitted over
such a channel at a rate close to channel capacity
with negligible risk of error (provided the
messages are long enough), so that the gambler can 
(with arbitrarily high probability)
increase her fortune at a certain optimum rate
provided that she can continue to bet long enough.

Although the analogy between betting and communication
channels is very pretty, it was the suggestion that
those making a long sequence of bets should aim to
maximise the expectation of the logarithm 
(now called Kelly's criterion) which
made the paper famous. Although Kelly 
seems never to have used his idea in practice,
mathematicians like Thorp, Berlekamp and Shannon himself
have made substantial fortunes in the stock market
and claim to have used Kelly's ideas\footnote{However,
we hear more about mathematicians who win on the 
stock market than those who lose.}.

Kelly is also famous for an early demonstration of
speech synthesis in which a computer sang
`Daisy Bell'. This inspired the corresponding
scene in the film \emph{2001}.

Before rushing out to the race track or stock 
exchange\footnote{A sprat which thinks it's a shark
will have a very short life.},
the reader
is invited to run computer simulations of the result
of Kelly gambling for various values of $u$ and $p$.
She will observe that although, \emph{in the very long run},
the system works, the short run can be be very
unpleasant indeed.   
\begin{exercise}\label{E;Slow Kelly} 
Returning to our original problem, show that,
if you bet less than the optimal proportion, your fortune
will still tend to increase but more slowly, but, if you bet
more than some proportion $w_{1}$, your fortune will decrease.
Write down the equation for $w_{1}$.

$[$Moral: If you use the Kelly criterion veer on the side under-betting.$]$
\end{exercise}  
\section{Linear codes} The next few sections involve no
probability at all. We shall only be interested in constructing
codes which are easy to handle and have all their code words
at least a certain Hamming distance apart.

Just as ${\mathbb R}^{n}$ is a
vector space over ${\mathbb R}$ and
${\mathbb C}^{n}$ is a
vector space over ${\mathbb C}$, so
${\mathbb F}_{2}^{n}$ is a
vector space over ${\mathbb F}_{2}$.
(If you know about vector spaces over fields,
so much the better, if not, just follow
the obvious paths.)
A \emph{linear code} is a subspace of ${\mathbb F}_{2}^{n}$.
More formally, we have the following definition.
\begin{definition} A linear code is a subset of
${\mathbb F}_{2}^{n}$ such that

(i) ${\boldsymbol 0}\in C$,

(ii) if ${\mathbf x},{\mathbf y}\in C$,
then ${\mathbf x}+{\mathbf y}\in C$.
\end{definition}
Note that, if $\lambda\in{\mathbb F}$, then $\lambda=0$
or $\lambda=1$, so that condition (i) of the definition
just given guarantees that $\lambda{\mathbf x}\in C$
whenever ${\mathbf x}\in C$. We shall see that
linear codes have many useful properties.

\begin{example}
(i) The repetition code with
\[C=\{{\mathbf x}:{\mathbf x}=(x,x,\dots x)\}\]
is a linear code.

(ii)  The paper tape code
\[C=\left\{{\mathbf x}:\sum_{j=0}^{n}x_{j}=0\right\}\]
is a linear code.

(iii) Hamming's original code is a linear code.
\end{example}
The verification is easy. In fact,
examples (ii) and (iii) are `parity check
codes' and so automatically linear
as we see from the next lemma.

\begin{definition} Consider a set $P$ in
${\mathbb F}_{2}^{n}$. We say that $C$
is the code defined by the set of
\emph{parity checks}  $P$ if
the elements of $C$ are precisely
those ${\mathbf x}\in{\mathbb F}_{2}^{n}$
with
\[\sum_{j=1}^{n}p_{j}x_{j}=0\]
for all ${\mathbf p}\in P$.
\end{definition}
\begin{lemma} If $C$ is a code defined
by parity checks, then $C$ is linear.
\end{lemma}

We now prove the converse result.
\begin{definition}
If  $C$ is a linear code, we write
$C^{\perp}$ for the set of  ${\mathbf p}\in{\mathbb F}^{n}$
such that
\[\sum_{j=1}^{n}p_{j}x_{j}=0\]
for all ${\mathbf x}\in C$.
\end{definition}
\noindent
Thus $C^{\perp}$ is the set of parity
checks satisfied by $C$.
\begin{lemma}\label{anhilate}
If $C$ is a linear code, then

(i) $C^{\perp}$ is a linear code,

(ii) $(C^{\perp})^{\perp}\supseteq C$.
\end{lemma}
\noindent
We call $C^{\perp}$ the \emph{dual code} to $C$.

In the language of
the course on linear mathematics,
$C^{\perp}$ is the annihilator of $C$.
The following is a standard theorem of
that course.
\begin{lemma}\label{dimension anhilator}
If  $C$ is a linear code in
${\mathbb F}_{2}^{n}$ then
\[\dim C+\dim C^{\perp}=n.\]
\end{lemma}
Since the treatment of dual spaces is not the most
popular piece of mathematics in IB, we
shall give an independent proof later
(see the note after Lemma~\ref{kernel}).
Combining Lemma~\ref{anhilate}~(ii)
with Lemma~\ref{dimension anhilator},
we get the following corollaries.
\begin{lemma} If $C$ is a linear code, then
$(C^{\perp})^{\perp}=C$.
\end{lemma}
\begin{lemma}\label{parity}
Every linear code is defined by parity
checks.
\end{lemma}

Our treatment of linear codes has
been rather abstract. In order
to put computational flesh on
the  dry theoretical bones,
we introduce the notion of
a generator matrix.
\begin{definition} If $C$ is a linear
code of length $n$, any  $r\times n$
matrix whose rows form a basis
for $C$ is called a \emph{generator matrix}
for $C$. We say that $C$ has \emph{dimension}
or \emph{rank} $r$.
\end{definition}
\begin{example} As examples, we can find
generator matrices for the repetition code,
the paper tape code and the original Hamming code.
\end{example}
Remember that the Hamming code is the
code of length 7 given by the parity conditions
\begin{align*}
x_{1}+x_{3}+x_{5}+x_{7}&=0\\
x_{2}+x_{3}+x_{6}+x_{7}&=0\\
x_{4}+x_{5}+x_{6}+x_{7}&=0.
\end{align*}

By using row operations and column permutations
to perform Gaussian elimination, we
can give a constructive proof of
the following lemma.
\begin{lemma} Any linear code of length $n$ has
(possibly after permuting the order of coordinates)
a generator matrix of the form
\[(I_{r}|B).\]
\end{lemma}
Notice that this means that any codeword ${\mathbf x}$
can be written as
\[({\mathbf y}|{\mathbf z})=({\mathbf y}|{\mathbf y}B)\]
where ${\mathbf y}=(y_{1},y_{2},\dots,y_{r})$ may
be considered as the message and the vector
${\mathbf z}={\mathbf y}B$ of length $n-r$ may
be considered the check digits. Any code whose codewords
can be split up in this manner is called \emph{systematic}.

We now give a more computational treatment  of parity checks.
\begin{lemma}\label{kernel} If $C$ is a linear code
of length $n$
with generator matrix $G$, then ${\mathbf a}\in C^{\perp}$
if and only if
\[G{\mathbf a}^{T}={\boldsymbol  0}^{T}.\]
Thus
\[C^{\perp}=(\ker G)^{T}.\]
\end{lemma}
\noindent
Using the rank, nullity theorem,
we get a second proof of Lemma~\ref{dimension anhilator}.

Lemma~\ref{kernel} enables us to characterise
$C^{\perp}$.
\begin{lemma} If $C$ is a linear code of length $n$
and dimension $r$ with generator the $n\times r$
matrix $G$, then, if $H$ is any $n\times(n-r)$--
matrix with columns forming a basis of $\ker G$,
we know that $H$ is a parity check
matrix for $C$ and its transpose $H^{T}$ is a
generator for $C^{\perp}$.
\end{lemma}
\begin{example} (i) The dual of the paper tape
code  is the repetition code.

(ii) Hamming's original code has dual with generator
matrix
\[\begin{pmatrix}
1&0&1&0&1&0&1\\
0&1&1&0&0&1&1\\
0&0&0&1&1&1&1
\end{pmatrix}\]
\end{example}

We saw above that the codewords of a linear code
can be written
\[({\mathbf y}|{\mathbf z})=({\mathbf y}|{\mathbf y}B)\]
where ${\mathbf y}$ may be considered as the vector
of message
digits and ${\mathbf z}={\mathbf y}B$ as the vector
of check digits. Thus \emph{encoders} for linear
codes are easy to construct.

What about decoders? Recall that every linear code
of length $n$ has
a (non-unique) associated  parity check matrix $H$
with the property that ${\mathbf x}\in C$ if and only
if ${\mathbf x}H={\boldsymbol 0}$. If
${\mathbf z}\in{\mathbb F}_{2}^{n}$,
we define the \emph{syndrome}
of ${\mathbf z}$ to be ${\mathbf z}H$.
The following lemma is mathematically trivial
but forms the basis of the method of
\emph{syndrome decoding}.
\begin{lemma} Let $C$ be a linear code with
parity check matrix $H$.
If we are given
${\mathbf z}={\mathbf x}+{\mathbf e}$
where ${\mathbf x}$ is a code word
and the `error vector'
${\mathbf e}\in{\mathbb F}_{2}^{n}$,
then
\[{\mathbf z}H={\mathbf e}H.\]
\end{lemma}
Suppose we have tabulated the syndrome ${\mathbf u}H$
for all ${\mathbf u}$ with `few' non-zero entries
(say, all ${\mathbf u}$ with
$d({\mathbf u},{\boldsymbol 0})\leq K$).
When our decoder receives ${\mathbf z}$, it computes the
syndrome ${\mathbf z}H$. If the syndrome is zero,
then ${\mathbf z}\in C$ and the decoder assumes
the transmitted message was ${\mathbf z}$. If
the syndrome of the received message is a non-zero
vector ${\mathbf w}$, the decoder searches its list
until it finds an ${\mathbf e}$  with
${\mathbf e}H={\mathbf w}$. The decoder then
assumes that the transmitted message was
${\mathbf x}={\mathbf z}-{\mathbf e}$
(note that ${\mathbf z}-{\mathbf e}$ will
always be a codeword, even if not the right one).
This procedure will fail if ${\mathbf w}$
does not appear in the list, but, for this to
be case, at least $K+1$
errors must have occurred.

If we take $K=1$, that is we only want a 1 error
correcting code, then, writing ${\mathbf e}^{(i)}$
for the vector in ${\mathbb F}_{2}^{n}$ with
$1$ in the $i$th place and $0$ elsewhere, we
see that the syndrome ${\mathbf e}^{(i)}H$
is the $i$th row of $H$. If the transmitted
message ${\mathbf z}$ has syndrome
${\mathbf z}H$ equal to the $i$th row of $H$,
then the decoder assumes that there has been an error
in the $i$th place and nowhere else.
(Recall the special case of Hamming's original code.)

If $K$ is large the task of searching the list
of possible syndromes becomes onerous and,
unless (as sometimes happens) we can find
another trick,
we find that `decoding becomes dear'
although `encoding remains cheap'.

We conclude this section by looking at weights
and the
weight enumeration polynomial for a linear code.
The idea here is to exploit the fact that, if
$C$ is linear code and ${\mathbf a}\in C$,
then ${\mathbf a}+C=C$. Thus the `view of $C$'
from any codeword ${\mathbf a}$ is the same
as the `view of $C$' from the particular codeword
${\boldsymbol 0}$.
\begin{definition} The \emph{weight} $w({\mathbf x})$
of a vector ${\mathbf x}\in{\mathbb F}_{2}^{n}$
is given by
\[w({\mathbf x})=d({\boldsymbol 0},{\mathbf x}).\]
\end{definition}
\begin{lemma} If $w$ is the weight function on
${\mathbb F}_{2}^{n}$ and
${\mathbf x},\,{\mathbf y}\in{\mathbb F}_{2}^{n}$,
then

(i) $w({\mathbf x})\geq 0$,

(ii) $w({\mathbf x})=0$ if and only if
${\mathbf x}={\boldsymbol 0}$,

(iii) $w({\mathbf x})+w({\mathbf y})\geq
w({\mathbf x}+{\mathbf y})$.
\end{lemma}

Since the minimum (non-zero) weight in a linear code
is the same as the minimum (non-zero)
distance, we can talk
about linear codes of minimum weight $d$
when we mean linear codes of minimum distance $d$.

The pattern of distances in a linear code
is encapsulated in the
weight enumeration polynomial.
\begin{definition}\label{enumeration}
Let $C$ be a linear code of length $n$.
We write $A_{j}$ for the number of codewords
of weight $j$ and define the
\emph{weight enumeration polynomial} $W_{C}$
to be the polynomial in two real variables
given by
\[W_{C}(s,t)=\sum_{j=0}^{n}A_{j}s^{j}t^{n-j}.\]
\end{definition}
Here are some simple properties of $W_{C}$.
\begin{lemma} Under the assumptions and
with the notation of Definition~\ref{enumeration},
the following results are true.

(i) $W_{C}$ is a homogeneous polynomial of degree $n$.

(ii)  If $C$ has rank $r$, then $W_{C}(1,1)=2^{r}$.

(iii) $W_{C}(0,1)=1$.

(iv) $W_{C}(1,0)$ takes the value $0$ or $1$.

(v)  $W_{C}(s,t)=W_{C}(t,s)$ for all $s$ and $t$
if and only if $W_{C}(1,0)=1$.
\end{lemma}
\begin{lemma} For our standard model of communication
along an error prone channel with independent errors
of probability $p$ and a linear code $C$ of length $n$,
\[W_{C}(p,1-p)=\Pr(\text{receive a code word}\ |
\ \text{code word transmitted})\]
and
\[\Pr(\text{receive incorrect code word}\ |
\ \text{code word transmitted})=W_{C}(p,1-p)-(1-p)^{n}.\]
\end{lemma}
\begin{example}\label{Example William}
(i) If $C$ is the repetition code,
$W_{C}(s,t)=s^{n}+t^{n}$.

(ii) If $C$ is the paper tape code of length $n$,
$W_{C}(s,t)=\frac{1}{2}((s+t)^{n}+(t-s)^{n})$.
\end{example}

Example~\ref{Example William} is a special case of the
MacWilliams identity.
\begin{theorem}{\bf [MacWilliams identity]} If $C$ is a linear
code
\[W_{C^{\perp}}(s,t)=2^{-\dim C}W_{C}(t-s,t+s).\]
\end{theorem}
\noindent
We give a proof as Exercise~\ref{E;MacWilliams}.
(The result is thus not bookwork though it
could be set as a problem with appropriate hints.)
\section{Some general constructions} However
interesting the theoretical study of codes may be
to a pure mathematician,
the engineer would prefer to have an arsenal
of practical codes so that she can select
the one most suitable for the job in hand.
In this section we discuss the general Hamming
codes and the Reed-Muller codes as well as some
simple methods of obtaining new codes from
old.

\begin{definition} Let $d$ be a strictly
positive integer and let $n=2^{d}-1$.
Consider the  (column) vector
space $D={\mathbb F}_{2}^{d}$. Write down
a $d\times n$ matrix $H$ whose columns
are the $2^{d}-1$ distinct non-zero vectors
of $D$. The Hamming $(n,n-d)$ code is
the linear code of length $n$ with $H^{T}$
as parity check matrix.
\end{definition}

Of course the Hamming $(n,n-d)$ code is only
defined up to permutation of coordinates.
We note that $H$ has rank $d$, so a simple
use of the rank nullity theorem shows
that our notation is consistent.
\begin{lemma}  The Hamming $(n,n-d)$ code is
a linear code of length $n$ and rank $n-d$ $[n=2^{d}-1]$.
\end{lemma}
\begin{example} The Hamming $(7,4)$ code
is the original Hamming code.
\end{example}

The fact that any two rows of $H$ are linearly
independent and a look at the appropriate syndromes
gives us the main property of the general Hamming code.
\begin{lemma} The Hamming $(n,n-d)$ code
has minimum weight $3$ and is a perfect
1 error correcting code $[n=2^{d}-1]$.
\end{lemma}
Hamming codes are ideal in situations where
very long strings of binary digits must be transmitted
but the chance of an error in any individual
digit is very small. (Look at Exercise~\ref{bacon}.)
Although the search for perfect codes other than the Hamming codes
produced
the Golay code (not discussed here) and much interesting
combinatorics, the reader is warned that, from a practical
point of view, it represents a dead
end\footnote{If we confine ourselves to the  binary
codes discussed in this course, it is known
that perfect codes of length $n$ with Hamming spheres of radius
$\rho$ exist for $\rho=0$, $\rho=n$, $\rho=(n-1)/2$,
with $n$ odd (the three codes just mentioned are
easy to identify), $\rho=3$ and $n=23$ (the Golay code,
found by direct search) and $\rho=1$ and $n=2^{m}-1$.
There are known to be non-Hamming codes with  $\rho=1$ and $n=2^{m}-1$,
it is suspected that there are many of them and
they are the subject of much research, but,
of course they present no practical advantages.
The only linear perfect codes with $\rho=1$ and $n=2^{m}-1$
are the Hamming codes.}.

Here are a number of simple tricks for creating new
codes from old.
\begin{definition} If $C$ is a code of
length $n$, the \emph{parity check extension}
$C^{+}$ of $C$ is the code of length $n+1$ given by
\[C^{+}=\left\{{\mathbf x}\in{\mathbb F}_{2}^{n+1}:
(x_{1},x_{2},\dots,x_{n})\in C,\ \sum_{j=1}^{n+1}x_{j}=0
\right\}.\]
\end{definition}
\begin{definition} If $C$ is a code of
length $n$, the \emph{truncation}
$C^{-}$ of $C$ is the code of length $n-1$ given by
\[C^{-}=\{(x_{1},x_{2},\dots,x_{n-1}):
(x_{1},x_{2},\dots,x_{n})\in C
\ \text{for some $x_{n}\in{\mathbb F}_{2}$}\}.\]
\end{definition}
\begin{definition} If $C$ is a code of
length $n$, the \emph{shortening} (or \emph{puncturing})
$C'$ of $C$ by the symbol $\alpha$ (which may be $0$
or $1$)
is the code of length $n-1$ given by
\[C'=\{(x_{1},x_{2},\dots,x_{n-1}):
(x_{1},x_{2},\dots,x_{n-1},\alpha)\in C\}.\]
\end{definition}
\begin{lemma} If $C$ is linear, so is its
parity check extension $C^{+}$, its
truncation $C^{-}$ and its shortening $C'$ (provided
that the symbol chosen is $0$).
\end{lemma}

How can we combine two linear codes $C_{1}$ and $C_{2}$?
Our first thought might be to look at their
direct sum
\[C_{1}\oplus C_{2}=\{({\mathbf x}|{\mathbf y})
:{\mathbf x}\in C_{1},\ {\mathbf y}\in C_{2}\},\]
but this is unlikely to be satisfactory.
\begin{lemma} If $C_{1}$ and $C_{2}$ are linear codes,
then we have the following relation between
minimum distances.
\[d(C_{1}\oplus C_{2})=\min\big(d(C_{1}),d(C_{2})\big).\]
\end{lemma}
On the other hand, if $C_{1}$ and $C_{2}$ satisfy
rather particular conditions, we can obtain
a more promising construction.
\begin{definition} Suppose $C_{1}$ and $C_{2}$
are linear codes of length $n$
with $C_{1}\supseteq C_{2}$
(i.e. with $C_{2}$ a subspace of $C_{1}$). We define
the \emph{bar product} $C_{1}|C_{2}$ of $C_{1}$
and $C_{2}$ to be the code of length $2n$ given by
\[C_{1}|C_{2}=\{({\mathbf x}|{\mathbf x}+{\mathbf y})
:{\mathbf x}\in C_{1},\ {\mathbf y}\in C_{2}\}.\]
\end{definition}
\begin{lemma} Let $C_{1}$ and $C_{2}$
be linear codes of length $n$
with $C_{1}\supseteq C_{2}$. Then
the bar product $C_{1}|C_{2}$
is a linear code with
\[\rank C_{1}|C_{2}=\rank C_{1}+\rank C_{2}.\]
The minimum distance of $C_{1}|C_{2}$ satisfies
the equality
\[d(C_{1}|C_{2})=\min(2d(C_{1}),d(C_{2})).\]
\end{lemma}
We now return to the construction of specific codes.
Recall that the Hamming codes are suitable for situations
when the error rate $p$ is very small and we want
a high information rate. The Reed-Muller are suitable
when the error rate is very high and we are prepared
to sacrifice information rate. They were used by NASA
for the radio transmissions from its planetary probes
(a task which has been compared
to signalling across
the Atlantic with a child's torch\footnote{Strictly
speaking, the comparison is meaningless. However,
it sounds impressive and that is the main thing.}).

We start by considering the $2^{d}$ points $P_{0}$, $P_{1}$,
\dots, $P_{2^{d}-1}$ of the space
$X={\mathbb F}_{2}^{d}$. Our code
words will be of length $n=2^{d}$ and will
correspond to the indicator functions
${\mathbb I}_{A}$ on $X$. More specifically,
the possible code word ${\mathbf c}^{A}$ is given
by
\begin{alignat*}{2}
c_{i}^{A}&=1&&\qquad\text{if $P_{i}\in A$}\\
c_{i}^{A}&=0&&\qquad\text{otherwise}.
\end{alignat*}
for some $A\subseteq X$.

In addition to
the usual vector space structure on ${\mathbb F}_{2}^{n}$,
we define a new operation
\[{\mathbf c}^{A}\wedge{\mathbf c}^{B}
={\mathbf c}^{A\cap B}.\]
Thus, if ${\mathbf x},{\mathbf y}\in {\mathbb F}_{2}^{n}$,
\[(x_{0},x_{1},\dots,x_{n-1})\wedge(y_{0},y_{1},\dots,y_{n-1})
=(x_{0}y_{0},x_{1}y_{1},\dots,x_{n-1}y_{n-1}).\]
Finally we consider the collection of $d$ hyperplanes
\[\pi_{j}=\{{\mathbf p}\in X:p_{j}=0\}\ \  [1\leq j\leq d]\]
in ${\mathbb F}_{2}^{n}$
and the corresponding indicator functions
\[{\mathbf h}^{j}={\mathbf c}^{\pi_{j}},\]
together with the special vector
\[{\mathbf h}^{0}={\mathbf c}^{X}=(1,1,\dots,1).\]
\begin{exercise} Suppose that
${\mathbf x},{\mathbf y},{\mathbf z}\in {\mathbb F}_{2}^{n}$
and $A,B\subseteq X$.

(i) Show that
${\mathbf x}\wedge{\mathbf y}={\mathbf y}\wedge{\mathbf x}$.

(ii) Show that
$({\mathbf x}+{\mathbf y})\wedge{\mathbf z}
={\mathbf x}\wedge{\mathbf z}+{\mathbf y}\wedge{\mathbf z}$.

(iii) Show that ${\mathbf h}^{0}\wedge{\mathbf x}={\mathbf x}$.

(iv) If ${\mathbf c}^{A}+{\mathbf c}^{B}={\mathbf c}^{E}$,
find $E$ in terms of $A$ and $B$.

(v) If ${\mathbf h}^{0}+{\mathbf c}^{A}={\mathbf c}^{E}$,
find $E$ in terms of $A$.
\end{exercise}

We refer to ${\mathcal A}_{0}=\{{\mathbf h}^{0}\}$ as the
set of terms of order zero.
If ${\mathcal A}_{k}$ is the set of terms of order at most $k$,
then the set ${\mathcal A}_{k+1}$ of terms of order at most $k+1$
is defined by
\[{\mathcal A}_{k+1}=\{{\mathbf a}\wedge{\mathbf h}^{j}:
{\mathbf a}\in {\mathcal A}_{k},\ 1\leq j\leq d\}.\]
Less formally, but more clearly, the elements
of order 1 are the ${\mathbf h}^{i}$, the elements
of order 2 are the ${\mathbf h}^{i}\wedge{\mathbf h}^{j} $
with $i<j$, the elements
of order 3 are the
${\mathbf h}^{i}\wedge{\mathbf h}^{j}\wedge{\mathbf h}^{k}$
with $i<j<k$ and so on.
\begin{definition} Using the notation established
above, the Reed-Muller code  $RM(d,r)$ is the linear code
(i.e. subspace of ${\mathbb F}_{2}^{n}$) generated
by the terms of order $r$ or less.
\end{definition}

Although the formal definition of the Reed-Muller codes
looks pretty impenetrable at first sight, once
we have looked at sufficiently many examples
it should become clear what is going on.
\begin{example}
(i) The $RM(3,0)$ code is the repetition
code of length $8$.

(ii) The $RM(3,1)$ code is the parity check
extension of Hamming's original code.

(iii) The $RM(3,2)$ code is the paper tape code
of length $8$.

(iii) The $RM(3,3)$ code is the trivial code
consisting of all the elements of
${\mathbb F}^{3}_{2}$.
\end{example}

We now prove the key properties of the Reed-Muller
codes. We use the notation established above.
\begin{theorem}\label{theorem Reed}
(i) The elements of order $d$
or less (that is the collection of all possible wedge products
formed from the ${\mathbf h}^{i}$) span
${\mathbb F}^{n}_{2}$.

(ii) The elements of order $d$
or less are linearly independent.

(iii) The dimension of the Reed-Muller code $RM(d,r)$
is
\[\binom{d}{0}+\binom{d}{1}+\binom{d}{2}
+\dots+\binom{d}{r}.\]

(iv) Using the bar product notation, we
have
\[RM(d,r)=RM(d-1,r)|RM(d-1,r-1).\]

(v) The minimum weight of $RM(d,r)$ is exactly
$2^{d-r}$.
\end{theorem}
\begin{exercise} The Mariner mission to Mars used
the $RM(5,1)$ code. What was its information
rate? What proportion of errors could it correct
in a single code word?
\end{exercise}
\begin{exercise} Show that the $RM(d,d-2)$ code is the
parity extension code of the Hamming $(N,N-d)$ code
with $N=2^{d}-1$. (This is useful because we often
want codes of length $2^{d}$.)
\end{exercise}
\section{Polynomials and fields} This section is
starred. Its
object is to make plausible the few facts from
modern\footnote{Modern, that is, in 1920.} algebra
that we shall need. They were covered, along with
much else, in various post-IA algebra courses,
but attendance at those courses is no more
required for this course than is reading Joyce's
\emph{Ulysses} before going for a night out at
an Irish pub. Anyone capable of criticising
the imprecision and general slackness of the
account that follows obviously can do better
themselves and should rewrite this section in an appropriate
manner.

A field $K$ is an object equipped with addition
and multiplication which follow the same rules
as do addition and multiplication in ${\mathbb R}$.
The only rule which will cause us trouble
is
\begin{equation*}
\text{If $x\in K$ and $x\neq 0$, then we can find
$y\in K$ such that $xy=1$.}\tag*{$\bigstar$}
\end{equation*}
Obvious examples of fields include ${\mathbb R}$, ${\mathbb C}$
and ${\mathbb F}_{2}$.

We are particularly interested in polynomials over
fields, but here an interesting difficulty arises.
\begin{example}
We have $t^{2}+t=0$ for all $t\in{\mathbb F}_{2}$.
\end{example}
To get round this, we distinguish between the
polynomial in the `indeterminate' $X$
\[P(X)=\sum_{j=0}^{n}a_{j}X^{j}\]
with coefficients $a_{j}\in K$ and its
evaluation $P(t)=\sum_{j=0}^{n}a_{j}t^{j}$
for some $t\in K$. We manipulate polynomials in $X$
according to the standard rules for polynomials,
but say that
\[\sum_{j=0}^{n}a_{j}X^{j}=0\]
if and only if $a_{j}=0$ for all $j$.
Thus $X^{2}+X$ is a non-zero polynomial
over ${\mathbb F}_{2}$ all of whose values are zero.

The following result is familiar, in essence,
from school mathematics.
\begin{lemma}{\bf [Remainder theorem]}\label{remainder}
(i) If $P$ is a polynomial over a field $K$
and $a\in K$, then we can find a polynomial $Q$ and an
$r\in K$ such that
\[P(X)=(X-a)Q(X)+r.\]

(ii) If $P$ is a polynomial over a field $K$
and $a\in K$ is such that $P(a)=0$, then
we can find a polynomial $Q$ such that
\[P(X)=(X-a)Q(X).\]
\end{lemma}

The key to much of the elementary theory of
polynomials lies in the fact that we can apply
Euclid's algorithm to obtain results like the
following.
\begin{theorem}\label{greatest common}
Suppose that $\mathcal{P}$ is a
set of polynomials which contains at least one
non-zero polynomial and has the following properties.

(i) If $Q$ is any polynomial and $P\in\mathcal{P}$,
then the product $PQ\in\mathcal{P}$.

(ii) If $P_{1},P_{2}\in \mathcal{P}$,
then $P_{1}+P_{2}\in \mathcal{P}$.

Then we can find a non-zero $P_{0}\in\mathcal{P}$ which
divides every $P\in\mathcal{P}$.
\end{theorem}
\begin{proof} Consider a non-zero polynomial $P_{0}$
of smallest degree in ${\mathcal P}$.
\end{proof}

Recall that the polynomial $P(X)=X^{2}+1$ has no roots
in ${\mathbb R}$ (that is $P(t)\neq 0$ for all $t\in{\mathbb R}$).
However, by considering the collection
of formal expressions $a+bi$ $[a,b\in{\mathbb R}]$
with the
obvious formal definitions of addition and multiplication
and subject to the further
condition $i^{2}+1=0$, we obtain a field
${\mathbb C}\supseteq{\mathbb R}$ in which $P$ has a root
(since $P(i)=0$).
We can perform a similar trick with other fields.
\begin{example} If $P(X)=X^{2}+X+1$, then $P$ has
no roots in ${\mathbb F}_{2}$. However, if we consider
\[{\mathbb F}_{2}[\omega]=
\{0,\ 1,\ \omega,\ 1+\omega\}\]
with
obvious formal definitions of addition and multiplication
and subject to the further
condition $\omega^{2}+\omega+1=0$, then
${\mathbb F}_{2}[\omega]$ is a field containing ${\mathbb F}_{2}$
in which $P$ has a root (since $P(\omega)=0$).
\end{example}
\begin{proof}  The only thing we really
need prove is that
${\mathbb F}_{2}[\omega]$ is a field and to
do that the only thing we need to prove is that
$\bigstar$ holds. Since
\[(1+\omega)\omega=1\]
this is easy.
\end{proof}

In order to state a correct generalisation of the
ideas of the previous paragraph we need a preliminary
definition.
\begin{definition} If $P$ is a polynomial over a field $K$,
we say that $P$ is \emph{reducible} if there exists
a non-constant polynomial $Q$ of degree strictly
less than $P$ which divides $P$. If $P$ is a
non-constant polynomial which is not
reducible, then $P$ is \emph{irreducible}.
\end{definition}
\begin{theorem}\label{add ideal} If $P$ is an irreducible
polynomial of degree $n\geq 2$ over a field $K$,
then $P$ has
no roots in $K$. However, if we consider
\[K[\omega]=
\left\{\sum_{j=0}^{n-1}a_{j}\omega^{j}:
a_{j}\in K\right\}\]
with the
obvious formal definitions of addition and multiplication
and subject to the further
condition $P(\omega)=0$, then
$K[\omega]$ is a field containing $K$
in which $P$ has a root.
\end{theorem}
\begin{proof}  The only thing we really
need prove is that
$K[\omega]$ is a field and to
do that the only thing we need to prove is that
$\bigstar$ holds. Let $Q$ be a non-zero
polynomial of
degree at most $n-1$.
Since $P$ is irreducible,
the polynomials $P$ and $Q$ have no common
factor of degree $1$ or more. Hence, by
Euclid's algorithm, we can find polynomials
$R$ and $S$ such that
\[R(X)Q(X)+S(X)P(X)=1\]
and so $R(\omega)Q(\omega)+S(\omega)P(\omega)=1$.
But $P(\omega)=0$, so $R(\omega)Q(\omega)=1$
and we have proved $\bigstar$.
\end{proof}
In a proper algebra course we would simply define
\[K[\omega]=K[X]/(P(X))\]
where $(P(X))$ is the ideal generated by $P(X)$.
This is a cleaner procedure which avoids the
use of such phrases as `the
obvious formal definitions of addition and multiplication'
but the underlying idea remains the same.
\begin{lemma} If $P$ is a polynomial over a field $K$
which does not factorise completely into linear
factors, then we can find a field $L\supseteq K$
in which $P$ has more linear factors.
\end{lemma}
\begin{proof} Factor $P$ into irreducible factors
and choose a factor $Q$ which is not linear.
By Theorem~\ref{add ideal}, we can find a field $L\supseteq K$
in which $Q$ has a root $\alpha$ say
and so, by Lemma~\ref{remainder},
a linear factor $X-\alpha$. Since any linear factor of
$P$ in $K$ remains a factor in the bigger field $L$,
we are done.
\end{proof}
\begin{theorem}\label{splits one}
If $P$ is a polynomial over a field $K$,
then we can find a field $L\supseteq K$
in which $P$ factorises completely
into linear factors.
\end{theorem}

We shall be interested in finite fields (that is
fields $K$ with only a finite number of elements).
A glance at our method of proving Theorem~\ref{splits one}
shows that the following result holds.
\begin{lemma}\label{splits two}
If $P$ is a polynomial over a finite field $K$,
then we can find a finite field $L\supseteq K$
in which $P$ factorises completely.
\end{lemma}

In this context, we note yet another useful
simple consequence of Euclid's algorithm.
\begin{lemma} Suppose that
$P$ is an irreducible
polynomial over a field $K$
which has a linear factor $X-\alpha$
in some field $L\supseteq K$.
If $Q$ is a polynomial over $K$ which
has the factor $X-\alpha$ in $L$,
then $P$ divides $Q$.
\end{lemma}

We shall need a lemma on repeated roots.
\begin{lemma} Let $K$ be a field. If
$P(X)=\sum_{j=0}^{n}a_{j}X^{j}$ is a polynomial
over $K$, we define $P'(X)=\sum_{j=1}^{n}ja_{j}X^{j-1}$.

(i) If $P$ and $Q$ are polynomials, $(P+Q)'=P'+Q'$
and $(PQ)'=P'Q+PQ'$.

(ii) If $P$ and $Q$ are polynomials with
$P(X)=(X-a)^{2}Q(X)$, then
\[P'(X)=2(X-a)Q(X)+(X-a)^{2}Q'(X).\]

(iii) If $P$ is divisible by $(X-a)^{2}$, then $P(a)=P'(a)=0$.
\end{lemma}

If $L$ is a field containing ${\mathbb F}_{2}$, then
$2y=(1+1)y=0y=0$ for all $y\in L$. We can thus
deduce the following result which will be used
in the next section.
\begin{lemma}\label{no repeat}
If $L$ is a field containing ${\mathbb F}_{2}$
and $n$ is an odd integer, then
$X^{n}-1$ can have no repeated linear factors
as a polynomial over $L$.
\end{lemma}

We also need a result on roots of unity
given as part~(v) of the next lemma.
\begin{lemma}\label{primitive} (i) If $G$
is a finite Abelian
group and $x,y\in G$ have coprime orders
$r$ and $s$, then $xy$ has order $rs$.

(ii) If $G$ is a finite Abelian
group and $x,y\in G$ have orders
$r$ and $s$, then we can find an element
$z$ of $G$ with order the lowest common multiple
of $r$ and $s$.

(iii) If $G$ is a finite Abelian
group, then there exists an $N$ and
an $h\in G$ such that $h$ has order $N$
and $g^{N}=e$ for all $g\in G$.

(iv) If $G$ is a finite subset of a field $K$
which is a group under multiplication, then
$G$ is cyclic.

(v) Suppose $n$ is an odd integer.
If $L$ is a field containing ${\mathbb F}_{2}$
such that $X^{n}-1$ factorises completely
into linear terms, then we can find
an $\omega\in L$ such that the roots
of $X^{n}-1$ are $1$, $\omega$, $\omega^{2}$,
\dots $\omega^{n-1}$. (We call $\omega$ a
\emph{primitive} $n$th root of unity.)
\end{lemma}
\begin{proof} (ii) Consider $z=x^{u}y^{v}$
where $u$ is a divisor of $r$, $v$ is a divisor
of $s$, $r/u$ and $s/v$ are coprime and
$rs/(uv)=\lcm(r,s)$.

(iii) Let $h$ be an element of highest order
in $G$ and use (ii).

(iv) By (iii) we can find an integer
$N$ and a $h\in G$ such that $h$
has order $N$ and any element $g\in G$
satisfies $g^{N}=1$. Thus $X^{N}-1$
has a linear factor $X-g$ for each $g\in G$
and so $\prod_{g\in G}(X-g)$ divides $X^{N}-1$.
It follows that the order $|G|$ of $G$ cannot
exceed $N$. But by Lagrange's theorem $N$ divides $G$.
Thus $|G|=N$ and $g$ generates $G$.

(v) Observe that $G=\{\omega:\omega^{n}=1\}$ is
an Abelian group with exactly $n$ elements
(since $X^{n}-1$ has no repeated roots) and
use (iv).
\end{proof}

Here is another interesting consequence of
Lemma~\ref{primitive}~(iv).
\begin{lemma}\label{primitive field}
If $K$ is a field with $m$ elements,
then there is an
element $k$ of $K$ such that
\[K=\{0\}\cup\{k^{r}:0\leq r\leq m-2\}\]
and $k^{m-1}=1$.
\end{lemma}
\begin{proof} Observe that $K\setminus\{0\}$ forms 
an Abelian group
under multiplication.
\end{proof}
We call an element $k$ with the properties
given in Lemma~\ref{primitive field}
a \emph{primitive element} of $K$.
\begin{exercise} Find all the primitive elements of ${\mathbb F}_{7}$.
\end{exercise}

With this hint, it is not hard to show that there is
indeed a field with $2^{n}$ elements
containing ${\mathbb F}_{2}$.
\begin{lemma}\label{Galois}
Let $L$ be some field containing
${\mathbb F}_{2}$ in which $X^{2^{n}-1}-1=0$
factorises completely. Then
\[K=\{x\in L:x^{2^{n}}=x\}\]
is a field with $2^{n}$ elements
containing ${\mathbb F}_{2}$.
\end{lemma}

Lemma~\ref{primitive field} shows that
there is (up to field isomorphism) only one
field with $2^{n}$ elements
containing ${\mathbb F}_{2}$.
We call it ${\mathbb F}_{2^{n}}$.
\section{Cyclic codes} In this section, we
discuss a subclass of linear codes, the
so-called \emph{cyclic codes}.
\begin{definition} A linear code $C$
in ${\mathbb F}^{n}_{2}$
is called \emph{cyclic} if
\[(a_{0},a_{1},\dots,a_{n-2},a_{n-1})\in C
\Rightarrow (a_{1},a_{2},\dots,a_{n-1},a_{0})\in C.\]
\end{definition}
Let us establish a correspondence between
${\mathbb F}^{n}_{2}$ and the polynomials
on ${\mathbb F}_{2}$ modulo $X^{n}-1$
by setting
\[P_{\mathbf a}=\sum_{j=0}^{n-1}a_{j}X^{j}\]
whenever ${\mathbf a}\in {\mathbb F}_{2}^{n}$.
(Of course, $X^{n}-1=X^{n}+1$ but in this context
the first expression seems more natural.)
\begin{exercise} With the notation
just established, show that

(i) $P_{\mathbf a}+P_{\mathbf b}=P_{{\mathbf a}+{\mathbf b}}$,

(ii) $P_{\mathbf a}=0$ if and only if ${\mathbf a}={\boldsymbol 0}$.
\end{exercise}

\begin{lemma} A  code $C$
in ${\mathbb F}^{n}_{2}$
is cyclic if and only if
${\mathcal P}_{C}=\{P_{\mathbf a}:{\mathbf a}\in C\}$
satisfies the following two
conditions (working modulo $X^{n}-1$).

(i) If $f,g\in {\mathcal P}_{C}$, then $f+g\in {\mathcal P}_{C}$.

(ii) If $f\in {\mathcal P}_{C}$ and $g$ is
any polynomial, then the product $fg\in{\mathcal P}_{C}$.
\end{lemma}
\noindent
(In the language of abstract algebra, $C$
is cyclic if and only if
${\mathcal P}_{C}$ is an \emph{ideal} of the
quotient ring ${\mathbb F}_{2}[X]/(X^{n}-1)$.)

From now on we shall talk of the code word $f(X)$
when we mean the code word ${\mathbf a}$ with
$P_{\mathbf a}(X)=f(X)$. An application
of Euclid's algorithm gives the following useful
result.
\begin{lemma} A code $C$ of length $n$ is
cyclic if and only if (working modulo
$X^{n}-1$, and using the conventions established
above) there exists a polynomial $g$ such that
\[C=\{f(X)g(X):\text{$f$ a polynomial}\}\]
\end{lemma}

\noindent (In the language of abstract algebra,
${\mathbb F}_{2}[X]$ is a Euclidean domain
and so a principal ideal domain. Thus the
quotient ${\mathbb F}_{2}[X]/(X^{n}-1)$ is
a principal ideal domain.) We call $g(X)$
a generator polynomial for $C$.

\begin{lemma} A polynomial $g$ is a generator
for a cyclic code of length $n$ if and
only if it divides $X^{n}-1$.
\end{lemma}
Thus we must seek generators among the
factors of $X^{n}-1=X^{n}+1$.
If there are no conditions on $n$,
the result can be rather disappointing.
\begin{exercise} If we work with polynomials over
${\mathbb F}_{2}$, then
\[X^{2^{r}}+1=(X+1)^{2^{r}}.\]
\end{exercise}

In order to avoid this problem
and to be able to make use of
Lemma~\ref{no repeat}, we shall
take $n$ odd from now on. (In this case,
the cyclic codes are said to be separable.)
Notice that the task of finding
irreducible factors  (that is factors with
no further factorisation) is a finite one.

\begin{lemma} Consider codes of length $n$.
Suppose that $g(X)h(X)=X^{n}-1$. Then $g$
is a generator of a cyclic code $C$
and $h$ is a generator for a cyclic code
which is the reverse of $C^{\perp}$.
\end{lemma}
As an immediate corollary, we have the following
remark.
\begin{lemma} The dual of a cyclic code
is itself cyclic.
\end{lemma}
\begin{lemma}\label{cyclic basis}
If a cyclic code $C$ of length
$n$ has generator $g$ of degree $n-r$ then
$g(X)$, $Xg(X)$, \dots, $X^{r-1}g(X)$
form a basis for $C$.
\end{lemma}
Cyclic codes are thus easy to specify
(we just need to write down the generator
polynomial $g$) and to encode.

We know that $X^{n}+1$ factorises completely
over some
larger finite field and, since $n$ is odd,
we know, by Lemma~\ref{no repeat},
that it has no
repeated factors. The same is therefore
true for any polynomial dividing it.
\begin{lemma} Suppose that $g$ is a
generator of a cyclic code $C$ of
odd length $n$. Suppose further
that $g$ factorises
completely into linear factors in some
field $K$ containing ${\mathbb F}_{2}$.
If $g=g_{1}g_{2}\dots g_{k}$ with each $g_{j}$ irreducible
over ${\mathbb F}_{2}$
and $A$ is a subset of the set of all the roots of
all the $g_{j}$ 
and containing at least one root
of each $g_{j}$ $[1\leq j\leq k]$,
then
\[C=\{f\in {\mathbb F}_{2}[X]:f(\alpha)=0 
\ \text{for all $\alpha\in A$}\}.\]
\end{lemma}
\begin{definition} A \emph{defining set} for a cyclic
code $C$ is a set $A$ of elements in some
field $K$ containing ${\mathbb F}_{2}$
such that $f\in{\mathbb F}_{2}[X]$ belongs
to $C$ if and only if $f(\alpha)=0$
for all $\alpha\in A$.
\end{definition}
\noindent
(Note that, if $C$ has length $n$,
$A$ must be a set of zeros
of $X^{n}-1$.)

\begin{lemma}\label{field check} Suppose that
\[A=\{\alpha_{1},\alpha_{2},\dots,\alpha_{r}\}\]
is a defining set for a cyclic
code $C$  in some
field $K$ containing ${\mathbb F}_{2}$.
Let $B$ be the $n\times r$ matrix over $K$
whose $j$th column is
\[(1,\alpha_{j},\alpha_{j}^{2},\dots,\alpha_{j}^{n-1})^{T}\]
Then a vector ${\mathbf a}\in{\mathbb F}_{2}^{n}$
is a code word in $C$ if and only if
\[{\mathbf a}B={\boldsymbol 0}\]
in $K$.
\end{lemma}
\noindent
The columns in $B$ are not parity checks in the usual
sense since the code entries lie in ${\mathbb F}_{2}$
and the computations take place in the larger field $K$.

With this background we can
discuss a famous family of codes known as
the BCH (Bose, Ray-Chaudhuri, Hocquenghem) codes.
Recall that a primitive $n$th root of unity is
an root $\alpha$ of $X^{n}-1=0$
such that every root is a power of $\alpha$.
\begin{definition}\label{definition BCH}
Suppose that $n$ is odd
and $K$ is a field containing
${\mathbb F}_{2}$ in which $X^{n}-1$ factorises into
linear factors. Suppose that
$\alpha\in K$ is a primitive
$n$th root of unity.
A cyclic code $C$ with defining set
\[A=\{\alpha,\alpha^{2},\dots,\alpha^{\delta-1}\}\]
is a \emph{BCH code of design distance} $\delta$.
\end{definition}
\noindent
Note that the rank of $C$ will be $n-k$, where $k$
is the degree of the product of
those irreducible factors
of $X^{n}-1$ over ${\mathbb F}_{2}$ which have a
zero in $A$. Notice also
that $k$ may be very much larger
than $\delta$.

\begin{example}\label{Hamming BCH}
(i) If $K$ is a field containing
${\mathbb F}_{2}$, then $(a+b)^{2}=a^{2}+b^{2}$
for all $a,b\in K$.

(ii) If $P\in {\mathbb F}_{2}[X]$ and $K$ is a field containing
${\mathbb F}_{2}$, then $P(a)^{2}=P(a^{2})$
for all $a\in K$.

(iii) Let $K$ be a field containing
${\mathbb F}_{2}$ in which $X^{7}-1$ factorises
into linear factors. If $\beta$ is a root of $X^{3}+X+1$
in $K$, then $\beta$ is a primitive root of unity
and $\beta^{2}$ is also a root of $X^{3}+X+1$.

(iv) We continue with the notation (iii).
The BCH
code with $\{\beta,\beta^{2}\}$ as defining set
is Hamming's original (7,4) code.
\end{example}

The next theorem contains the key fact about BCH codes.
\begin{theorem}\label{BCH}
The minimum distance for a BCH code
is at least as great as the design distance.
\end{theorem}

Our proof of Theorem~\ref{BCH} relies on showing that
the matrix $B$ of Lemma~\ref{field check} is of full rank
for a BCH. To do this we use a result which every undergraduate
knew in 1950.
\begin{lemma}{\bf [The van der Monde determinant]}\label{L;van der Monde}
We work over a field $K$. The determinant
\begin{equation*}
\begin{vmatrix}
1&1&1&\hdots&1\\
x_{1}&x_{2}&x_{3}&\hdots&x_{n}\\
x_{1}^{2}&x_{2}^{2}&x_{3}^{2}&\hdots&x_{n}^{2}\\
\vdots&\vdots&\vdots&\ddots&\vdots\\
x_{1}^{n-1}&x_{2}^{n-1}&x_{3}^{n-1}&\hdots&x_{n}^{n-1}
\end{vmatrix}
=\prod_{1\leq j<i\leq n}(x_{i}-x_{j}).
\end{equation*}
\end{lemma}

How can we construct a decoder for a BCH code? From
now on, until the end of this section, we shall
suppose that we are using the BCH code $C$ described
in Definition~\ref{definition BCH}. In
particular, $C$ will have length $n$ and
defining set
\[A=\{\alpha,\alpha^{2},\dots,\alpha^{\delta-1}\}\]
where $\alpha$ is a primitive $n$th root of unity
in $K$. Let $t$ be the largest integer with $2t+1\leq\delta$.
We show how we can correct up to $t$ errors.


Suppose that a codeword
${\mathbf c}=(c_{0},c_{1},\dots,c_{n-1})$ is
transmitted and that the string received is
${\mathbf r}$. We write
${\mathbf e}={\mathbf r}-{\mathbf c}$
and assume that
\[{\mathcal E}=\{0\leq j\leq n-1:e_{j}\neq 0\}\]
has no more than $t$ members.
In other words, ${\mathbf e}$ is the error vector
and we assume that there are no more than
$t$ errors.
We write
\begin{align*}
c(X)&=\sum_{j=0}^{n-1}c_{j}X^{j},\\
r(X)&=\sum_{j=0}^{n-1}r_{j}X^{j},\\
e(X)&=\sum_{j=0}^{n-1}e_{j}X^{j}.
\end{align*}


\begin{definition} The \emph{error locator polynomial} is
\[\sigma(X)=\prod_{j\in{\mathcal E}}(1-\alpha^{j}X)\]
and the \emph{error co-locator} is
\[\omega(X)=\sum_{i=0}^{n-1}e_{i}\alpha^{i}
\prod_{j\in{\mathcal E},\ j\neq i}(1-\alpha^{j}X).\]
\end{definition}
\noindent
Informally, we write
\[\omega(X)=\sum_{i=0}^{n-1}e_{i}\alpha^{i}
\frac{\sigma(X)}{1-\alpha^{i}X}.\]
We take $\omega(X)=\sum_{j}\omega_{j}X^{j}$ and
$\sigma(X)=\sum_{j}\sigma_{j}X^{j}$. Note that
$\omega$ has degree at most $t-1$ and $\sigma$
degree at most $t$. Note that we know that
$\sigma_{0}=1$ so both the polynomials
$\omega$ and $\sigma$ have $t$ unknown coefficients.
\begin{lemma} If the error locator polynomial
is given the value of ${\mathbf e}$ and so of ${\mathbf c}$
can be obtained directly.
\end{lemma}


We wish to make use of relations of the form
\[\frac{1}{1-\alpha^{j}X}=\sum_{r=0}^{\infty}(\alpha^{j}X)^{r}.\]
Unfortunately, it is not clear what meaning to assign
to such a relation. One way round is to work modulo $Z^{2t}$
(more formally, to work in $K[Z]/(Z^{2t})$). We then
have $Z^{u}\equiv 0$ for all integers $u\geq 2t$.
\begin{lemma} If we work modulo $Z^{2t}$ then
\[(1-\alpha^{j}Z)\sum_{m=0}^{2t-1}(\alpha^{j}Z)^{m}\equiv 1.\]
\end{lemma}
Thus, if we work modulo
$Z^{2t}$, as we shall from now on,
we may define
\[\frac{1}{1-\alpha^{j}Z}=\sum_{m=0}^{2t-1}(\alpha^{j}Z)^{m}.\]

\begin{lemma}\label{BCH decode}
With the conventions already introduced.

(i) ${\displaystyle
\frac{\omega(Z)}{\sigma(Z)}\equiv
\sum_{m=0}^{2t-1}Z^{m}e(\alpha^{m+1})}$.

(ii) $e(\alpha^{m})=r(\alpha^{m})$ for all $0\leq m\leq 2t-1$.

(iii) ${\displaystyle
\frac{\omega(Z)}{\sigma(Z)}\equiv
\sum_{m=0}^{2t-1}Z^{m}r(\alpha^{m+1})}$.

(iv) $\omega(Z)\equiv\sum_{m=0}^{2t-1}Z^{m}r(\alpha^{m+1})
\sigma(Z).$

(v) ${\displaystyle 
\omega_{j}=\sum_{u+v=j}r(\alpha^{u+1})\sigma_{v}}$ for all
$0\leq j\leq t-1$.

(vi) ${\displaystyle 
0=\sum_{u+v=j}r(\alpha^{u+1})\sigma_{v}}$ for all $t\leq j\leq 2t-1$.

(vii) The conditions in (vi) determine $\sigma$ completely.
\end{lemma}
\noindent
Part~(vi) of Lemma~\ref{BCH decode} completes our search
for a decoding method, since $\sigma$ determines ${\mathcal E}$,
${\mathcal E}$ determines ${\mathbf e}$ and ${\mathbf e}$
determines ${\mathbf c}$. It is worth noting that
the system of equations in part (v) suffice to determine
the pair $\sigma$ and $\omega$ directly.

Compact disc players use BCH codes. Of course,  errors
are likely to occur in bursts (corresponding to
scratches etc) and this is dealt with by
distributing the bits (digits) in a single codeword
over a much longer stretch of track. The code used
can correct a burst of 4000 consecutive errors
(2.5 mm of track).


Unfortunately, none of the codes we have considered
work anywhere near the Shannon bound
(see Theorem~\ref{T;Shannon}). We might suspect
that this is because they are linear,
but Elias has shown that this is not the case.
(We just state the result without proof.)
\begin{theorem}
In  Theorem~\ref{T;Shannon}  we can replace `code'
by `linear code'.
\end{theorem}
\noindent

The advance of computational power and the 
ingenuity of the discoverers\footnote{People
like David MacKay, now better known for his
superb `Sustainable Energy Without the Hot Air'
--- rush out and read it.} have lead to
new codes which appear to come close to
the Shannon bounds. But that is another story.  

Just as pure algebra has contributed greatly
to the study of error correcting codes, so
the study of error correcting codes  has contributed
greatly to the study of pure algebra.
The story of one such contribution is set out
in T.~M.~Thompson's
\emph{From Error-correcting Codes through Sphere Packings
to Simple Groups}~\cite{Thompson} --- a good, not too
mathematical, account of
the discovery of the last
sporadic simple groups by Conway and others.
\section{Shift registers}\label{section shift}
In this section we move
towards cryptography, but the topic discussed will
turn out to have connections with the decoding
of BCH codes as well.

\begin{definition}\label{general feed}
A \emph{general feedback shift register} is a map
$f:{\mathbb F}_{2}^{d}\rightarrow {\mathbb F}_{2}^{d}$
given by
\[f(x_{0},x_{1},\dots,x_{d-2},x_{d-1})
=(x_{1},x_{2},\dots,x_{d-1},C(x_{0},x_{1},\dots,x_{d-2},x_{d-1}))\]
with $C$ a map $C:{\mathbb F}_{2}^{d}\rightarrow {\mathbb F}_{2}$.
The \emph{stream} associated to an \emph{initial fill}
$(y_{0},y_{1},\dots,y_{d-1})$ is the sequence
\[y_{0},y_{1},\dots,y_{j},y_{j+1},\dots
\ \text{with}\ y_{n}=C(y_{n-d},y_{n-d+1},\dots,y_{n-1})
\ \text{for all $n\geq d$.}\]
\end{definition}

\begin{example} If the general feedback shift $f$
given in Definition~\ref{general feed} is a permutation,
then $C$ is linear in the first variable, i.e.
\[C(x_{0},x_{1},\dots,x_{d-2},x_{d-1})
=x_{0}+C'(x_{1},x_{2},\dots,x_{d-2},x_{d-1}).\]
\end{example}
\begin{definition}~\label{linear feed}
We say that the function $f$
of Definition~\ref{general feed} is
a \emph{linear feedback register} if
\[C(x_{0},x_{1},\dots,x_{d-1})=
a_{0}x_{0}+a_{1}x_{1}+\ldots+a_{d-1}x_{d-1},\]
with $a_{0}=1$.
\end{definition}
\begin{exercise} Discuss briefly the effect
of omitting the condition
$a_{0}=1$ from Definition~\ref{linear feed}.
\end{exercise}

The discussion of the linear recurrence
\[x_{n}=a_{0}x_{n-d}+a_{1}x_{n-d+1}+\dots+a_{d-1}x_{n-1}\]
over ${\mathbb F}_{2}$ follows the IA discussion
of the same problem over ${\mathbb R}$ but
is complicated by the fact that
\[n^{2}=n\]
in ${\mathbb F}_{2}$. We assume that $a_{0}\neq 0$
and consider the \emph{auxiliary polynomial}
\[C(X)=X^{d}-a_{d-1}X^{d-1}-\dots-a_{1}X-a_{0}.\]
In the exercise below, ${\displaystyle \binom{n}{v}}$
is the appropriate polynomial in $n$.
\begin{exercise}\label{Solve recurrence}
Consider the linear recurrence
\begin{equation*}
x_{n}=a_{0}x_{n-d}+a_{1}x_{n-d+1}+\ldots+a_{d-1}x_{n-1}\tag*{$\bigstar$}
\end{equation*}
with $a_{j}\in {\mathbb F}_{2}$ and $a_{0}\neq 0$.

(i) Suppose $K$ is a field containing ${\mathbb F}_{2}$
such that the auxiliary polynomial $C$ has a root $\alpha$
in $K$. Show that $x_{n}=\alpha^{n}$ is a solution of $\bigstar$ in $K$.

(ii) Suppose $K$ is a field containing ${\mathbb F}_{2}$
such that the auxiliary polynomial $C$ has
$d$ distinct roots $\alpha_{1}$, $\alpha_{2}$,
\dots, $\alpha_{d}$ in $K$. Show that the general solution
of $\bigstar$ in $K$ is
\[x_{n}=\sum_{j=1}^{d}b_{j}\alpha_{j}^{n}\]
for some $b_{j}\in K$.
If $x_{0},x_{1},\dots,x_{d-1}\in {\mathbb F}_{2}$,
show that $x_{n}\in {\mathbb F}_{2}$ for all $n$.

(iii) Work out the first few lines of Pascal's triangle
modulo 2. Show that the functions
$f_{j}:{\mathbb Z}\rightarrow{\mathbb F}_{2}$
\[f_{j}(n)=\binom{n}{j}\]
are linearly independent in the sense that
\[\sum_{j=0}^{m}b_{j}f_{j}(n)=0\]
for all $n$ implies $b_{j}=0$ for $0\leq j\leq m$.

(iv) Suppose $K$ is a field containing ${\mathbb F}_{2}$
such that the auxiliary polynomial $C$ factorises
completely into linear factors. If the
root $\alpha_{u}$ has multiplicity $m(u)$, $[1\leq u\leq q]$,
show that the general solution
of $\bigstar$ in $K$ is
\[x_{n}=\sum_{u=1}^{q}\sum_{v=0}^{m(u)-1}
b_{u,v}\binom{n}{v}\alpha_{u}^{n}\]
for some $b_{u,v}\in K$.
If $x_{0},x_{1},\dots,x_{d-1}\in {\mathbb F}_{2}$,
show that $x_{n}\in {\mathbb F}_{2}$ for all $n$.
\end{exercise}

A strong link with the problem of BCH decoding
is provided by Theorem~\ref{quotient} below.
\begin{definition}
If we have a sequence (or stream) $x_{0}$, $x_{1}$,
$x_{2}$, \dots of elements of ${\mathbb F}_{2}$ then
its \emph{generating function} $G$ is given by
\[G(Z)=\sum_{n=0}^{\infty}x_{j}Z^{j}.\]
If the recurrence relation for
a linear feedback generator is
\[\sum_{j=0}^{d}c_{j}x_{n-j}=0\]
for $n\geq d$ with $c_{0},c_{d}\neq 0$
we call
\[C(z)=\sum_{j=0}^{d}c_{j}Z^{j}\]
the auxiliary polynomial of the generator.
\end{definition}
\begin{theorem}\label{quotient} The stream $(x_{n})$
comes from a linear feedback generator with
auxiliary polynomial $C$ if and only if
the generating function for the stream is (formally)
of the form
\[G(Z)=\frac{B(Z)}{C(Z)}\]
with $B$ a polynomial of degree strictly smaller than that
of $C$.
\end{theorem}
\noindent
If we can recover $C$ from $G$ then we have recovered
the linear feedback generator from the stream.

The link with BCH codes is established
by looking at Lemma~\ref{BCH decode}~(iii) and
making the following remark.
\begin{lemma}\label{how long}
If a stream $(x_{n})$
comes from a linear feedback generator with
auxiliary polynomial $C$ of degree $d$,
then $C$ is determined by the condition
\[G(Z)C(Z)\equiv B(Z)\mod{Z^{2d}}\]
with $B$ a polynomial of degree at most $d-1$.
\end{lemma}

We thus have the following problem.

\noindent{\bf Problem}
\emph{Given a generating function $G$
for a stream  and knowing that
\[G(Z)=\frac{B(Z)}{C(Z)}\]
with $B$ a polynomial of degree less than that
of $C$ and the constant term in $C$ is
$c_{0}=1$, recover $C$.}

\vspace{1\baselineskip}


\noindent\emph{The Berlekamp--Massey method}
In this method we do not assume that
the degree $d$ of $C$ is known.
The Berlekamp--Massey solution to this problem
is based on the observation that, since
\[\sum_{j=0}^{d}c_{j}x_{n-j}=0\]
(with $c_{0}=1$)
for all $n\geq d$, we have
\begin{equation*}
\begin{pmatrix}
x_{d}&x_{d-1}&\dots&x_{1}&x_{0}\\
x_{d+1}&x_{d}&\dots&x_{2}&x_{1}\\
\vdots&\vdots&\ddots&\vdots&\vdots\\
x_{2d}&x_{2d-1}&\dots&x_{d+1}&x_{d}
\end{pmatrix}
\begin{pmatrix}
1\\c_{1}\\ \vdots\\c_{d}
\end{pmatrix}
=\begin{pmatrix}
0\\0\\ \vdots\\0
\end{pmatrix}\,.
\tag*{$\bigstar$}
\end{equation*}

The Berlekamp--Massey method
tells us to look successively
at the matrices
\[A_{1}=(x_{0}),
\ A_{2}=
\begin{pmatrix}
x_{1}&x_{0}\\
x_{2}&x_{1}
\end{pmatrix},
\ A_{3}=
\begin{pmatrix}
x_{2}&x_{1}&x_{0}\\
x_{3}&x_{2}&x_{1}\\
x_{4}&x_{3}&x_{2}
\end{pmatrix},
\dots
\]
starting at $A_{r}$ if it is known that $r\geq d$.
For each $A_{j}$ we evaluate $\det A_{j}$.
If $\det A_{j}\neq 0$, then $j-1\neq d$.
If $\det A_{j}=0$, then $j-1$ is a good candidate
for $d$ so we solve $\bigstar$ on the assumption that
$d=j-1$. (Note that a one dimensional subspace of
${\mathbb F}^{d+1}$ contains only one non-zero
vector.) We then check our candidate for
$(c_{0},c_{1},\dots,c_{d})$ over as many
terms of the stream as we wish. If it fails the
test, we then know that $d\geq j$ and we start
again\footnote{Note that, over ${\mathbb F}_{2}$,
$\det A_{j}$ can only take two values so there will
be many false alarms. Note also that the determinant
may be evaluated much faster using reduction to
(rearranged) triangular form than by Cramer's rule
and that once the system is in (rearranged) triangular form
it is easy to solve the associated equations.}.


As we have stated it, the Berlekamp--Massey method
is not an algorithm in the strict sense
of the term although it becomes one if we put an
upper bound on the possible values of $d$. (A little
thought shows that, if no upper bound is put on $d$,
no algorithm is possible because, with a suitable
initial stream, a linear feedback register with
large $d$ can be made to produce a stream whose
initial values would be produced by
a linear feedback register with much smaller $d$.
For the same reason the Berlekamp--Massey method
will produce the $B$ of smallest degree which
gives $G$ and not necessarily the original $B$.)
In practice, however, the Berlekamp--Massey method
is very effective in cases when $d$ is unknown.

By careful arrangement of the work it is possible
to cut down considerably  on the labour involved.

The solution of linear equations gives us
a method of `secret sharing'\label{P;secret sharing}.
\begin{problem} It is not generally known that
CMS when reversed forms the initials of 
of `Secret Missile Command'. If the University is
attacked by HEFCE\footnote{An institution like
SPECTRE but without the charm.}, the Faculty Board
will retreat to a bunker known as
Meeting Room 23. Entry to the room involves
tapping out a positive integer $S$ (the secret)
known only to the Chairman of the Faculty Board.
Each of the $n$ members of the Faculty Board
knows a certain pair of numbers (their shadow)
and it is required that, in the absence
of the Chairman, any $k$ members of the
Faculty can reconstruct $S$ from their shadows,
but no $k-1$ members can do so. How can this
be done?
\end{problem}

Here is one neat solution. Suppose $S$ must lie between
$0$ and $N$ (it is sensible to choose $S$ at random).
The chairman chooses
a prime $p>N,\,n$. She then chooses integers
$a_{1}$, $a_{2}$, \dots, $a_{k-1}$ at random
and distinct integers
$x_{1}$, $x_{2}$, \dots, $x_{n}$ at random 
subject to $0\leq a_{j}\leq p-1$, 
$1\leq x_{j}\leq p-1$, sets $a_{0}=S$
and computes 
\[P(r)\equiv a_{0}+a_{1}x_{r}+a_{2}x_{r}^{2}+\dots+a_{k-1}x_{r}^{k-1}\mod{p}\]
choosing $0\leq P(r)\leq p-1$.
She then gives the $r$th member of the Faculty Board
the pair of numbers $\big(x_{r},P(r)\big)$ (the shadow pair), 
to be kept secret
from everybody else) and tells everybody the value of $p$.
She then burns her calculations.

Suppose that $k$ members of the Faculty Board
with shadow pairs
$\big(y_{j},Q(j)\big)=\big(x_{r_{j}},P(r_{j})\big)$ 
$[1\leq j\leq k]$ are together.
By the properties of the Van der Monde determinant
(see~Lemma~\ref{L;van der Monde})   
\begin{align*}
\begin{vmatrix}
1&y_{1}&y_{1}^{2}&\hdots&y_{1}^{k-1}\\
1&y_{2}&y_{2}^{2}&\hdots&y_{2}^{k-1}\\
1&y_{3}&y_{3}^{2}&\hdots&y_{3}^{k-1}\\
\vdots&\vdots&\vdots&\ddots&\vdots\\
1&y_{k}&y_{k}^{2}&\hdots&y_{k}^{k-1}
\end{vmatrix}
&\equiv
\begin{vmatrix}
1&1&1&\hdots&1\\
y_{1}&y_{2}&y_{3}&\hdots&y_{k}\\
y_{1}^{2}&y_{2}^{2}&y_{3}^{2}&\hdots&y_{k}^{2}\\
\vdots&\vdots&\vdots&\ddots&\vdots\\
y_{1}^{k-1}&y_{2}^{k-1}&y_{3}^{k-1}&\hdots&y_{k}^{k-1}
\end{vmatrix}\\
&\equiv\prod_{1\leq j<i\leq k-1}(y_{i}-y_{j})\not\equiv 0\mod{p}.
\end{align*}

Thus the system of equations
\begin{align*}
z_{0}+y_{1}z_{1}+y_{1}^{2}z_{2}+\hdots+y_{1}^{k-1}z_{k-1}
&\equiv Q_{1}\\
z_{0}+y_{2}z_{1}+y_{2}^{2}z_{2}+\hdots+y_{2}^{k-1}z_{k-1}
&\equiv Q_{2}\\
z_{0}+y_{3}z_{1}+y_{3}^{2}z_{2}+\hdots+y_{3}^{k-1}z_{k-1}
&\equiv Q_{3}\\
&\ \,\vdots\\
z_{0}+y_{k}z_{1}+y_{k}^{2}z_{2}+\hdots+y_{k}^{k-1}z_{k-1}&\equiv Q_{k}
\end{align*}
has a unique solution ${\mathbf z}$. But we know that
${\mathbf a}$ is a solution, so ${\mathbf z}={\mathbf a}$
and the secret $S=z_{0}$.

On the other hand,
\[
\begin{vmatrix}
y_{1}&y_{1}^{2}&\hdots&y_{1}^{k-1}\\
y_{2}&y_{2}^{2}&\hdots&y_{2}^{k-1}\\
y_{3}&y_{3}^{2}&\hdots&y_{3}^{k-1}\\
\vdots&\vdots&\ddots&\vdots\\
y_{k-1}&y_{k-1}^{2}&\hdots&y_{k-1}^{k-1}
\end{vmatrix}
\equiv y_{1}y_{2}\ldots y_{k-1}
\prod_{1\leq j<i\leq k-1}(y_{i}-y_{j})\not\equiv 0\mod{p},\]
so the system of equations
\begin{align*}
z_{0}+y_{1}z_{1}+y_{1}^{2}z_{2}+\hdots+y_{1}^{k-1}z_{k-2}
&\equiv Q_{1}\\
z_{0}+y_{2}z_{1}+y_{2}^{2}z_{2}+\hdots+y_{2}^{k-1}z_{k-2}
&\equiv Q_{2}\\
z_{0}+y_{3}z_{1}+y_{3}^{2}z_{2}+\hdots+y_{3}^{k-1}z_{k-2}
&\equiv Q_{3}\\
&\ \,\vdots\\
z_{0}+y_{k-1}z_{1}+y_{k-1}^{2}z_{2}+
\hdots+y_{k-1}^{k-1}z_{k-2}&\equiv Q_{k-1}
\end{align*}
has a solution, \emph{whatever value of $z_{0}$ we take},
so $k-1$ members of the Faculty Board have no
way of saying that any possible values of $S$ is
more likely than any other.

One way of looking at this method of `secret sharing'
is to note that a polynomial
of degree $k-1$ can be recovered from its value at $k$ points
but not from its value at $k-1$ points. However, the proof
that the method works needs to be substantially more careful.
\begin{exercise} Is the secret compromised if the
values of the $x_{j}$ become known?
\end{exercise}    
\section{A short homily on cryptography}
Cryptography is the science of
code making. Cryptanalysis is the
art of
code  breaking.

Two thousand years ago, Lucretius wrote that
`Only recently has the true nature of things
been discovered'. In the same way, mathematicians
are apt to feel that `Only recently has
the true nature of cryptography been
discovered'. The new mathematical science
of cryptography with its promise of codes
which are `provably hard to break' seems
to make everything that has gone before
irrelevant.

It should, however, be observed that the
best cryptographic systems of our ancestors
(such as diplomatic `book codes') served their
purpose of ensuring secrecy for a relatively
small number of messages between a relatively
small number of people extremely well. It
is the modern requirement for secrecy
\emph{on an industrial scale} to cover
endless streams of messages between
many centres which has made necessary
the modern science of cryptography.

More pertinently, it should be remembered
that the German Naval Enigma codes
not only appeared to be `provably hard to break'
(though not against the modern criteria
of what this should mean) but, \emph{considered
in isolation}, probably were unbreakable
in practice\footnote{Some versions remained
unbroken until the end of the war.}.
Fortunately the Submarine codes formed part
of an `Enigma system' with certain
exploitable weaknesses. (For an account
of how these weaknesses arose and how they
were exploited see Kahn's \emph{Seizing
the Enigma}~\cite{Kahn Enigma}.)

Even the best codes are like the
lock on a safe. However good the lock is,
the safe may be broken open by brute force,
or stolen together with its contents,
or a key holder may be
persuaded by fraud or force to open
the lock, or the presumed
contents of the safe
may have been tampered with before they
go into the safe, or \dots.
The coding schemes we shall consider,
are at best, cryptographic \emph{elements}
of larger possible cryptographic \emph{systems}.
The planning of cryptographic systems
requires not only mathematics but also
engineering, economics, psychology,
humility
and an ability to learn from past mistakes.
Those who do not learn the lessons of history are
condemned to repeat them.

In considering a cryptographic system, it
is important to consider its purpose.
Consider a message $M$ sent by $A$ to $B$.
Here are some possible aims.

\noindent
{\bf Secrecy} $A$ and $B$ can be sure that no
third party $X$ can read the message $M$.

\noindent
{\bf Integrity} $A$ and $B$ can be sure that no
third party $X$ can alter the message $M$.

\noindent
{\bf Authenticity} $B$ can be sure that $A$
sent the message $M$.

\noindent
{\bf Non-repudiation} $B$ can prove to
a third party that $A$
sent the message $M$.

When you fill out a cheque giving the sum both
in numbers and words you are seeking to protect
the \emph{integrity} of the cheque. When you
sign a traveller's cheque `in the presence of
the paying officer' the process is intended,
from your point of view,
to protect \emph{authenticity} and, from
the bank's point of view, to produce
\emph{non-repudiation}.

Another point to consider is the level of
security aimed at. It hardly matters if
a few people use forged tickets to travel
on the underground, it does matter
if a single unauthorised individual can gain
privileged access to a bank's central computer
system. If \emph{secrecy} is aimed at, how long
must the secret be kept? Some military
and financial secrets need only remain secret
for a few hours, others must remain secret
for years.

We must also, to conclude this non-exhaustive
list, consider the level of security required.
Here are three possible levels.

(1) Prospective opponents should find
it hard to compromise your system even
if they are in possession of
a plentiful supply of encoded messages $C_{i}$.

(2) Prospective opponents should find
it hard to compromise your system even
if they are in possession of
a plentiful supply of pairs $(M_{i},C_{i})$
of messages $M_{i}$ together with their encodings $C_{i}$.

(3) Prospective opponents should find
it hard to compromise your system even
if they are allowed to produce messages $M_{i}$
and given their encodings $C_{i}$.

\noindent Clearly, safety
at level (3) implies safety at level (2)
and safety at level (2) implies safety
at level (1). Roughly speaking, the best Enigma codes
satisfied (1). The German Navy believed on good
but mistaken grounds that they satisfied (2).
Level (3) would have appeared evidently impossible
to attain until a few years ago.
Nowadays, level (3) is considered
a minimal requirement for a really secure system.

\section{Stream ciphers} One natural way of
enciphering is to use a \emph{stream cipher}.
We work with streams (that is, sequences) of
elements of ${\mathbb F}_{2}$.
We use a \emph{cipher stream} $k_{0}$, $k_{1}$,
$k_{2}$ \dots. The \emph{plain text stream}
$p_{0}$, $p_{1}$, $p_{2}$, \dots is enciphered
as the \emph{cipher text stream}
$z_{0}$, $z_{1}$, $z_{2}$, \dots  given by
\[z_{n}=p_{n}+k_{n}.\]

This is an example of a \emph{private key} or
\emph{symmetric} system. The security of
the system depends on a secret (in our
case the cipher stream) ${\mathbf k}$
shared between the cipherer and the encipherer.
Knowledge of an enciphering method makes it
easy to work out a deciphering method
and vice versa. In our case a deciphering method
is given by the observation that
\[p_{n}=z_{n}+k_{n}.\]
(Indeed, writing $\alpha(\mathbf{p})=\mathbf{p}+\mathbf{z}$,
we see that the enciphering function $\alpha$ has
the property that $\alpha^{2}=\iota$ the identity map.
Ciphers like this are called \emph{symmetric}.)

In the one-time pad, first discussed by
Vernam in 1926, the cipher stream is a random
sequence $k_{j}=K_{j}$, where the $K_{j}$ are
independent random variables with
\[\Pr(K_{j}=0)=\Pr(K_{j}=1)=1/2.\]
If we write $Z_{j}=p_{j}+K_{j}$, then we see that
the $Z_{j}$ are
independent random variables with
\[\Pr(Z_{j}=0)=\Pr(Z_{j}=1)=1/2.\]
Thus (in the absence of any knowledge of the ciphering stream)
the code-breaker is just faced by a stream of
perfectly random binary digits. Decipherment
is impossible in principle.

It is sometimes said that it is hard to find
random sequences, and it is, indeed, rather harder
than might appear at first sight, but it is
not too difficult to rig up a system for producing
`sufficiently random' sequences\footnote{Take ten
of your favourite long books, convert to
binary sequences $x_{j,n}$ and set
$k_{n}=\sum_{j=1}^{10}x_{j,1000+j+n}+s_{n}$
where $s_{n}$ is the output of your favourite
`pseudo-random number generator'
(in this connection see Exercise~\ref{E;add}). Give a memory stick
with a copy of ${\mathbf k}$ to your friend
and, provided both of you obey some elementary rules,
your correspondence
will be safe from MI5. The anguished debate in the US
about codes and
privacy refers to the privacy of large organisations
and their clients, not the privacy of communication
from individual to individual.}.
The secret
services of the former Soviet Union were
particularly fond of one-time 
pads. 
The real
difficulty lies in the necessity for sharing
the secret sequence ${\mathbf k}$. If a random sequence
is reused it ceases to be random (it becomes
`the same code as last Wednesday' or the
`the same code as Paris uses') so, when
there is a great deal of code traffic\footnote{In 1941, 
the Soviet Union's
need for one-time pads suddenly
increased and it appears that pages were reused
in different pads. If the reader reflects, she will see that,
though this is a mistake, 
it is one which it is very difficult to exploit.
However, under the pressure of the cold war, US code-breakers
managed to decode messages which, although several years old,
still provided useful information. After 1944, the Soviet Union's
one-time pads became genuinely one-time again and the
coded messages became indecipherable.}, new
one-time pads must be sent out.
If random bits can be safely communicated,
so can ordinary messages and the exercise
becomes pointless.

In practice, we would like to start from a short
shared secret `seed' and generate a ciphering
string ${\mathbf k}$ that `behaves like a random
sequence'. This leads us straight into deep
philosophical waters\footnote{Where we drown
at once, since the best (at least, in
my opinion)
modern view is that any sequence that can be generated
by a program of reasonable length from a `seed'
of reasonable size is automatically non-random.}.
As might be expected, there is an illuminating
discussion in Chapter~III
of Knuth's marvellous \emph{The Art of
Computing Programming}~\cite{Knuth}.
Note, in particular, his warning:
\begin{quote} \emph{\dots random numbers
should not be generated with a method chosen at
random.} Some theory should be used.
\end{quote}

One way that we might try to generate
our ciphering string is to use a general
feedback shift register $f$ of length $d$
with the initial fill $(k_{0},k_{1},\dots,k_{d-1})$
as the secret seed.
\begin{lemma} If $f$ is a general
feedback shift register of length
$d$, then, given any
initial fill $(k_{0},k_{1},\dots,k_{d-1})$,
there will exist $N,M\leq 2^{d}$
such that the output stream ${\mathbf k}$
satisfies $k_{r+N}=k_{r}$ for all $r\geq M$.
\end{lemma}
\begin{exercise}\label{E;rational} 
Show that the decimal expansion of
a rational number must be a recurrent expansion.
Give a bound for the period in terms of the quotient.
Conversely, by considering geometric series, or otherwise,
show that a recurrent decimal represents
a rational number.
\end{exercise}
\begin{lemma}\label{maximum linear}
Suppose that $f$ is a linear
feedback register of length $d$.

(i) $f(x_{0},x_{1},\dots,x_{d-1})=(x_{0},x_{1},\dots,x_{d-1})$
if $(x_{0},x_{1},\dots,x_{d-1})=(0,0,\dots,0)$.

(ii) Given any
initial fill $(k_{0},k_{1},\dots,k_{d-1})$,
there will exist $N,M\leq 2^{d}-1$
such that the output stream ${\mathbf k}$
satisfies $k_{r+N}=k_{r}$ for all $r\geq M$.
\end{lemma}
We can complement Lemma~\ref{maximum linear}
by using Lemma~\ref{Galois} and the associated
discussion.
\begin{lemma}\label{L;period bound}
A linear feedback register of length
$d$ attains its maximal period $2^{d}-1$
(for a non-trivial initial fill)
when the roots of the auxiliary polynomial\footnote{In
this sort of context we shall sometimes refer
to the `auxiliary polynomial' as the `feedback polynomial'.} 
are primitive elements of ${\mathbb F}_{2^{d}}$.
\end{lemma}
\noindent
(We will note why this result is plausible, but
we will not prove it. See Exercise~\ref{E;maximum period}
for a proof.)


It is well known that short period streams are dangerous.
During World War~II the British Navy used codes
whose period was adequately long for peace time use.
The massive increase in traffic required by
war time conditions meant that the period was
now too short. By dint of immense toil, German naval
code breakers were able to identify coincidences
and crack the British codes.

Unfortunately, whilst short periods are definitely
unsafe, it does not follow that long periods
guarantee safety. Using the
Berlekamp--Massey method we see that stream
codes based on linear feedback registers
are unsafe at level (2).
\begin{lemma}\label{can opener}
Suppose that an unknown
\emph{cipher stream} $k_{0}$, $k_{1}$,
$k_{2}$ \dots is produced by an
unknown linear feedback
register $f$ of unknown length $d\leq D$.
The \emph{plain text stream}
$p_{0}$, $p_{1}$, $p_{2}$, \dots is enciphered
as the \emph{cipher text stream}
$z_{0}$, $z_{1}$, $z_{2}$, \dots  given by
\[z_{n}=p_{n}+k_{n}.\]
If we are given $p_{0}$, $p_{1}$, \dots $p_{2D-1}$
and $z_{0}$, $z_{1}$, \dots $z_{2D-1}$
then we can find $k_{r}$ for all $r$.
\end{lemma}
Thus if we have a message of length twice the
length of the linear feedback register
together with its encipherment the code
is broken.

It is easy to construct immensely complicated looking
linear feedback registers with hundreds of registers.
Lemma~\ref{can opener}
shows that, from the point of view of a determined,
well equipped and technically competent
opponent, cryptographic systems based on such registers
are the equivalent of leaving your house key hidden
under the door mat.
Professionals say that such systems seek `security
by obscurity'.

However, if you do not wish
to baffle the CIA, but merely prevent little old
ladies in tennis shoes watching subscription
television without paying for it, systems
based on linear feedback registers are cheap
and quite effective. Whatever they may say in public,
large companies are happy to tolerate a certain level
of fraud. So long as 99.9\% of the calls made are
paid for, the profits of a telephone company are
essentially unaffected by the .1\% which `break the
system'.


What happens if we try some simple tricks
to increase the complexity of the
cipher text stream?
\begin{lemma}\label{clown one}
If $x_{n}$ is a stream
produced by a linear feedback system of length $N$
with auxiliary
polynomial $P$ and $y_{n}$  is a stream
produced by a linear feedback system of length $M$
with auxiliary polynomial $Q$,
then $x_{n}+y_{n}$ is a stream
produced by a linear feedback system of length $N+M$
with auxiliary polynomial $P(X)Q(X)$.
\end{lemma}
\noindent Note that this means that adding streams
from two linear feedback system
is no more  economical than producing
the same effect with one. Indeed the situation
may be worse since \emph{a stream produced by
linear feedback system of given length may, possibly,
also be produced  by another linear feedback system
of shorter length}.

\begin{lemma}\label{clown two}
Suppose that $x_{n}$ is a stream
produced by a linear feedback system of length $N$
with auxiliary
polynomial $P$ and $y_{n}$  is a stream
produced by a linear feedback system of length $M$
with auxiliary polynomial $Q$.
Let  $P$ have roots $\alpha_{1}$, $\alpha_{2}$,
\dots $\alpha_{N}$ and $Q$ have roots
$\beta_{1}$, $\beta_{2}$,
\dots $\beta_{M}$ over some field $K\supseteq {\mathbb F}_{2}$.
Then $x_{n}y_{n}$ is a stream produced by
a linear feedback system of length $NM$
with auxiliary polynomial
\[\prod_{1\leq i\leq N}\prod_{1\leq i\leq M}(X-\alpha_{i}\beta_{j}).\]
\end{lemma}
\noindent
We shall probably only prove Lemmas~\ref{clown one}
and~\ref{clown two} in the case when all roots are
distinct, leaving the more general case as an
easy exercise. We shall also not prove that
the polynomial
$\prod_{1\leq i\leq N}\prod_{1\leq i\leq M}(X-\alpha_{i}\beta_{j})$
obtained in Lemma~\ref{clown two}
actually lies in ${\mathbb F}_{2}[X]$ but (for those who
are familiar with the phrase in quotes)
this is an easy exercise
in `symmetric functions of roots'.

Here is an even easier remark.
\begin{lemma}\label{clown three}
Suppose that $x_{n}$ is a stream
which is periodic with period $N$
and $y_{n}$ is a stream which is periodic
with period $M$. Then the streams
$x_{n}+y_{n}$ and $x_{n}y_{n}$
are periodic with periods dividing
the lowest common multiple of $N$ and $M$.
\end{lemma}
\begin{exercise}\label{Fish}
One of the most confidential
German codes (called FISH by the British)
involved a complex mechanism which
the British found could be simulated
by two loops of paper tape of
length $1501$ and $1497$. If $k_{n}=x_{n}+y_{n}$
where $x_{n}$ is a stream of period $1501$
and $y_{n}$ is a stream of period $1497$,
what is the longest possible period of $k_{n}$?
How many consecutive values of $k_{n}$ would you
need to to find the underlying linear feedback register
using the Berlekamp--Massey method if you did
not have the information given in the question? 
If you had
all the information given in the question
how many values of $k_{n}$ would you need?
(Hint, look at $x_{n+1497}-x_{n}$.)

You have shown that, given $k_{n}$ for sufficiently
many consecutive $n$ we can find $k_{n}$ for all $n$.
Can you find $x_{n}$ for all $n$? 
\end{exercise}

It might be thought that the lengthening
of the underlying linear feedback system
obtained in Lemma~\ref{clown two} is worth
having, but it is bought at a substantial
price. Let me illustrate this by an informal
argument. Suppose we have 10
streams  $x_{j,n}$ (without any peculiar
properties) produced by
linear feedback registers of length about
100. If we form
$k_{n}=\prod_{j=1}^{10}x_{j,n}$, then
the Berlekamp--Massey method requires
of the order of $10^{20}$ consecutive
values of $k_{n}$ and the periodicity
of $k_{n}$ can be made still more
astronomical. Our cipher key stream
$k_{n}$ appears safe from prying eyes.
However it is doubtful if the prying
eyes will mind. Observe that (under reasonable
conditions) about $2^{-1}$ of the $x_{j,n}$
will have the value $1$ and about
$2^{-10}$ of the $k_{n}=\prod_{j=1}^{10}x_{j,n}$
will have value $1$. Thus, if
$z_{n}=p_{n}+k_{n}$, in more than
$999$ cases out of a $1000$ we will have
$z_{n}=p_{n}$. Even if we just combine
two streams $x_{n}$ and $y_{n}$ in the
way suggested we may expect $x_{n}y_{n}=0$
for about 75\% of the time.


Here is another example where the apparent
complexity of the cipher key stream
is substantially greater than its true
complexity.
\begin{example} The following is a simplified
version of a standard satellite TV decoder.
We have 3 streams  $x_{n}$, $y_{n}$, $z_{n}$
produced by linear feedback registers.
If the cipher key stream is defined by
\begin{align*}
k_{n}=&x_{n}\qquad\text{if $z_{n}=0$},\\
k_{n}=&y_{n}\qquad\text{if $z_{n}=1$},
\end{align*}
then
\[k_{n}=(y_{n}+x_{n})z_{n}+x_{n}\]
and the cipher key stream is that produced
by linear feedback register.
\end{example}

We must not jump to the conclusion 
that the best way round
these difficulties is to use a non-linear
feedback generator $f$. This is not the
easy way out that it appears. If chosen
by an amateur, the  complicated looking
$f$ so produced will have the apparent
advantage that we do not know what is
wrong with it and the very real disadvantage
that we do not know what is
wrong with it.



Another approach is to observe that,
so far as the potential code breaker is
concerned, the
cipher stream method only combines the
`unknown secret' (here the feedback generator
$f$ together with the seed $(k_{0},k_{1},\dots,k_{d-1})$)
with the unknown message ${\mathbf p}$ in a rather
simple way. It might be better to consider a system
with two functions
$F:{\mathbb F}_{2}^{m}\times{\mathbb F}_{2}^{n}
\rightarrow{\mathbb F}_{2}^{q}$ and
$G:{\mathbb F}_{2}^{m}\times{\mathbb F}_{2}^{q}
\rightarrow{\mathbb F}_{2}^{n}$
such that
\[G({\mathbf k},F({\mathbf k},{\mathbf p}))={\mathbf p}.\]
Here ${\mathbf k}$ will be
the shared secret, ${\mathbf p}$ the message, and
${\mathbf z}=F({\mathbf k},{\mathbf p})$ the
encoded message which can be decoded by
using the fact that $G({\mathbf k},{\mathbf z})
={\mathbf p}$.

In the next section we shall see that an even
better arrangement is possible. However, arrangements
like this have the disadvantage that
the message ${\mathbf p}$ must be entirely known
before it is transmitted  and the encoded
message ${\mathbf z}$ must have been entirely
received before it can be decoded. Stream ciphers
have the advantage that they can be decoded
`on the fly'. They are also much more error
tolerant. A mistake in the coding, transmission
or decoding of a single element only produces
an error in a single place of the sequence.
There will continue to be circumstances
where stream ciphers are appropriate.

There is one further remark to be made.
Suppose, as is often the case, that
we know $F$, that $n=q$
and we know the `encoded message'
${\mathbf z}$. Suppose also
that we know that the `unknown secret' or `key'
${\mathbf k}\in \mathcal{K}\subseteq {\mathbb F}_{2}^{m}$
and the
`unknown message' 
${\mathbf p}\in \mathcal{P}\subseteq {\mathbb F}_{2}^{n}$.
We are then faced with the problem:-
Solve the system
\begin{equation*}
{\mathbf z}=F({\mathbf k},{\mathbf p})
\ \text{where}\ {\mathbf k}\in \mathcal{K},
\  {\mathbf p}\in \mathcal{P}. \tag*{$\bigstar$}
\end{equation*}
Speaking roughly, the task is
hopeless unless $\bigstar$ has a unique
solution\footnote{`According to some, the primordial
Torah was inscribed in black flames on white fire.
At the moment of its creation, it appeared as a series
of letters not yet joined up in the form of words.
For this reason, in the Torah rolls there appear neither
vowels nor punctuation, nor accents; for the original
Torah was nothing but a disordered heap of letters.
Furthermore, had it not been for Adam's sin, these letters
might have been joined differently to form
another story. For the kabalist, God will abolish the
present ordering of the letters, or else will teach us
how to read them according to a new disposition
only after the coming of the Messiah.' (\cite{Eco}, Chapter 2.)
A reader of this footnote has directed me to the 
\emph{International Torah Codes Society}.}
Speaking even more roughly, this is unlikely to happen
if $|\mathcal{K}||\mathcal{P}|>2^{n}$ and is
likely to happen if $2^{n}$ is substantially
greater than $|\mathcal{K}||\mathcal{P}|$.
(Here, as usual, $|\mathcal{B}|$
denotes the number of elements of $\mathcal{B}$.)

Now recall the definition of the information rate
given in Definition~\ref{information rate}.
If the message set ${\mathcal M}$
has information rate $\mu$
and the key set (that is the
shared secret set) $\mathcal{K}$
has information rate $\kappa$, then, taking logarithms,
we see that, if
\[n-m\kappa-n\mu\]
is substantially greater than $0$, then $\bigstar$
is likely to have a unique solution, but, if it is
substantially smaller, this is unlikely.
\begin{example} Suppose that,
instead of using binary
code, we consider an alphabet of 27 letters
(the English alphabet plus a space). We must
take logarithms to the base 27, but the considerations
above continue to apply. The English language
treated in this way has information rate
about .4. (This is very much a ball park figure.
The information rate is certainly less than .5
and almost certainly greater than .2.)

(i) In the Caesar code, we replace the $i$th element
of our alphabet by the $i+j$th (modulo 27). The
shared secret is a single letter (the code for $A$ say).
We have $m=1$, $\kappa=1$ and $\mu\approx .4$. Thus
\[n-m\kappa-n\mu\approx .6n-1.\]
If $n=1$ (so $n-m\kappa-n\mu\approx-.4$) it is obviously
impossible to decode the message. If $n=10$
(so $n-m\kappa-n\mu\approx 5$) a simple search through the
27 possibilities will almost always give a single
possible decode.

(ii) In a simple substitution code, a permutation of the
alphabet is chosen and applied to each letter of the
code in turn. The shared secret is a sequence of
26 letters (given the coding of the first 26 letters,
the 27th can then be deduced).
We have $m=26$, $\kappa=1$ and $\mu\approx .4$. Thus
\[n-m\kappa-n\mu\approx .6n-26.\]
In \emph{The Dancing Men}, Sherlock Holmes
solves such a code with $n=68$
(so $n-m\kappa-n\mu\approx 15$) without straining
the reader's credulity too much and I
would think that, unless the message is
very carefully chosen, most of my audience
could solve such a code with $n=200$
(so $n-m\kappa-n\mu\approx 100$).

(iii) In the one-time pad $m=n$ and $\kappa=1$,
so (if $\mu>0$)
\[n-m\kappa-n\mu=-n\mu\rightarrow-\infty\]
as $n\rightarrow\infty$.

(iv) Note that the larger $\mu$ is, the slower
$n-m\kappa-n\mu$ increases. This corresponds
to the very general statement that the
higher the information rate of the messages,
the harder it is to break the code in which they
are sent.
\end{example}

The ideas just introduced can be formalised
by the notion of unicity distance.
\begin{definition}\label{unicity}
The \emph{unicity distance}
of a code is the number of bits of message
required to exceed the number of bits of
information in the key plus the number
of bits of information in the message.
\end{definition}

(The notion of information content
brings us back to Shannon whose paper
\emph{Communication theory of secrecy systems}\footnote{Available
on the web and in his \emph{Collected Papers}.},
published in 1949, forms the first modern treatment
of cryptography in the open literature.) 

If we only use our code once to send a message
which is substantially shorter than the unicity
distance, we can be confident that no code breaker,
however gifted, could break it, simply because
there is no unambiguous decode.
(A one-time pad has unicity distance infinity.)
However, the fact that there is a unique solution
to a problem does not mean that it is easy
to find.
We have excellent reasons, some of which
are spelled out in the next section, to believe that
there exist codes for which the unicity distance
is essentially irrelevant to the maximum safe length
of a message. For these codes, even though there may
be a unique solution, the amount of work required
to find the solutions makes (it is hoped) any
attempt impractical.
\section{Asymmetric systems}\label{S;symmetric} Towards the end
of the previous section, we discussed a general
coding scheme depending on a shared secret
key ${\mathbf k}$ known to the encoder
and the decoder. The scheme can be generalised
still further by splitting the secret in two.
Consider a system
with two functions
$F:{\mathbb F}_{2}^{m}\times{\mathbb F}_{2}^{n}
\rightarrow{\mathbb F}_{2}^{q}$ and
$G:{\mathbb F}_{2}^{p}\times{\mathbb F}_{2}^{q}
\rightarrow{\mathbb F}_{2}^{n}$
such that
\[G({\mathbf l},F({\mathbf k},{\mathbf p}))={\mathbf p}.\]
Here $({\mathbf k},{\mathbf l})$ will be
be a pair of secrets, ${\mathbf p}$ the message and
${\mathbf z}=F({\mathbf k},{\mathbf p})$ the
encoded message which can be decoded by
using the fact that $G({\mathbf l},{\mathbf z})
={\mathbf p}$. In this scheme, the encoder
must know ${\mathbf k}$, but need not know ${\mathbf l}$
and the decoder must know ${\mathbf l}$,
but need not know ${\mathbf k}$. Such a system
is called asymmetric.

So far the idea is interesting but not exciting.
Suppose, however, that we can show that

(i) knowing $F$, $G$ and $\mathbf k$ 
it is very hard to find ${\mathbf l}$

(ii) if we do not know ${\mathbf l}$ then, even if we know $F$, $G$
and ${\mathbf k}$, it is very hard to find ${\mathbf p}$
from $F({\mathbf k},{\mathbf p})$.

\noindent
Then the code is secure at what we called level (3).
\begin{lemma}~\label{level 3}
Suppose that the conditions specified
above hold. Then an opponent who is entitled to
demand the encodings ${\mathbf z}_{i}$ of
any messages ${\mathbf p}_{i}$ they choose
to specify will still find it very hard to find
$\mathbf{p}$ when given
$F({\mathbf k},{\mathbf p})$.
\end{lemma}

Let us write $F({\mathbf k},{\mathbf p})={\mathbf p}^{K_{A}}$
and $G({\mathbf l},{\mathbf z})={\mathbf z}^{K_{A}^{-1}}$
and think of ${\mathbf p}^{K_{A}}$ as participant
$A$'s encipherment of $\mathbf{p}$ and
${\mathbf z}^{K_{A}^{-1}}$ as participant $B$'s
decipherment of ${\mathbf z}$. We then have
\[({\mathbf p}^{K_{A}})^{K_{A}^{-1}}={\mathbf p}.\]
Lemma~\ref{level 3} tells us that such a system
is secure however many messages are sent.
Moreover, if we think of $A$ as a spy-master,
he can broadcast $K_{A}$ to the world (that
is why such systems are called public key
systems) and invite anybody who
wants to spy for him to send him secret messages
in total confidence\footnote{Although we
make statements about certain codes along the lines
of `It does not matter who knows this', you
should remember the German naval saying
`All radio traffic is high treason'. If 
any aspect of a code can be
kept secret, it should be kept secret.}.

It is all very well to describe such a code,
but do they exist? There is very strong evidence
that they do, but, so far, all mathematicians
have been able to do is to show that
provided certain mathematical problems which
are believed to be hard are indeed hard,
then good codes exist.

The following problem is believed to be hard.

\noindent
{\bf Problem} \emph{Given an integer $N$, which is
known to be the product $N=pq$ of two primes
$p$ and $q$, find $p$ and $q$.}

\noindent
Several schemes have been proposed based on the
assumption that this factorisation is hard.
(Note, however, that it is easy to find large `random'
primes $p$ and $q$.) We give a very elegant
scheme due to Rabin and Williams. It makes use
of some simple number theoretic
results from IA and IB.

The reader may well have seen the following results 
before.
In any case, they are
easy to obtain by considering primitive
roots.
\begin{lemma} If $p$ is an odd prime the congruence
\[x^{2}\equiv d\mod{p}\]
is soluble if and only if $d\equiv 0$ or
$d^{(p-1)/2}\equiv 1$ modulo $p$.
\end{lemma}
\begin{lemma} Suppose $p$ is a prime such that
$p=4k-1$ for some integer $k$. Then, if the congruence
\[x^{2}\equiv d\mod{p}\]
has any solution, it has $d^{k}$ as a solution.
\end{lemma}
We now call on the Chinese remainder theorem.
\begin{lemma}\label{square root}
Let $p$ and $q$ be primes of the
form $4k-1$ and set $N=pq$.
Then the following two problems are of equivalent
difficulty.

(A) Given $N$ and $d$ find all the $m$ satisfying
\[m^{2}\equiv d\mod{N}.\]

(B) Given $N$ find $p$ and $q$.
\end{lemma}
\noindent
(Note that, provided that $d\not\equiv 0$,
knowing the solution to (A) for any $d$
gives us the four solutions for the case $d=1$.)
The result is also true but much harder
to prove for general primes $p$ and $q$.



At the risk of giving aid and comfort
to followers of the
Lakatosian heresy, it must be admitted  that
the statement of Lemma~\ref{square root}
does not really tell us what the result
we are proving is,
although the proof makes it  clear
that the result (whatever it may be) is certainly
true. However, with more work, everything can be
made precise.

We can now give the Rabin--Williams scheme.
The spy-master $A$ selects two very large
primes $p$ and $q$. (Since he has only done
an undergraduate course in mathematics,
he will take $p$ and $q$ of the form $4k-1$.)
He keeps the pair $(p,q)$ secret, but broadcasts
the public key $N=pq$. If $B$ wants to
send him a message, she writes it in binary code and
splits it into blocks of length $m$ with
$2^{m}<N<2^{m+1}$. Each of these blocks
is a number $r_{j}$ with $0\leq r_{j}<N$.
$B$ computes $s_{j}$ such that $r_{j}^{2}\equiv s_{j}$
modulo $N$ and sends $s_{j}$. The spy-master
(who knows $p$ and $q$) can use the
method of Lemma~\ref{square root} to find
one of four possible values for $r_{j}$
(the four square roots of $s_{j}$).
Of these four possible message blocks
it is almost
certain that three will be garbage, so
the fourth will be the desired message.

If the reader reflects, she will see that
the ambiguity of the root is genuinely
unproblematic. (If the decoding is mechanical
then fixing 50 bits scattered throughout
each block will reduce the risk of ambiguity
to negligible proportions.) Slightly more problematic,
from the practical point of view,
is the possibility that someone could be
known to have sent a very short message,
that is to have started with an $m$  such
that $1\leq m\leq N^{1/2}$ but, provided
sensible precautions are taken,
this should not occur.

If I Google `Casino', then I am instantly put in touch
with several of the world's `most trusted electronic casinos'
who subscribe to `responsible gambling'
and who have their absolute probity established
by `internationally recognised Accredited Test
Facilities'. Given these assurances, it seems churlish
to introduce Alice and Bob who
live in different cities, can only communicate
by e-mail and are so suspicious of each other
that neither will accept the word of the other
as to the outcome of the toss of a coin. 

If, in spite of this difficulty, Alice and Bob wish to
play heads and tails (the technical expression is
`bit exchange' or `bit sharing'), 
then the ambiguity of the Rabin--Williams
scheme becomes an advantage. Let us set out the
steps of a `bit sharing scheme' based on Rabin--Williams.

\noindent{STEP 1} Alice chooses at random two large primes $p$ and $q$
such that $p\equiv q\equiv 3\mod{4}$. She computes $n=pq$
and sends $n$ to Bob.

\noindent{STEP 2} Bob chooses a random integer $r$ with $1<r<n/2$.
(He wishes to hide $r$ from Alice, so he may take whatever 
other precautions he wishes in choosing $r$.) He computes
$m\equiv r^{2}\mod{n}$ and sends $m$ to Alice.

\noindent{STEP 3} Since Alice knows $p$ and $q$ she can
easily compute the $4$ square roots of $m$ modulo $n$.
Exactly two of the roots $r_{1}$ and $r_{2}$ will
satisfy  $1<r<n/2$. (If $s$ is a root, so is $-s$.)
However, Alice has no means of telling which is $r$.
Alice writes out $r_{1}$ and $r_{2}$ in binary
and chooses a place (the $k$th digit say)
where they differ. She then tells Bob `I choose
the value $u$ for the $k$th bit'.

\noindent{STEP 4} Bob tells Alice the value of $r$.
If the value of the $k$th bit of $r$ is $u$, then Alice wins.
If not, Bob wins. Alice checks that $r^{2}\equiv m\mod{n}$.
Since, $r_{1}r_{2}^{-!}$ is a square root of unity 
which is neither $1$ nor $-1$,
knowing $r_{1}$ and $r_{2}$ is equivalent to factoring $n$,
she knows that Bob could not lie about the value of $r$.
Thus Alice is happy.

\noindent{STEP 5} Alice tells Bob the values of $p$ and $q$.
He checks that $p$ and $q$ are primes 
(see Exercise~\ref{Q;Alice cheats} for why he does this)
and finds $r_{1}$ and $r_{2}$. After Bob has verified that
$r_{1}$ and $r_{2}$ do indeed differ in the $k$th bit, he
also is happy, since there is no way Alice could know from 
inspection of $m$ which root he started with. 


\section{Commutative public key systems}
In the previous sections we introduced
the coding and decoding functions
$K_{A}$ and $K_{A}^{-1}$ with the property that
\[({\mathbf p}^{K_{A}})^{K_{A}^{-1}}={\mathbf p},\]
and satisfying the condition that knowledge
of $K_{A}$ did not help very much in finding $K_{A}^{-1}$.
We usually require, in addition, that our
system be \emph{commutative} in the sense that
\[({\mathbf p}^{K_{A}^{-1}})^{K_{A}}={\mathbf p}.\]
and that knowledge
of $K_{A}^{-1}$ does not help very much
in finding $K_{A}$. The Rabin--Williams scheme, as
described in the last section, does not have this
property.

Commutative public key codes
are very flexible and provide us with simple means
for maintaining integrity, authenticity and
non-repudiation. (This is not to say that
non-commutative codes can not do the same;
simply that commutativity makes many
things easier.)

\vspace{1\baselineskip}

\noindent
{\bf Integrity and non-repudiation}
Let $A$ `own a code', that is
know both $K_{A}$ and $K_{A}^{-1}$. Then $A$ can
broadcast $K_{A}^{-1}$ to everybody so that
everybody can decode but only $A$ can encode.
(We say that $K_{A}^{-1}$ is the
\emph{public key} and  $K_{A}$ the
\emph{private key}.)
Then, for example,
$A$ could issue tickets to the castle
ball carrying the coded message `admit Joe Bloggs'
which could be read by the recipients and the
guards but would be unforgeable. However, for
the same reason, $A$ could not deny
that he had issued the invitation.

\vspace{1\baselineskip}


\noindent
{\bf Authenticity} If $B$ wants to be sure that
$A$ is sending a message then $B$ can send $A$
a harmless random message ${\mathbf q}$.
If $B$ receives back a message ${\mathbf p}$
such that ${\mathbf p}^{K_{A}^{-1}}$
ends with the message ${\mathbf q}$ then
$A$ must have sent it to $B$. (Anybody
can \emph{copy} a coded message but
only $A$ can control the content.)

\vspace{1\baselineskip}

\noindent
{\bf Signature} Suppose now that $B$ also
owns a commutative
code pair $(K_{B},K_{B}^{-1})$ and has broadcast
$K_{B}^{-1}$. If $A$ wants to send a message
${\mathbf p}$ to $B$ he computes
${\mathbf q}={\mathbf p}^{K_{A}}$ and
sends ${\mathbf p}^{K_{B}^{-1}}$
followed by ${\mathbf q}^{K_{B}^{-1}}$.
$B$ can now use the fact that
\[({\mathbf q}^{K_{B}^{-1}})^{K_{B}}=\mathbf{q}\]
to recover ${\mathbf p}$ and ${\mathbf q}$.
$B$ then observes that ${\mathbf q}^{K_{A}^{-1}}={\mathbf p}$.
Since only $A$ can produce a pair $({\mathbf p},{\mathbf q})$
with this property, $A$ must have written it.

\vspace{1\baselineskip}

There is now a charming little branch of the mathematical
literature based on these ideas
in which Albert gets Bertha to authenticate
a message from  Caroline to David using information
from Eveline, Fitzpatrick, Gilbert and Harriet whilst
Ingrid, Jacob, Katherine and Laszlo play bridge without
using a pack of cards. However, a cryptographic
system is only as strong as its weakest link.
Unbreakable password systems do not prevent
computer systems being regularly penetrated by
`hackers' and however `secure' a transaction
on the net may be it can still involve a rogue at
one end and a fool at the other.

The most famous candidate for a commutative
public key system  is the RSA (Rivest, Shamir, Adleman)
system. It was the RSA system\footnote{A truly
patriotic lecturer would refer to the ECW system,
since Ellis, Cocks and Williamson discovered
the system earlier. However, they worked for GCHQ
and their work was kept secret.}
that first convinced the
mathematical community that public key systems
might be feasible. The reader will have met
the RSA in IA, but
we will push the ideas a little bit further.


\begin{lemma}~\label{lambda}
Let $p$ and $q$ be primes. If $N=pq$
and $\lambda(N)=\lcm(p-1,q-1)$, then
\[M^{\lambda(N)}\equiv 1\pmod{N}\]
for all integers $M$ coprime to $N$.
\end{lemma}

Since we wish to appeal to Lemma~\ref{square root},
we shall assume in what follows that
we have secretly chosen large primes $p$ and $q$.
We choose
an integer $e$ and then use Euclid's algorithm
to check that $e$ and $\lambda(N)$ are coprime
and to find an integer $d$ such that
\[de\equiv 1\pmod{\lambda(N)}.\]
If Euclid's algorithm reveals that
$e$ and $\lambda(N)$ are not coprime, we try another $e$.
Since others may be better psychologists than we are,
we would be wise to use some sort of random method
for choosing $p$, $q$ and $e$.

The public key includes the value of $e$ and $N$,
but we keep secret the value of $d$. Given
a number $M$ with $1\leq M\leq N-1$, we encode
it as the integer $E$ with $1\leq E\leq N-1$
\[E\equiv M^{d}\pmod{N}.\]
The \emph{public} decoding method is given by the
observation that
\[E^{e}\equiv M^{de}\equiv M\]
for $M$ coprime to $N$. (The probability that
$M$ is not coprime to $N$ is so small that it
can be neglected.)
As was observed in IA, high powers are easy to compute.

\begin{exercise} Show how $M^{2^{n}}$ can be
computed using $n$ multiplications. If $1\leq r\leq 2^{n}$
show how $M^{r}$ can be computed using at most $2n$
multiplications. 
\end{exercise} 


To show that (providing that factoring $N$ is
indeed hard) finding $d$ from $e$ and $N$
is hard we use the following lemma.

\begin{lemma}\label{some root}
Suppose that $d$, $e$ and $N$ are
as above. Set $de-1=2^{a}b$ where $b$ is odd.

(i) $a\geq 1$.

(ii) If $y\equiv x^{b}\pmod{N}$ and $y\not\equiv 1$
then there exists
an $r$ with $0\leq r\leq a-1$ such that
\[z=y^{2^{r}}\not\equiv 1\ \text{but}
\ z^{2}\equiv 1\pmod{N}.\]
\end{lemma}

Combined with Lemma~\ref{square root}, the idea
of Lemma~\ref{some root} gives a fast \emph{probabilistic
algorithm} where, by making random choices of $x$,
we very rapidly reduce the probability that
we can not find $p$ and $q$ to as close to zero as we wish.
\begin{lemma}\label{L;fast probable} The problem of finding $d$
from the public information $e$ and $N$ is
essentially as hard as factorising $N$.
\end{lemma}

\emph{Remark 1} At first glance, we seem to have done as well
for the RSA code as for the Rabin--Williams code. But
this is not so. In Lemma~\ref{square root} we showed
that finding the four solutions of $M^{2}\equiv E \pmod{N}$
was equivalent to factorising $N$. In the absence
of further information, finding one root is
as hard as finding another. Thus the ability to
break the Rabin-Williams code (without some tremendous
stroke of luck) is equivalent to the ability to
factor $N$.
On the other hand it is, a priori, possible that
someone may find a decoding method
for the RSA code which does not involve knowing $d$.
They would have broken
the RSA code without finding $d$.
It must, however, be said that, in spite of this
problem, the RSA code is much used in practice
and the Rabin--Williams code is not.

\emph{Remark 2} It is natural to ask what evidence
there is that the factorisation problem really is hard.
Properly organised, trial division requires
$O(N^{1/2})$ operations to factorise a number $N$.
This order of magnitude was not bettered
until 1972 when Lehman produced a $O(N^{1/3})$
method. In 1974, Pollard\footnote{Although
mathematically trained, Pollard worked
outside the professional mathematical
community.} produced a $O(N^{1/4})$ method.
In 1979, as interest in the problem grew
because of its connection with secret codes,
Lenstra made a breakthrough to a
$O(e^{c((\log N)(\log\log N))^{1/2}})$
method with $c\approx 2$. Since then
some progress has been made (Pollard
reached $O(e^{2((\log N)(\log\log N))^{1/3}})$
but, in spite of intense efforts,  mathematicians
have not produced anything which would be
a real threat to codes based on the factorisation problem.
A series of challenge numbers is hosted on the Wikipedia
article entitled RSA.
In 1996, it was possible to factor 100 (decimal) digit
numbers routinely, 150 digit numbers with immense
effort but 200 digit numbers were out of
reach. 
In May 2005, the 200 digit challenge number was factored by
F. Bahr, M. Boehm, J. Franke and T. Kleinjunge
as follows
\begin{align*}
&27997833911221327870829467638722601621\\
&07044678695542853756000992932612840010\\
&76093456710529553608560618223519109513\\
&65788637105954482006576775098580557613\\
&57909873495014417886317894629518723786\\
&9221823983\\
&=\ 35324619344027701212726049781984643\\
&686711974001976250236493034687761212536\\
&79423200058547956528088349\\
&\times7925869954478333033347085841480059687\\
&737975857364219960734330341455767872818\\
&152135381409304740185467
\end{align*}
but the 210 digit challenge 
\begin{align*}
&24524664490027821197651766357308801846\\
&70267876783327597434144517150616008300\\
&38587216952208399332071549103626827191\\
&67986407977672324300560059203563124656\\
&12184658179041001318592996199338170121\\
&49335034875870551067
\end{align*}
remains (as of mid-2008) unfactored.
Organisations which use the RSA and related systems
rely on `security through publicity'. Because
the problem of cracking RSA codes is so notorious,
any breakthrough is likely to be publicly
announced\footnote{And if not, is most likely
to be a government rather than a Mafia secret.}.
Moreover, even if a breakthrough occurs, it
is unlikely to be one which can be easily
exploited by the average criminal. So long
as the secrets covered by RSA-type codes
need only be kept for a few months rather
than forever\footnote{If a sufficiently robust
`quantum computer' could be built, then
it could solve the factorisation problem
and the discrete logarithm problem
(mentioned later) with high probability
extremely fast. It is highly unlikely
that such a machine would be or could be
kept secret, since it would have many more
important applications than
code breaking.}, the codes can be considered
to be one of the strongest links in the
security chain.
\section{Trapdoors and signatures}\label{trapdoors}
It might be thought that secure codes
are all that are needed to ensure
the security of communications,
but this is not so. It is not necessary
to read a message to derive information
from it\footnote{During World War~II,
British bomber crews used to spend the morning
before a night raid testing their equipment,
this included the radios.}. In the same way,
it may not be necessary to be able to
write a message in order to tamper with it.

Here is a somewhat far fetched but worrying example.
Suppose that, by wire tapping or by looking over
people's shoulders, I discover that a bank
creates messages in the form $M_{1}$, $M_{2}$
where $M_{1}$ is the name of the client
and $M_{2}$ is the sum to be transferred
to the client's account.
The messages are then encoded
according to the RSA scheme discussed after Lemma~\ref{lambda}
as $Z_{1}=M_{1}^{d}$ and $Z_{2}=M_{2}^{d}$. I then
enter into a transaction with the bank
which adds \$ 1000  to my account.
I observe the resulting $Z_{1}$ and $Z_{2}$
and then transmit $Z_{1}$ followed by $Z_{2}^{3}$.
\begin{example} What will (I hope) be the result
of this transaction?
\end{example}
\noindent
We say that the RSA scheme is vulnerable to
`homomorphism attack' that is to say an attack
which makes use of the fact our code is a homomorphism.
(If $\theta(M)=M^{d}$, then 
$\theta(M_{1}M_{2})=\theta(M_{1})\theta(M_{2})$.)


One way of increasing security against tampering
is to first code our message by a classical coding
method and then use our RSA (or similar) scheme
on the result.
\begin{exercise} Discuss briefly the effect
of first using an RSA scheme and then a classical
code.
\end{exercise}
However there is another way forward which has the
advantage of wider applicability since it
also can be used
to protect the integrity of open (non-coded) messages
and to produce password systems. These are
the so-called \emph{signature systems}.
(Note that we shall be concerned with the
`signature of the message' and not the
signature of the sender.)

\begin{definition}\label{Hash}
A  \emph{signature} or \emph{trapdoor}
or \emph{hashing}
function is a mapping
$H:{\mathcal M}\rightarrow{\mathcal S}$
from the space ${\mathcal M}$ of possible messages
to the space ${\mathcal S}$ of possible signatures.
\end{definition}
\noindent
(Let me admit, at once,
that Definition~\ref{Hash} is more of a statement
of notation than a useful definition.)
The first requirement of a good signature function
is that the space ${\mathcal M}$ should be
much larger than the space ${\mathcal S}$
so that $H$ is a many-to-one function (in fact
a great-many-to-one function) and we can not
work back from $H(M)$ to $M$.  The second requirement
is that ${\mathcal S}$  should be large
so that a forger can not (sensibly) hope
to hit on $H(M)$ by luck.

Obviously we should aim at the same kind of security
as that offered by our `level~2' for codes:-
\begin{quote}
Prospective opponents should find
it hard to find $H(M)$ given $M$ even
if they are in possession of
a plentiful supply of message--signature pairs
$(M_{i},H(M_{i}))$
of messages $M_{i}$ together with their encodings $C_{i}$.
\end{quote}
I leave it to the reader to think about level~3
security (or to look at section 12.6 of \cite{Welsh}).

Here is a signature  scheme due to 
Elgamal\footnote{This 
is Dr Elgamal's own choice of spelling 
according to Wikipedia.}\label{P;Elgamal}.
The message sender $A$ chooses a very large
prime $p$, some integer $1<g<p$
and some other integer $u$ with $1<u<p$
(as usual, some randomisation scheme should be used).
$A$ then releases the values of $p$, $g$
and $y=g^{u}$ (modulo $p$) but keeps the
value of $u$ secret. Whenever he sends
a message $m$ (some positive integer),
he chooses another integer $k$
with $1\leq k\leq p-2$ at random
and computes
$r$ and $s$ with $1\leq r\leq p-1$
and $0\leq s\leq p-2$
by the rules\footnote{There is
a small point which
I have glossed over here and elsewhere.
Unless $k$ and and $p-1$ are coprime
the equation (**) may not be soluble.
However the quickest way to solve (**),
if it is soluble, is Euclid's algorithm
which will also reveal if (**) is
insoluble. If (**) is insoluble, we simply
choose another $k$ at random and try
again.}
\begin{equation*}
r\equiv g^{k} \pmod{p},\tag*{(*)}
\end{equation*}
\begin{equation*}
m\equiv ur+ks \pmod{p-1}.\tag*{(**)}
\end{equation*}
\begin{lemma} If conditions (*) and (**) are satisfied,
then
\[g^{m}\equiv y^{r}r^{s}\pmod{p}.\]
\end{lemma}
\noindent
If $A$ sends the message $m$ followed by the signature $(r,s)$,
the recipient need only verify the relation
$g^{m}\equiv y^{r}r^{s}\pmod{p}$ to check
that the message is authentic\footnote{Sometimes,
$m$ is replaced by some hash function $H(m)$ of $m$
so $(**)$ becomes $H(m)\equiv ur+ks \pmod{p-1}$.
In this case the recipient
checks that $g^{H(m)}\equiv y^{r}r^{s}\pmod{p}$.}.

Since $k$ is random, it is \emph{believed} that the
only way to forge signatures is to find $u$
from $g^{u}$ (or $k$ from $g^{k}$) and it is \emph{believed}
that this problem, which is known as
the discrete logarithm problem, is very
hard.

Needless to say, even if it is impossible to
tamper with a message--signature pair it is always
possible to copy one. Every message should thus
contain a unique identifier such as a time stamp.

The evidence that the  discrete logarithm problem is very
hard is of the same kind of nature and strength
as the evidence that the factorisation problem
is very hard. We conclude our discussion with
a description of the Diffie--Hellman key exchange
system which is also based on the discrete logarithm problem.

The modern coding schemes which we have discussed
have the disadvantage that they require lots of computation.
This is not a disadvantage when we deal slowly
with a few important messages. For the Web, where
we must deal speedily with a lot of less than
world shattering messages sent by impatient
individuals, this is a grave disadvantage.
Classical coding schemes are fast but become
insecure with reuse. \emph{Key exchange schemes}
use modern codes to communicate a new
secret key for each message. Once the secret
key has been sent slowly, a fast classical
method based on the secret key is used to
encode and decode the message. Since a different
secret key is used each time, the classical code
is secure.

How is this done? Suppose $A$ and $B$ are at opposite
ends of a tapped telephone line. $A$ sends $B$
a (randomly chosen) large prime $p$ and a
randomly chosen $g$ with $1<g<p-1$. Since the
telephone line is insecure, $A$ and $B$ must
assume that $p$ and $g$ are public knowledge.
$A$ now chooses randomly a secret number $\alpha$ and
tells $B$ the value of $g^{\alpha}$ (modulo $p$). $B$
chooses randomly a secret number $\beta$ and
tells $A$ the value of $g^{\beta}$ (modulo $p$). Since
\[g^{\alpha\beta}\equiv(g^{\alpha})^{\beta}\equiv(g^{\beta})^{\alpha},\]
both $A$ and $B$ can compute $k=g^{\alpha\beta}$
modulo $p$ and $k$ becomes the shared secret key.

The eavesdropper is left with the problem of
finding $k\equiv g^{\alpha\beta}$ from knowledge
of $g$, $g^{\alpha}$ and $g^{\beta}$ (modulo $p$).
It is conjectured that this is essentially
as hard as finding $\alpha$ and $\beta$
from the values of $g$, $g^{\alpha}$ and $g^{\beta}$ (modulo $p$)
and this is the discrete logarithm problem.
\section{Quantum cryptography} 
In the days 
when messages were sent in the form of letters,
suspicious people might examine the creases 
where the paper was folded for evidence that
the letter had been read by others. Our final
cryptographic system has the advantage that it
too will reveal attempts to read it. It also
has the advantage that, instead of relying
on the unproven belief that a certain mathematical
task is hard, it depends on the fact that a certain
physical task is impossible\footnote{If you
believe our present theories of the universe.}.

We shall deal with a highly idealised system.
The business of dealing with realistic systems
is a topic of active research within the faculty.
The system we sketch is called the BB84 system
(since it was invented by Bennett and Brassard in 1984)
but there is another system invented by Ekert.

Quantum mechanics tells us that a polarised photon
has a state
\[\phi=\alpha|\updownarrow\rangle+
\beta|\leftrightarrow\rangle\]
where $\alpha,\,\beta\in{\mathbb R}$,
$\alpha^{2}+\beta^{2}=1$, 
$\updownarrow\rangle$ is the vertically polarised
state and $\leftrightarrow\rangle$ is the
horizontally polarised state. Such a photon
will pass through a vertical polarising filter
with probability $\alpha^{2}$ 
and its state will then be $\updownarrow\rangle$.
It will pass through a horizontal polarising
filter with probability $\beta^{2}$ 
and its state will then be $\leftrightarrow\rangle$.
We have an orthonormal basis consisting of
$\updownarrow\rangle$
and $\leftrightarrow\rangle$ by $+$.

We now consider a second basis given
by
\[\diagup\rangle=\frac{1}{\sqrt 2}|\updownarrow\rangle+
\frac{1}{\sqrt 2}|\leftrightarrow\rangle
\ \text{and}
\ \diagdown\rangle=\frac{1}{\sqrt 2}|\updownarrow\rangle
-\frac{1}{\sqrt 2}|\leftrightarrow\rangle\]
in which the states correspond to polarisation
at angles $\pi/4$ and $-\pi/4$ to the horizontal.
Observe that a photon in either state
will have a probability $1/2$ of passing through
either a vertical or a horizontal filter
and  will then be in the appropriate state.

Suppose Eve\footnote{This is a traditional pun.}
intercepts a photon passing between Alice and Bob.
If Eve knows that it is either horizontally
or vertically polarised, then she can use a vertical
filter. If the photon passes through, she knows that
it was vertically polarised when Alice sent it
and can pass on a vertically polarised photon to Bob.
If the photon does not pass through through, she knows that
the photon was horizontally polarised
and can pass on a horizontally polarised photon to Bob.
However, if Alice's photon was actually diagonally
polarised (at angle $\pm\pi/4$), this procedure
will result in Eve sending Bob a photon which is
horizontally or vertically polarised.

It is possible that the finder of a fast factorising
method would get a Field's medal. It is certain that
anyone who can do better than Eve would get the Nobel 
prize for physics since they would have overturned
the basis of Quantum Mechanics.

Let us see how this can (in principle) be used
to produce a key exchange scheme
(so that Alice and Bob can agree on a random number
to act as the basis for a classical code).

\noindent{STEP 1} Alice produces a secret
random sequence $a_{1}a_{2}\dots$ of bits (zeros and ones)
and Bob produces another secret
random sequence $b_{1}b_{2}\dots$ of bits.

\noindent{STEP 2} Alice produces another
secret
random sequence $c_{1}c_{2}\dots$.
She transmits it to Bob as follows.
\begin{center}
If $a_{j}=0$ and $c_{j}=0$, she uses a vertically polarised photon.\\
If $a_{j}=0$ and $c_{j}=1$, she uses a horizontally polarised photon.\\
If $a_{j}=1$ and $c_{j}=0$, she uses a `left diagonally' polarised photon.\\
If $a_{j}=1$ and $c_{j}=1$, she uses a `right diagonally' polarised photon.
\end{center}

\noindent{STEP 3} If $b_{j}=0$, Bob uses a vertical polariser
to examine the $j$th photon. If he records a vertical polarisation,
he sets  $d_{j}=0$, if a horizontal he sets $d_{j}=1$.
If $b_{j}=1$, Bob uses a $\pi/4$ diagonal  polariser
to examine the $j$th photon. If he records a left diagonal polarisation,
he sets  $d_{j}=0$, if a right he sets $d_{j}=1$.

\noindent{STEP 4} Bob and Alice use another communication
channel to tell each other the values of the $a_{j}$ and
$b_{j}$. Of course, they should try to keep these communication
secret, but we shall assume that worst has happened and
these values become known to Eve.

\noindent{STEP 5} If the sequences are long, we can be pretty sure,
by the law of large numbers, that $a_{j}=b_{j}$ in about half the cases.
(If not, Bob and Alice can agree to start again.)  
In particular,  we can ensure that, with probability of at least 
$1-\epsilon/4$ (where $\epsilon$ is chosen in advance), 
the number of agreements is sufficiently
large for the purposes set out below.
Alice
and Bob only look at the `good cases' when $a_{j}=b_{j}$.
In such cases, if Eve does not examine the associated photon,
then $d_{j}=c_{j}$. If Eve does examine the associated photon,
then with probability $1/4$, $d_{j}\neq c_{j}$.

To see this, we examine the case when $c_{j}=0$ and
Eve uses a diagonal polariser. (The other cases
may be treated in exactly the same way.)
With probability $1/2$, $a_{j}=1$ so the photon is diagonally
polarised, Eve records the correct polarisation
and sends Bob a correctly polarised photon. Thus $d_{j}=c_{j}$.
With probability $1/2$, $a_{j}=0$ so the photon is vertically
or horizontally polarised. Since Eve records a diagonal polarisation
she will send a diagonally polarised photon to Bob  
and, since Bob's polariser is vertical, he will
record a vertical polarisation
with probability $1/2$.  
  
\noindent{STEP 6} Alice uses another communication
channel to tell Bob the value of a randomly chosen sample
of good cases. Standard statistical techniques
tell Alice and Bob that, if the number of discrepancies
is below a certain level, the probability that
Eve is intercepting more than a previously 
chosen proportion $p$
of photons is less than $\epsilon/4$.
If the number of discrepancies is greater than the
chosen level, Alice and Bob will abandon 
the  attempt to communicate. 

\noindent{STEP 7} If Eve is intercepting less than a proportion $p$
of photons and $q>p$ (with $q$ chosen in advance)
the probability that she will have intercepted more than
a proportion $q$ of the remaining `good' photons is less than $\epsilon/4$.
Although we shall not do this, the reader who has
ploughed through these notes will readily accept that
Bob and Alice can use the message conveyed through the 
remaining good photons to construct 
a common secret such that Eve has probability 
less than $\epsilon/4$ of guessing
it.

Thus, unless they decide that their
messages are being partially read,
Alice and Bob can agree a shared secret
with probability less than $\epsilon$ that
an eavesdropper can guess it.

There are various gaps in the exposition above.
First we have assumed that Eve must hold her polariser
at a small fixed number of angles. A little thought
shows that allowing her a free choice of angle will make
little difference. Secondly, since physical systems
always have imperfections, some `good' photons
will produce errors even in the absence of
Eve. This means that $p$ in STEP~5 must
be chosen above the `natural noise level'
and the sequences must be longer but, again,
this ought to make little difference.
There is a further engineering problem that
it is very difficult just to send single photons
every time. If there are too many groups of photons,
then Eve only need capture one and let the rest go,
so we can not detect eavesdropping. If there are only a few,
then the values of $p$ and $q$ can be adjusted to
take account of this. There are several networks in existence
which employ quantum cryptography.

 
Quantum cryptography has definite advantages when
matched individually against RSA, secret sharing (using a large
number of independent channels) or one-time pads.
It is less easy to find applications where
it is better than the best choice of one of
these three `classical' methods\footnote{One problem
is indicated by the first British military action
in World War I which was to cut the undersea 
telegraph cables linking Germany to the outside world.
Complex systems are easier to disrupt than simple ones.}.

Of course, quantum cryptography will appeal to
those who need to persuade others that they are using
the latest and most expensive technology to
guard their secrets. However as I said before
\emph{coding schemes
are at best, cryptographic elements
of larger possible cryptographic systems.}
If smiling white coated technicians install big 
gleaming machines with `Unbreakable
Quantum Code Company' painted in large letters
above the keyboard in the homes of Alice and Bob,
it does not automatically follow that their
communications are safe. Money will buy
the appearance of security. Only thought will buy
the appropriate security for a given purpose
at an appropriate cost. And even then we can not
be sure.

\begin{verse}
As we know,\\
There are known knowns.\\
There are things we know we know.\\
We also know\\
There are known unknowns.\\
That is to say\\
We know there are some things\\
We do not know.\\
But there are also unknown unknowns,\\
The ones we don't know\\
We don't know\footnote{Rumsfeld}.
\end{verse}

\section{Further reading} For many students
this will be one of the last university mathematics
course they will take. Although the twin
subjects of error-correcting codes and
cryptography occupy a small place in the
grand panorama of modern mathematics,
it seems to me that they form a very suitable
topic for such a final course.

Outsiders often think of mathematicians as
guardians of abstruse but settled knowledge.
Even those who understand that there are
still problems unsettled, ask what
mathematicians will do when they run out
of problems. At a more subtle
level, Kline's magnificent
\emph{Mathematical Thought from Ancient to
Modern Times}~\cite{Kline} is pervaded by
the melancholy thought
that, though the problems
will not run out, they may become more
and more baroque and inbred. `You are not the mathematicians
your
parents were' whispers Kline
`and your problems are not the
problems your parents' were.'

However, when we look at this course, we
see that the idea of error-correcting codes
did not exist before 1940. The best
designs of such codes  depend on the kind
of `abstract algebra' that historians
like Kline and Bell  consider a dead end,
and lie behind the superior
performance of  CD players and
similar artifacts.

In order to go further into the study of codes, whether
secret or error correcting, we need to go
into the question of  how the information
content of a message is to be measured.
`Information theory' has its roots in the
code breaking of World War II (though
technological needs would doubtless
have led to the same ideas shortly
thereafter anyway). Its development
required a level of sophistication
in treating probability which was simply
not available in the 19th century.
(Even the Markov chain is essentially 20th
century\footnote{We are now in the 21st century,
but I suspect that we are still part
of the mathematical `long 20th century'
which started in the 1880s with the work of Cantor
and like minded contemporaries.}.)

The question of what makes a calculation difficult
could not even have been thought about until
G\"{o}del's theorem (itself a product of
the great `foundations crisis' at the beginning
of the 20th century). Developments by
Turing and Church of G\"{o}del's theorem
gave us a theory of computational complexity
which is still under development today.
The question of whether there exist
`provably hard' public codes is intertwined
with still unanswered questions in complexity
theory. There are links with the profound
(and very 20th century) question of what
constitutes a random number.

Finally, the invention of the electronic computer
has produced a cultural change in the attitude
of mathematicians towards algorithms. Before 1950,
the construction of algorithms was a minor interest
of a few mathematicians. (Gauss and Jacobi were
considered unusual in the amount of thought they
gave to actual computation.) Today, we would consider
a mathematician
as much as a maker of algorithms as a prover of theorems.
The notion of the \emph{probabilistic algorithm}
which hovered over much of our discussion of
secret codes is a typical invention of the last
decades of the 20th century.

Although both the subjects of error correcting and   
secret codes are now `mature' in the
sense that they provide usable  and well tested
tools for practical application, they still
contain deep unanswered questions. For example

How close to the Shannon bound can a
`computationally easy' error correcting code
get?

Do provably hard public codes exist?

Even if these questions are too hard, there
must surely exist error correcting and public
codes based on new ideas\footnote{Just as quantum
cryptography was.}. Such ideas would
be most welcome
and, although they are most likely to come
from the professionals, they might come
from outside the usual charmed circles.

Those who wish to learn about error correction
from the horse's mouth will consult Hamming's
own book on the matter~\cite{Hamming}.
For the present course,
the best book I know for further reading
is  Welsh~\cite{Welsh}. After this,
the book of Goldie and Pinch~\cite{Pinch} provides
a deeper idea of the meaning of information
and its connection with the topic. The book by
Koblitz~\cite{Koblitz} develops the number theoretic
background.
The economic and practical importance of
transmitting, storing and processing data
far outweighs the importance of hiding it.
However, hiding data is more romantic.
For budding cryptologists and cryptographers
(as well as those who want a good read),
Kahn's \emph{The Codebreakers}~\cite{Kahn Code}
has the same role as is taken by Bell's
\emph{Men of Mathematics}
for budding mathematicians.


I conclude with a quotation from Galbraith
(referring to his time as ambassador to India)
taken from Koblitz's entertaining text~\cite{Koblitz}.
\begin{quotation}
I had asked that a cable from Washington to New Delhi
\dots be reported to me through the Toronto consulate.
It arrived in code; no facilities existed for decoding.
They brought it to me at the airport --- a mass of
numbers. I asked if they assumed I could read it.
They said no. I asked how they managed.
They said that when something arrived in code,
they phoned Washington and had the original
read to them.
\end{quotation}

\begin{thebibliography}{99}
\bibitem{Eco} U.~Eco \emph{The Search for the Perfect Language}
(English translation), Blackwell, Oxford 1995.
\bibitem{Hamming} R.~W.~Hamming \emph{Coding and Information Theory}
(2nd edition) Prentice Hall, 1986.
\bibitem{Kahn Code} D.~Kahn \emph{The Codebreakers:
The Story of Secret Writing} MacMillan, New York, 1967.
(A lightly revised edition has recently appeared.)
\bibitem{Kahn Enigma} D.~Kahn
\emph{Seizing the Enigma} Houghton Mifflin, Boston, 1991.
\bibitem{Kline} M.~Kline
\emph{Mathematical Thought from Ancient to
Modern Times} OUP, 1972.
\bibitem{Koblitz} N.~Koblitz
\emph{A Course in Number Theory and Cryptography}
Springer, 1987.
\bibitem{Knuth} D.~E.~Knuth
\emph{The Art of Computing Programming}
Addison-Wesley. The third edition of
Volumes I to III is appearing during
this year and the next (1998--9).
\bibitem{Pinch} G.~M.~Goldie and R.~G.~E.~Pinch
\emph{Communication Theory}
CUP, 1991.
\bibitem{Thompson} T.~M.~Thompson
\emph{From Error-correcting Codes through Sphere Packings
to Simple Groups} Carus Mathematical Monographs {\bf 21},
MAA, Washington DC, 1983.
\bibitem{Welsh} D.~Welsh \emph{Codes and Cryptography}
OUP, 1988.
\end{thebibliography}


\newpage
There is a widespread superstition, believed both by supervisors
and supervisees, that exactly twelve questions are required
to provide full understanding of six hours of  mathematics
and that the same twelve questions should be appropriate
for students of all abilities and all levels of diligence. I have tried
to keep this in mind, but have provided some extra questions
in the various exercise sheets
for those who scorn such old wives' tales.
 
\section{Exercise Sheet 1}
\begin{question}\label{C1.1} 
(Exercises~\ref{E;Morse} and~\ref{E;ASCII}.) (i) 
Consider Morse code.
\begin{align*}
A\mapsto \bullet-*\qquad
&&B\mapsto -\bullet\bullet\bullet*\qquad
&&C\mapsto-\bullet-\bullet*\\
D\mapsto -\bullet\bullet*\qquad
&&E\mapsto \bullet*\qquad
&&F\mapsto\bullet\bullet-\bullet*\\
O\mapsto ---*\qquad
&&S\mapsto\bullet\bullet\bullet*\qquad
&&7\mapsto--\bullet\bullet\bullet*
\end{align*}
Decode
$-\bullet-\bullet*---*-\bullet\bullet* \bullet*$.


(ii) Consider ASCII code.
\begin{align*}
A\mapsto 1000001\qquad
&&B\mapsto 1000010\qquad
&&C\mapsto 1000011\\
a\mapsto 1100001\qquad
&&b\mapsto 1100010 \qquad
&&c\mapsto 1100011\\
+\mapsto 0101011\qquad
&&!\mapsto 0100001\qquad
&&7\mapsto 0110111
\end{align*}
Encode $b7!$. Decode $110001111000011100010$.
\end{question}

\begin{question}\label{C1.2} 
(Exercises~\ref{E;fix length},~\ref{E;decodable}
and~\ref{E;auto free}.)
Consider two alphabets ${\mathcal A}$
and ${\mathcal B}$
and a coding function $c:{\mathcal A}\rightarrow{\mathcal B}^{*}$

(i) Explain, without using the notion of prefix-free
codes, why, if $c$ is injective
and fixed length, $c$
is  decodable. Explain why, if $c$ is injective
and fixed length, $c$
is  prefix-free.

(ii) Let ${\mathcal A}={\mathcal B}=\{0,1\}$. If
$c(0)=0$, $c(1)=00$ show that $c$ is injective but $c^{*}$ is not.

(iii) Let ${\mathcal A}=\{1,2,3,4,5,6\}$ and ${\mathcal B}=\{0,1\}$.
Show that there is a variable length coding $c$ such that
$c$ is injective and all code words have length $2$ or less.
Show that there is no decodable coding $c$ such that
all code words have length $2$ or less
\end{question}
\begin{question}\label{C1.3} The product of two codes 
$c_{j}:{\mathcal A}_{j}\rightarrow{\mathcal B}^{*}_{j}$
is the code 
\[g:{\mathcal A}_{1}\times{\mathcal A}_{2}
\rightarrow({\mathcal B}_{1}\cup{\mathcal B}_{2})^{*}\]
given by
$g(a_{1},a_{2})=c_{1}(a_{1})c_{2}(a_{2})$. 

Show that the product of two prefix-free codes is prefix
free, but the product of a decodable code and a prefix-free
code need not even be decodable.
\end{question} 

\begin{question}\label{C1.4}\label{E;Huffman 1 again} 
(Exercises~\ref{E;Huffman 1} and~\ref{E;Huffman 3})

(i) Apply Huffman's algorithm to the nine messages $M_{j}$ where
$M_{j}$ has probability $j/45$
for $1\leq j\leq 9$.

(ii) Consider 
$4$ messages with the following properties. 
$M_{1}$ has probability $.23$,
$M_{2}$ has probability $.24$,
$M_{3}$ has probability $.26$
and $M_{4}$ has probability $.27$. Show that any
assignment of the code words $00$, $01$, $10$
and $11$ produces a best code in the sense of this course.
\end{question}
\begin{question}\label{C1.5} 
(Exercises~\ref{E;Huffman 2} and~\ref{E;Memory}.)
(i) Consider $64$ messages $M_{j}$.
$M_{1}$ has probability $1/2$,
$M_{2}$ has probability $1/4$ and $M_{j}$ has probability
$1/248$ for $3\leq j\leq 64$. 
Explain why, if we use code words of equal length,
then the length of a code word must be at least $6$.
By using the ideas of Huffman's algorithm (you should not
need to go through all the steps) obtain a set of
code words such that the \emph{expected} length of a code word
sent is no more than $3$.

(ii) Let $a,\,b>0$. Show that
\[\log_{a} b=\frac{\log b}{\log a}.\]
\end{question}
\begin{question}\label{C1.6} (Exercise~\ref{E;Fano})
(i) Let ${\mathcal A}=\{1,2,3,4\}$. Suppose that the
probability
that letter $k$ is chosen is $k/10$.
Use your calculator
to find $\lceil -\log_{2} p_{k}\rceil$
and write down a Shannon--Fano
code $c$.

(ii) We found a Huffman code $c_{h}$ for the system
in Example~\ref{E;Huffman do}.
Show
that the entropy is approximately $1.85$,
that ${\mathbb E}|c(A)|=2.4$
and that  ${\mathbb E}|c_{h}(A)|=1.9$.
Check that these results are consistent with
the appropriate theorems of the course.
\end{question}
\begin{question}\label{C1.7} 
(Exercise~\ref{E;Markov})
Suppose that we have a sequence
$X_{j}$ of random variables taking the values
$0$ and $1$. Suppose that $X_{1}=1$ with probability $1/2$
and
$X_{j+1}=X_{j}$ with probability
$.99$ independent of what has gone before.

(i) Suppose we wish to send $10$ successive bits
$X_{j}X_{j+1}\dots X_{j+9}$. Show that if we associate
the sequence of ten zeros with $0$, the sequence
of ten ones with $10$ and any other sequence
$a_{0}a_{1}\dots a_{9}$ with $11a_{0}a_{1}\dots a_{9}$,
we have a decodable code which on average requires
about $5/2$ bits to transmit the sequence.

(ii) Suppose we wish to send the bits
$X_{j}X_{j+10^{6}}X_{j+2\times 10^{6}}\dots X_{j+9\times 10^{6}}$. 
Explain why any decodable code will require
on average at least $10$ bits to transmit the sequence.
(You need not do detailed computations.)
\end{question}

\begin{question}\label{C1.8} 
In Bridge, a $52$ card pack is dealt
to provide $4$ hands of $13$ cards each. 

(i) Purely as
a matter of interest, we consider the following question.
If the contents of a hand are conveyed by one player 
to their partner by a series of nods and shakes of the head
how many movements of the head are required?
Show that at least $40$ movements are required.
Give a simple code requiring $52$ movements.

$[$You may assume for simplicity that the player to
whom the information is being communicated does not
look at her own cards. (In fact this does not make
a difference since the two players do not acquire
any shared information by looking at their
own cards.)$]$



(ii) If instead the player uses the initial letters
of words (say using the $16$ most common letters),
how many words will you 
need to utter\footnote{`Marked cards, M. l'Anglais?' I said, 
with a chilling sneer. 'They are used, I am told, 
to trap players--not unbirched schoolboys.'

'Yet I say that they are marked!' 
he replied hotly, in his queer foreign jargon. 
'In my last hand I had nothing. You doubled the stakes. 
Bah, sir, you knew! You have swindled me!'

'Monsieur is easy to swindle
-- when he plays with a mirror behind him,' 
I answered tartly. \emph{Under the Red Robe} S.~J.~Weyman}?
\end{question}

\begin{question}\label{C1.9} 
(i) In a \emph{comma code}, like
Morse code, one symbol from an alphabet of $m$ letters 
is reserved to end each code word. Show that
this code is prefix-free and give a direct
argument to show that it must satisfy Kraft's inequality.

(ii) Give an example of a code satisfying Kraft's inequality which 
is not decodable.
\end{question}
\begin{question}\label{C1.10} 
Show that if an optimal binary code has
word lengths $s_{1}$, $s_{2}$, \dots $s_{m}$ then
\[m\log_{2} m\leq s_{1}+s_{2}+\dots+s_{m}\leq (m^{2}+m-2)/2.\]
\end{question}
\begin{question}\label{C1.11} (i) It is known that 
exactly one member of the
starship Emphasise has contracted the Macguffin virus.
A test is available that will detect the virus at any dilution.
However, the power required is such that the ship's
force shields
must be switched off\footnote{`Captain, ye canna be serious.'}
for a minute during each test. Blood samples are taken
from all crew members. The ship's computer has worked
out that the probability of crew member number $i$
harbouring the virus is $p_{i}$. (Thus the probability that
the captain, who is, of course, number $1$, has the disease
is $p_{1}$.)   
Explain how, 
by testing pooled
samples, the expected number of tests can be minimised.
Write down the exact form of the test when there are $2^{n}$ 
crew members and $p_{i}=2^{-n}$.

(ii) Questions like~(i) are rather artificial,
since they require that exactly one person carries the virus.
Suppose that the probability that any member of a population
of $2^{n}$ has a certain disease is $p$ (and that
the probability is independent of the health of the others)
and there exists an error free test which can be carried out on pooled
blood samples which indicates the presence of the 
disease in at least one of the samples or its absence from all.

Explain why there cannot be a testing scheme
which can be guaranteed to require less than $2^{n}$
tests to diagnose all members of
the population. How does the scheme suggested
in the last sentence of~(i) need to be modified to take
account of the fact that more than one person 
may be ill (or, indeed, no one may be ill)?
Show that the expected number of tests required by
the modified scheme is no greater than
$pn2^{n+1}+1$. Explain why the cost of testing
a large population of size $x$ is no more than about
$2pcx\log_{2} x$ with $c$ the cost of a test.

(iii) In practice, pooling schemes will be less complicated.
Usually a group of $x$ people are tested jointly
and, if the joint test shows the disease, each is tested individually.
Explain why this is not sensible if $p$ is large
but is sensible (with a reasonable choice of $x$)
if $p$ is small.
If $p$ is small, explain why there is an optimum value for $x$
Write down (but do not 
attempt to solve)
an equation which indicates (in a `mathematical methods' sense)
that optimum value
in terms of $p$, the probability that an individual has the
disease. 

Schemes like these are only worthwhile if the disease is rare and
the test is both expensive and will work on pooled samples.
However, these circumstances do occur together from
time to time and the idea then produces public health
benefits much more cheaply than would otherwise be possible.
\end{question}
\begin{question}\label{C1.12} 
(i) Give the appropriate generalisation
of Huffman's algorithm to an alphabet with $a$ symbols
when you have $m$ messages and $m\equiv 1\mod{a-1}$.

(ii) Prove that your algorithm gives an optimal
solution.

(iii) Extend the algorithm to cover general $m$ by 
introducing messages of probability zero.
\end{question}  
\begin{question}\label{C1.13} 
(i) A set of $m$ apparently identical 
coins consists of $m-1$ coins and one heavier coin.
You are given a balance in which you can weigh
equal numbers of the coins and determine which side
(if either) contains the heavier coin.
You wish to find the heavy coin in the fewest
average number of weighings. 

If $3^{r}+1\leq m\leq 3^{r+1}$ show that you can label
each coin with a ternary number $a_{1}a_{2}\ldots a_{r+1}$
with $a_{j}\in\{0,1,2\}$ in such a way that
the number of coins
having 
$1$ in the $j$th place
equals the number of coins with $2$ in the $j$th place
for each $j$
(think Huffman ternary trees).

By considering the Huffman algorithm problem for
prefix-free codes on an alphabet with
three letters, solve the problem stated in the first 
part and show that you do indeed
have a solution. Show that your
solution also minimises the maximum number
of weighings that you might have to do.

(ii) Suppose the problem is as before but
$m=12$ and the odd coin may be heavier or lighter.
Show that you need at least $3$ weighings.

[In fact you can always do it in $3$ weighings, but
the problem of showing this
`is said to have been planted
during the war \dots by enemy agents
since Operational Research spent so many man-hours
on its solution.'\footnote{The quotation comes
from Pedoe's \emph{The Gentle Art of Mathematics}
which also gives a very pretty solution.
As might be expected, there are many accounts of this problem
on the web.}]
\end{question}
\begin{question}\label{C1.14} Extend the definition of entropy to
a random variable $X$ taking values in the non-negative
integers. (You must allow for the possibility
of infinite entropy.) 

Compute the expected value ${\mathbb E}Y$ and
entropy $H(Y)$ in the case when $Y$ has the geometric
distribution, that is to say 
$\Pr(Y=k)=p^{k}(1-p)$
$[0<p<1]$. Show that, amongst all random variables 
$X$ taking values in the non-negative
integers with the same expected value $\mu$ $[0<\mu<\infty]$,
the geometric distribution maximises the entropy.
\end{question}
\begin{question}\label{C1.15} A source produces a set ${\mathcal A}$
of messages 
$M_{1}$, $M_{2}$,
\dots, $M_{n}$ with non-zero probabilities $p_{1}$, $p_{2}$,
\dots $p_{n}$. Let $S$ be the codeword length when the message
is encoded by a decodable code $c:{\mathcal A}\rightarrow{\mathcal B}^{*}$ 
where ${\mathcal B}$ is an alphabet of $k$ letters.

(i) Show that
\[\left(\sum_{i=1}^{n}\sqrt{p_{i}}\right)^{2}
\leq {\mathbb E}(k^{S})\]

\noindent[Hint: Cauchy--Schwarz,
$p_{i}^{1/2}=p_{i}^{1/2}k^{s_{i}/2}k^{-s_{i}/2}$.]

(ii) Show that
\[\min{\mathbb E}(k^{S})\leq
k\left(\sum_{i=1}^{n}\sqrt{p_{i}}\right)^{2}.\]
where the minimum is taken over all decodable codes.

\noindent[Hint: Look for a code with codeword lengths
$s_{i}=\lceil -\log_{k} p_{i}^{1/2}/\lambda\rceil$ for
an appropriate $\lambda$.]
\end{question}
\newpage
\section{Exercise Sheet 2}
\begin{question}\label{C2.1}(Exercise~\ref{E;Cambridge exam}.)
In an exam
each candidate is asked to write down a Candidate Number of the
form $3234A$, $3235B$, $3236C$,\dots 
(the eleven possible letters are repeated cyclically)
and a desk number. (Thus candidate $0004$ sitting at desk $425$
writes down $0004D--425$.)
The first four numbers
in the Candidate Identifier identify the candidate
uniquely. 
Show that if the candidate makes one error
in the Candidate Identifier
then that error can be detected
without using the Desk Number.
Would this be true if there were $9$ possible
letters repeated cyclically? Would this be true
if there were $12$ possible
letters repeated cyclically?
Give reasons.

Show that if we combine the Candidate Number and
the Desk Number the combined code is one error correcting.
\end{question}
\begin{question}\label{C2.2}
(Exercise~\ref{Reverse}) In the model of a communication
channel, we take the probability $p$ of error
to be less than $1/2$.
Why do we not consider
the case $1\geq p>1/2$? What
if $p=1/2$?
\end{question}
\begin{question}\label{C2.3} (Exercise~\ref{ISBN}.)
If you look
at the inner
title page of almost any book
published between 1974 and 2007,
you will find its International Standard
Book Number (ISBN). The ISBN
uses single digits selected from 0, 1, \dots, 8, 9
and $X$ representing 10. Each ISBN consists
of nine such digits $a_{1}$, $a_{2}$, \dots, $a_{9}$
followed by a single check digit $a_{10}$ chosen
so that
\begin{equation*}
10a_{1}+9a_{2}+ \dots+2a_{9}+a_{10}\equiv 0\mod{11}.\tag*{(*)}
\end{equation*}
(In more sophisticated language, our code $C$ consists
of those elements ${\mathbf a}\in {\mathbb F}_{11}^{10}$
such that $\sum_{j=1}^{10}(11-j)a_{j}=0$.)


(i) Find a couple of books
and check that $(*)$ holds for their ISBNs.

(ii) Show that $(*)$ will not work if you make a mistake
in writing down one digit of an ISBN.

(iii) Show that
$(*)$ may fail to detect two errors.

(iv)  Show that $(*)$ will not work if you interchange
two distinct adjacent digits (a transposition error).

(v) Does~(iv) remain true if we replace `adjacent'
by `different'?
\noindent Errors of type (ii) and (iv) are the most common
in typing.

In communication between publishers and booksellers,
both sides are anxious that errors should be detected
but would prefer the other side to query errors
rather than to guess what the error might have been.

(vi) Since the ISBN contained information such as the name of the
publisher, only a small proportion of possible ISBNs could
be used\footnote{The same problem occurs with telephone numbers.
If we use the Continent, Country, Town, Subscriber system
we will need longer numbers than if we just numbered
each member of the human race.} 
and the system described above
started to `run out of numbers'. A new system
was  introduced which was
is compatible with the system used to label most consumer goods.
After January 2007, the appropriate ISBN became a $13$ digit number
$x_{1}x_{2}\dots x_{13}$ with each digit 
selected from $0$, $1$, \dots, $8$, $9$ and
the check digit $x_{13}$ computed by using the formula
\[x_{13}\equiv -(x_{1}+3x_{2}+x_{3}+3x_{4}+\cdots+x_{11}+ 3x_{12}) 
\mod{10}.\]
Show that we can detect single errors. Give an example
to show that we cannot detect all transpositions.
\end{question}
\begin{question}\label{C2.4}
(Exercise~\ref{bacon}.)
Suppose we use eight hole tape with
the standard paper tape code
and the probability that an error occurs at a particular
place on the tape (i.e. a hole occurs where it should
not or fails to occur where it should) is $10^{-4}$.
A program requires about 10\,000 lines of tape
(each line containing eight places)
using the paper tape code. Using
the Poisson approximation, direct calculation
(possible with a hand calculator but really no
advance on the Poisson method), or otherwise,
show that the probability that the tape
will be accepted as error free by the decoder
is less than .04\%.

Suppose now that we use the Hamming scheme
(making no use of the last place in each line).
Explain why the program requires about
17\,500 lines of tape but that any
particular line will be correctly decoded
with probability about $1-(21\times 10^{-8})$
and the probability that the entire program
will be correctly decoded is better than
99.6\%.
\end{question}
\begin{question}\label{C2.5} 
If $0<\delta<1/2$,
find an $A(\delta)>0$ such that, whenever
$0\leq r\leq n\delta$, we have
\[\sum_{j=0}^{r}\binom{n}{j}\leq A(\delta)\binom{n}{r}.\]
(We use weaker estimates in the course but this is
the most illuminating. The particular value of $A(\delta)$
is unimportant so do not waste time trying to find
a `good' value.)
\end{question}
\begin{question}\label{C2.6}
Show that the $n$-fold repetition code
is perfect if and only if $n$ is odd.
\end{question}
\begin{question}\label{C2.7} 
(i) What is the expected Hamming distance
between two randomly chosen code words in ${\mathbb F}_{2}^{n}$.
(As usual we suppose implicitly that the two choices are independent
and all choices are equiprobable.)

(ii) Three code words are chosen at random 
from ${\mathbb F}_{2}^{n}$. If $k_{n}$ is
the expected value of the distance between the
closest two,
show that $n^{-1}k_{n}\rightarrow 1/2$ as
$n\rightarrow\infty$. 

\noindent[There are many ways to do~(ii). One way is to
consider Tchebychev's inequality.]
\end{question}
\begin{question}\label{C2.8} 
(Exercises~\ref{E;fast Kelly}
and~\ref{E;Slow Kelly}.) Consider the situation
described in the first paragraph of Section~\ref{S;race track}.

(i) Show that for the situation described
you should not bet if $up\leq 1$ and should take
\[w=\frac{up-1}{u-1}\]
if $up>1$.

(ii) Let us write $q=1-p$. Show that, if $up>1$ and we choose
the optimum $w$,
\[{\mathbb E}\log Y_{n}=p\log p+q\log q+\log u-q\log(u-1).\]

(iii) Show that,
if you bet less than the optimal proportion, your fortune
will still tend to increase but more slowly, but, if you bet
more than some proportion $w_{1}$, your fortune will decrease.
Write down the equation for $w_{1}$.

[Moral: If you use the Kelly criterion veer on the side under-betting.]
\end{question}
\begin{question}\label{C2.9} Your employer announces that he is
abandoning the old-fashioned paternalistic scheme
under which he guarantees you a fixed sum $Kx$
(where, of course, $K,\,x>0$)
when you retire. Instead, he will empower you by
giving you a fixed sum $x$ now, to invest as you wish.
In order to help you
and the rest of the staff, your employer  
arranges that you should obtain advice 
from a financial whizkid with a top degree
from Cambridge. After a long lecture in which the whizkid
manages to be simultaneously condescending, boring
and incomprehensible, you come away with the following information.

When you retire,  the world will be in exactly
one of $n$ states. By means of a piece of financial
wizardry called ditching (or something like that)
the whizkid can offer you a pension plan which
for the cost of $x_{i}$ will return $Kx_{i}q_{i}^{-1}$
if the world is in state $i$, but nothing otherwise.
(Here  $q_{i}> 0$ and $\sum_{i=1}^{n}q_{i}=1$.)
The probability that the world will be in state $i$
is $p_{i}$. You must invest the entire fixed
sum. (Formally, $\sum_{i=1}^{n}x_{i}=x$. You must also take
$x_{i}\geq 0$.) On philosophical grounds you decide
to maximise the expected value $S$ of the logarithm
of the sum received on retirement.
Assuming that you will have to live
off this sum for the rest of your life,
explain, \emph{in your opinion}, why this choice is
reasonable or explain why it is unreasonable.

Find the appropriate choices of $x_{i}$.
Do they depend on the $q_{i}$?

Suppose that $K$ is fixed, but the whizkid can choose
$q_{i}$. We may suppose that what is good for you is bad for him
so
he will seek to minimise $S$ for your best choices.  
Show that he will choose $q_{i}=p_{i}$.
Show that,
with these choices,
\[S=\log Kx.\]
\end{question}
\begin{question}\label{C2.10}
Let $C$ be the code consisting of the
word 10111000100 and its cyclic shifts (that is
01011100010, 00101110001 and so on) together with
the zero code word. Is $C$ linear? Show that $C$
has minimum distance 5.
\end{question}
\begin{question}\label{C2.11} 
(i) The original Hamming code
was a $7$ bit code used in an $8$ bit system
(paper tape). Consider the code $c:\{0,1\}^{4}\rightarrow\{0,1\}^{8}$
obtained by using the Hamming code for the first $7$ bits
and the final bit as a check digit so that
\[x_{1}+x_{2}+\dots+x_{8}\equiv 0\mod{2}.\]
Find the minimum distance for this code. How many errors
can it detect? How many can it correct?

(ii) Given a code of length $n$ which corrects $e$ errors
can you always construct a code of length $n+1$
which detects $2e+1$ errors?
\end{question}
\begin{question}\label{C2.12} In general,
we work under the assumption
that all messages sent through our noisy channel
are equally likely. In this question
we drop this assumption. Suppose that
each bit sent through a channel has probability
$1/3$ of being mistransmitted. There are $4$
codewords $1100$, $0110$, $0001$, $1111$
sent with probabilities $1/4$, $1/2$, $1/12$, $1/6$.
If you receive $1001$ what will you decode it as, using
each of the following rules?

(i) The ideal observer rule: find ${\mathbf b}\in C$
so as to maximise 
\[\Pr({\mathbf b}\ \text{sent}\,|\,
{\mathbf u}\ \text{received}\}.\]

(ii) The maximum likelihood rule: find ${\mathbf b}\in C$
so as to maximise 
\[\Pr({\mathbf u}\ \text{received}\,|\,
{\mathbf b}\ \text{sent}\}.\] 

(iii) The minimum distance rule: find ${\mathbf b}\in C$
so as to minimise the Hamming distance 
$d({\mathbf b},{\mathbf u})$ from the 
received message ${\mathbf u}$.
\end{question}
\begin{question}\label{E;random ball}\label{C2.13}

(i) Show that $-t\geq \log(1-t)$ for $0\leq t<1$.

(ii) Show that, if $\delta_{N}>0$, $1-N\delta_{N}>0$ and
$N^{2}\delta_{N}\rightarrow\infty$, then
\[\prod_{m=1}^{N-1}(1-m\delta_{N})\rightarrow 0.\]

(iii) Let $V(n,r)$ be the number of points in a Hamming
ball of radius $r$ in ${\mathbb F}_{2}^{n}$ and
let $p(n,N,r)$ be the probability that $N$ such 
balls chosen at random do not intersect.
By observing that if $m$ non-intersecting
balls are already placed, then an $m+1$st ball
which does not intersect them
must certainly not have its centre
in one of the balls already placed,
show that, if $N_{n}^{2}2^{-n}V(n,r_{n})\rightarrow\infty$,
then  $p(n,N_{n},r_{n})\rightarrow 0$.

(iv) Show that,
if $2\beta+H(\alpha)>1$, 
then $p(n,2^{\beta n},\alpha n)\rightarrow 0$.

Thus simply throwing balls down at random will not
give very good systems of balls with empty intersections.
\end{question}  
\newpage
\section{Exercise Sheet 3}
\begin{question}\label{C3.1}
A message passes through a binary symmetric
channel with probability $p$ of error for each bit
and the resulting message is passed  through a 
second binary symmetric
channel which is identical except 
that there is probability $q$ of error $[0<p,q<1/2]$.
Show that the result behaves as if it had been
passed through
a binary symmetric
channel with probability of error to be determined.
Show that the probability of error is less than $1/2$.
Can we improve the rate at which messages are transmitted
(with low error) by coding,
sending through the first channel,
decoding with error correction
and then recoding, sending through the
second channel and decoding 
with error correction again or will this produce
no improvement on treating the whole thing as a single channel
and coding and decoding only once?
\end{question}
\begin{question}\label{C3.2}
Write down the weight enumerators
of the trivial code (that is to say, ${\mathbb F}_{2}^{n}$), the zero code
(that is to say, $\{{\boldsymbol 0}\}$),
the repetition code and the simple
parity code.
\end{question}
\begin{question}\label{C3.3}
List the codewords of the Hamming (7,4)
code and its dual. Write down the weight enumerators and verify
that they satisfy the MacWilliams identity.
\end{question}
\begin{question}\label{C3.4} (a) Show that if $C$ is linear,
then so are its extension $C^{+}$, truncation
$C^{-}$ and puncturing $C'$, provided the symbol
chosen to puncture by is 0. Give an example to show
that $C'$ may not be linear if we puncture by $1$.


(b) Show that extension followed by truncation does not change
a code. Is this true if we replace `truncation' by `puncturing'?

(c) Give an example where puncturing reduces the information
rate and an example where puncturing increases the information
rate.

(d) Show that the minimum distance of
the parity extension $C^{+}$ is the least even
integer $n$ with $n\geq d(C)$.

(e) Show that
the minimum distance of the truncation $C^{-}$ is $d(C)$ or $d(C)-1$
and that both cases can occur.

(f) Show that puncturing cannot decrease the minimum distance,
but give examples to show that the minimum distance can
stay the same or increase.
\end{question}
\begin{question}\label{3.5}
If $C_{1}$ and $C_{2}$ are linear codes of appropriate
type with generator matrices $G_{1}$ and $G_{2}$,
write down a generator matrix for $C_{1}|C_{2}$.
\end{question}
\begin{question}\label{C3.6}
Show that the weight enumerator
of $RM(d,1)$ is
\[y^{2^{d}}+(2^{d+1}-2)x^{2^{d-1}}y^{2^{d-1}}+x^{2^{d}}.\]
\end{question}
\begin{question}\label{C3.7} 
(i) Show that every codeword in $RM(d,d-1)$ has even weight.

(ii) Show that $RM(m,m-r-1)\subseteq RM(m,r)^{\perp}$.

(iii) By considering dimension, or otherwise, show
that $RM(m,r)$ has dual code $RM(m,m-r-1)$.
\end{question}
\begin{question}\label{C3.8}
(Exercises~\ref{Not perfect} and~\ref{Golay perfect}.) 
We show that,
even if $2^{n}/V(n,e)$ is an integer,
no perfect code may exist.

(i) Verify that
\[\frac{2^{90}}{V(90,2)}=2^{78}.\]

(ii) Suppose that $C$ is a perfect 2 error correcting
code of length $90$ and size $2^{78}$. Explain
why we may suppose, without loss of generality,
that ${\boldsymbol 0}\in C$.

(iii) Let $C$ be as in (ii) with ${\boldsymbol 0}\in C$.
Consider the set
\[X=\{{\mathbf x}\in{\mathbb F}_{2}^{90}:
x_{1}=1,\ x_{2}=1,\ d({\boldsymbol 0},{\mathbf x})=3\}.\]
Show that, corresponding to each ${\mathbf x}\in X$,
we can find a unique ${\mathbf c}({\mathbf x})\in C$
such that $d({\mathbf c}({\mathbf x}),{\mathbf x})=2$.


(iv) Continuing with the argument of (iii), show
that
\[d({\mathbf c}({\mathbf x}),{\boldsymbol 0})=5\]
and that $c_{i}({\mathbf x})=1$ whenever $x_{i}=1$.
If  $\mathbf{y}\in X$,
find the number of solutions to the equation
${\mathbf c}({\mathbf x})={\mathbf c}({\mathbf y})$
with $\mathbf{x}\in X$
and, by considering the number of elements of $X$,
obtain a contradiction.

(v) Conclude that there is no perfect $[90,2^{78}]$ code.

(vi) Show that $V(3,23)$ is a power of $2$. (In this case
a perfect code exists called the binary Golay code.)
\end{question}

\begin{question}{\bf[The MacWilliams identity for binary codes]}%
\label{E;MacWilliams}\label{C3.9}
Let $C\subseteq {\mathbb F}_{2}^{n}$ be a linear code
of dimension $k$.
 
(i) Show that
\[\sum_{{\mathbf x}\in C}(-1)^{{\mathbf x}.{\mathbf y}}
=\begin{cases}
2^{k}&\text{if ${\mathbf y}\in C^{\perp}$}\\
0&\text{if ${\mathbf y}\notin C^{\perp}$}.
\end{cases}\]

(ii) If $t\in{\mathbb R}$, show that
\[\sum_{{\mathbf y}\in {\mathbb F}_{2}^{n}}t^{w({\mathbf y})}
(-1)^{{\mathbf x}.{\mathbf y}}
=(1-t)^{w({\mathbf x})}(1+t)^{n-w({\mathbf x})}.\]

(iii) By using parts~(i) and~(ii) to evaluate
\[\sum_{{\mathbf x}\in C}
\left(\sum_{{\mathbf y}\in {\mathbb F}_{2}^{n}}
(-1)^{{\mathbf x}.{\mathbf y}}\left(\frac{s}{t}\right)^{w({\mathbf y})}
\right)\]
in two different ways, obtain the MacWilliams identity
\[W_{C^{\perp}}(s,t)=2^{-\dim C}W_{C}(t-s,t+s).\]
\end{question} 
\begin{question}\label{C3.10}
An \emph{erasure} is a digit which has
been made unreadable in transmission. Why are they
easier to deal with than errors? Find a necessary
and sufficient condition on the parity check matrix
for it to be always possible to correct
$t$ erasures.  Find a necessary
and sufficient condition on the parity check matrix
for it never to be possible to correct
$t$ erasures (ie whatever message you choose
and whatever $t$ erasures are made the recipient
cannot tell what you sent).
\end{question}
\begin{question}\label{C3.11}
Consider the collection $K$ of polynomials
\[a_{0}+a_{1}\omega\]
with $a_{j}\in{\mathbb F}_{2}$ manipulated
subject to the usual rules of polynomial
arithmetic and to the further condition
\[1+\omega+\omega^{2}=0.\]
Show by finding a generator and writing out its powers
that
$K^{*}=K\setminus\{0\}$ is a cyclic group
under multiplication and deduce that
$K$ is a finite field.

\noindent [Of course, this follows directly from
general theory but direct calculation is not
uninstructive.]
\end{question}
\begin{question}\label{C3.12}\label{E;Hamming original}
(i) Identify the cyclic codes of
length $n$ corresponding to each of the polynomials
$1$, $X-1$  and $X^{n-1}+X^{n-2}+\dots+X+1$.

(ii) Show that there are three cyclic codes
of length 7 corresponding to irreducible
polynomials of which two are versions of Hamming's
original code. What are the other cyclic codes?

(iii) Identify the dual codes for each of the
codes in (ii).
\end{question}
\begin{question}\label{C3.13}
(Example~\ref{Hamming BCH}.)
Prove the following results.

(i) If $K$ is a field containing
${\mathbb F}_{2}$, then $(a+b)^{2}=a^{2}+b^{2}$
for all $a,b\in K$.

(ii) If $P\in {\mathbb F}_{2}[X]$ and $K$ is a field containing
${\mathbb F}_{2}$, then $P(a)^{2}=P(a^{2})$
for all $a\in K$.

(iii) Let $K$ be a field containing
${\mathbb F}_{2}$ in which $X^{7}-1$ factorises
into linear factors. If $\beta$ is a root of $X^{3}+X+1$
in $K$, then $\beta$ is a primitive root of unity
and $\beta^{2}$ is also a root of $X^{3}+X+1$.

(iv) We continue with the notation of~(iii).
The BCH
code with $\{\beta,\beta^{2}\}$ as defining set
is Hamming's original (7,4) code.
\end{question}
\begin{question}\label{C3.14} Let $C$ be a binary linear code of length
$n$, rank $k$ and distance $d$.

(i) Show that $C$ contains a codeword ${\mathbf x}$ with exactly
$d$ non-zero digits.

(ii) Show that $n\geq d+k-1$.

(iii) Prove that truncating $C$ on the non-zero
digits of ${\mathbf x}$ produces a code
$C'$ of length $n-d$, rank $k-1$ and distance
$d'\geq\lceil \tfrac{d}{2}\rceil$.

\noindent[Hint: To show $d'\geq\lceil \tfrac{d}{2}\rceil$,
consider, for ${\mathbf y}\in C$, the coordinates
where $x_{j}=y_{j}$ and the coordinates
where $x_{j}\neq y_{j}$.]

(iv) Show that
\[n\geq d+\sum_{u=1}^{k-1}\lceil \tfrac{d}{2^{u}}\rceil.\]
Why does (iv) imply (ii)? Give an example where
$n>d+k-1$.  
\end{question}
\begin{question}\label{C3.15} Implement the secret sharing method
of page~\pageref{P;secret sharing} with $k=2$, $n=3$, $x_{j}=j+1$
$p=7$, $a_{0}=S=2$, $a_{1}=3$. Check directly that any two
people
can find $S$ but no single individual can.

If we take $k=3$, $n=4$,
$p=6$,  $x_{j}=j+1$ show that 
the first two members and the
fourth member of the Faculty Board will
be unable to determine $S$ uniquely.
Why does this not invalidate our method?
\end{question}
\newpage
\section{Exercise Sheet 4}
\begin{question} (Exercise~\ref{E;rational}.)\label{C4.1} 
Show that the decimal expansion of
a rational number must be a recurrent expansion.
Give a bound for the period in terms of the quotient.
Conversely, by considering geometric series, or otherwise,
show that a recurrent decimal represents
a rational number.
\end{question}

\begin{question}\label{C4.2}
A binary non-linear feedback register of length 4
has defining relation
\[x_{n+1}=x_{n}x_{n-1}+x_{n-3}.\]
Show that the state space contains 4 cycles of lengths
1, 2, 4 and 9
\end{question}
\begin{question}\label{C4.3}
A binary LFR was used to generate
the following stream
\[110001110001\dots\]
Recover the feedback polynomial by the Berlekamp--Massey
method. [The  LFR has length $4$ but you should work through
the trials for length $r$ for $1\leq r\leq 4$.]
\end{question}
\begin{question}\label{C4.4}
(Exercise~\ref{Solve recurrence}.)
Consider the linear recurrence
\begin{equation*}
x_{n}=a_{0}x_{n-d}+a_{1}x_{n-d+1}+\ldots+a_{d-1}x_{n-1}\tag*{$\bigstar$}
\end{equation*}
with $a_{j}\in {\mathbb F}_{2}$ and $a_{0}\neq 0$.

(i) Suppose $K$ is a field containing ${\mathbb F}_{2}$
such that the auxiliary polynomial $C$ has a root $\alpha$
in $K$. Show that $x_{n}=\alpha^{n}$ is a solution of $\bigstar$ in $K$.

(ii) Suppose $K$ is a field containing ${\mathbb F}_{2}$
such that the auxiliary polynomial $C$ has
$d$ distinct roots $\alpha_{1}$, $\alpha_{2}$,
\dots, $\alpha_{d}$ in $K$. Show that the general solution
of $\bigstar$ in $K$ is
\[x_{n}=\sum_{j=1}^{d}b_{j}\alpha_{j}^{n}\]
for some $b_{j}\in K$.
If $x_{0},x_{1},\dots,x_{d-1}\in {\mathbb F}_{2}$,
show that $x_{n}\in {\mathbb F}_{2}$ for all $n$.

(iii) Work out the first few lines of Pascal's triangle
modulo 2. Show that the functions
$f_{j}:{\mathbb Z}\rightarrow{\mathbb F}_{2}$
\[f_{j}(n)=\binom{n}{j}\]
are linearly independent in the sense that
\[\sum_{j=0}^{m}b_{j}f_{j}(n)=0\]
for all $n$ implies $b_{j}=0$ for $1\leq j\leq m$.

(iv) Suppose $K$ is a field containing ${\mathbb F}_{2}$
such that the auxiliary polynomial $C$ factorises
completely into linear factors. If the
root $\alpha_{u}$ has multiplicity $m(u)$ $[1\leq u\leq q]$,
show that the general solution
of $\bigstar$ in $K$ is
\[x_{n}=\sum_{u=1}^{q}\sum_{v=0}^{m(u)-1}
b_{u,v}\binom{n}{v}\alpha_{u}^{n}\]
for some $b_{u,v}\in K$.
If $x_{0},x_{1},\dots,x_{d-1}\in {\mathbb F}_{2}$,
show that $x_{n}\in {\mathbb F}_{2}$ for all $n$.
\end{question}
\begin{question}\label{C4.5} 
Consider the recurrence relation
\[u_{n+p}+\sum_{j=0}^{n-1}c_{j}u_{j+p}=0\]
over a field (if you wish, you may take the field 
to be ${\mathbb R}$
but the algebra is the same for all fields.)
We suppose $c_{0}\neq 0$. 
Write down an $n\times n$
matrix $M$ such that 
\[\left(\begin{matrix}
u_{1}\\u_{2}\\ \vdots\\u_{n}
\end{matrix}\right)
=M\left(\begin{matrix}
u_{0}\\u_{1}\\ \vdots\\u_{n-1}
\end{matrix}\right).\]

Find the characteristic and minimal polynomials for $M$.
Would your answers be the same if $c_{0}=0$?
\end{question} 
\begin{question}\label{C4.6}
(Exercise~\ref{Fish}.)
One of the most confidential
German codes (called FISH by the British)
involved a complex mechanism which
the British found could be simulated
by two loops of paper tape of
length $1501$ and $1497$. If $k_{n}=x_{n}+y_{n}$
where $x_{n}$ is a stream of period $1501$
and $y_{n}$ is a stream of period $1497$,
what is the longest possible period of $k_{n}$?
How many consecutive values of $k_{n}$ would you
need to to find the underlying linear feedback register
using the Berlekamp--Massey method if you did
not have the information given in the question?
If you had
all the information given in the question
how many values of $k_{n}$ would you need?
(Hint, look at $x_{n+1497}-x_{n}$.)

You have shown that, given $k_{n}$ for sufficiently
many consecutive $n$ we can find $k_{n}$ for all $n$.
Can you find $x_{n}$ for all $n$? 
\end{question} 

\begin{question}\label{C4.7}
We work in ${\mathbb F}_{2}$.
I have a secret sequence $k_{1}$, $k_{2}$, \dots  and a
message $p_{1}$, $p_{2}$, \dots, $p_{N}$. I transmit
$p_{1}+k_{1}$, $p_{2}+k_{2}$, \dots $p_{N}+k_{N}$
and then, by error, transmit
$p_{1}+k_{2}$, $p_{2}+k_{3}$, \dots $p_{N}+k_{N+1}$.
Assuming that you know this and that my
message makes sense, how would you go about
finding my message? Can you now decipher
other messages sent using the same
part of my secret sequence?
\end{question}
\begin{question}\label{C4.8}
Give an example of a homomorphism attack
on an RSA code. Show in reasonable
detail that the Elgamal
signature scheme defeats it.
\end{question}
\begin{question}\label{C4.9}
I announce that I shall
be using the Rabin--Williams scheme with
modulus $N$. My agent in X'Dofdro sends
me a message $m$ (with $1\leq m\leq N-1$)
encoded in the requisite form.
Unfortunately, my cat eats the piece of paper
on which the prime factors of $N$ are
recorded, so I am unable to decipher it.
I therefore find a new pair of primes
and announce that I shall
be using the Rabin--Williams scheme with
modulus $N'>N$. My agent now recodes
the message and sends it to me again.

The dreaded SNDO of X'Dofdro intercept
both code messages. Show that they
can find $m$. Can they decipher any
other messages sent to me using only
one of the coding schemes?
\end{question}
\begin{question}\label{C4.10}
Extend the Diffie--Hellman key exchange system
to cover three participants in a way that is
likely to be as secure as the two party scheme.

Extend the system to $n$ parties in such a way that
they can compute their common secret key by at 
most $n^{2}-n$ communications of `Diffie--Hellman type numbers'. 
(The numbers $p$
and $g$ of our original Diffie-Hellman system are
known by everybody in advance.) Show that this can be done using 
at most $2n-2$ communications by including several
`Diffie--Hellman type numbers' in one message.
\end{question}
 
\begin{question}\label{E;abacus}\label{C4.11}
St Abacus, who established written Abacan,
was led, on theological grounds, to use an alphabet containing
only
three letters $A$, $B$ and $C$ and to avoid the use of
spaces. (Thus an Abacan book consists of single word.) 
In modern Abacan, the letter
$A$ has frequency $.5$ and the letters $B$ and $C$ both 
have frequency $.25$. In order to disguise this,
the Abacan Navy uses codes in which the
$3r+i$th number is $x_{3r+i}+y_{i}$ modulo $3$ $[0\leq i\leq 2]$
where $x_{j}=0$ if the $j$th letter of the message is $A$,
$x_{j}=1$ if the $j$th letter of the message is $B$,
$x_{j}=2$ if the $j$th letter of the message is $C$
and $y_{0}$, $y_{1}$ and $y_{2}$ are the numbers $0$, $1$, $2$
in some order.  

Radio interception has picked up the following message.
\[120022010211121001001021002021\]
Although nobody in Naval Intelligence reads Abacan, 
it is believed that the last letter of the message will
be $B$ if the Abacan fleet is at sea. 
The Admiralty are desperate to know the last letter
and send a representative to your rooms
in Baker Street to ask your advice. Give it.
\end{question} 

\begin{question}\label{Q;Alice cheats}\label{C4.12} 
Consider the bit exchange scheme
proposed at the end of Section~\ref{S;symmetric}.
Suppose that we replace STEP~5 by:- Alice sends Bob
$r_{1}$ and $r_{2}$ and Bob checks that 
\[r_{1}^{2}\equiv r_{2}^{2}\equiv m\mod{n}.\]
 
Suppose further that Alice cheats by
choosing $3$ primes $p_{1}$, $p_{2}$, $p_{3}$,
and sending Bob $p=p_{1}$ and $q=p_{2}p_{3}$.
Explain how Alice can shift the odds of heads
to $3/4$. (She has other ways of cheating, but
you are only asked to consider this one.)
\end{question}
\begin{question}\label{C4.13}
(i) Consider the Fermat code
given by the following procedure.
`Choose $N$ a large prime. Choose $e$ and $d$ so that
$a^{de}\equiv a \mod{N}$, encrypt using the publicly known
$N$ and $e$, decrypt using the secret $d$.'
Why is this not a good code?

(ii) In textbook examples of the RSA code we frequently see
$e=65537$. How many multiplications
are needed to
compute $a^{e}$ modulo $N$?

(iii) Why is it unwise to choose primes $p$ and $q$
with $p-q$ small when forming $N=pq$ for the RSA method?
Factorise $1763$.
\end{question}
\begin{question}\label{C4.14} The University of Camford is proud of 
the excellence of its privacy system CAMSEC. 
To advertise this
fact to the world, the Vice-Chancellor decrees that 
the university telephone directory should bear on its cover
a number $N$ (a product of two very large secret primes)
and each name in the University Directory should 
be followed by their personal encryption number $e_{i}$.
The Vice-Chancellor knows all the secret decryption numbers
$d_{i}$ but gives these out on a need to know basis only.
(Of course each member of staff must know their
personal decryption number but they are instructed to keep it secret.)
Messages $a$ from the Vice-Chancellor
to members of staff 
are encrypted in the standard manner
as $a^{e_{i}}$ modulo $N$ and decrypted
as $b^{d_{i}}$ modulo $N$.

(i) The Vice-Chancellor sends a message to
all members of the University.
An outsider intercepts the encrypted message to
individuals $i$ and $j$ where $e_{i}$ and $e_{j}$
are coprime. How can the outsider read the message?
Can she read other messages sent from the Vice-Chancellor
to the $i$th member of staff only?

(ii) By means of a phone tapping device,
the  Professor of Applied Numismatics
(number $u$ in the University Directory)
has intercepted messages from the
Vice-Chancellor to her
hated rival, the
Professor of Pure Numismatics
(number $v$ in the University Directory).
Explain why she can decode them.

What moral should be drawn?
\end{question}
  
\begin{question}\label{C4.15} 
The Poldovian Embassy uses a one-time pad
to communicate with the notorious international spy Ivanovich Smith.
The messages are coded in the obvious way. (If the pad has $C$ the
$3$rd letter of the alphabet and the message has $I$ the $9$th
then the encrypted message has $L$ the $3+9$th. Work modulo $26$.)
Unknown to them, the person whom they employ
to carry the messages is actually the MI5 agent
`Union' Jack Caruthers in disguise.
MI5 are on the verge of arresting Ivanovich
when  `Union' Jack is given the message
\[LRPFOJQLCUD.\]
Caruthers knows that the actual message is 
\[FLYXATXONCE\]
and suggests that `the boffins change things a little'
so that Ivanovich deciphers the message as
\[REMAINXHERE.\] 
The only boffin available is you. Advise MI5.
\end{question} 
\begin{question}\label{E;add}\label{C4.16}
Suppose that $X$ and $Y$ are independent random variables
taking values in ${\mathbb Z}_{n}$. Show that 
\[H(X+Y)\geq \max\{H(X),H(Y)\}.\]
Why is this remark of interest in the context of one-time pads?

Does this result remain true if $X$ and $Y$ need not be independent?
Give a proof or counterexample.
\end{question}


\begin{question}\label{C4.17}
I use the Elgamal signature scheme
described on page~\pageref{P;Elgamal}. Instead of choosing $k$
at random, I increase the value used by $2$ each time I use it.
Show that it will often be possible to find
my privacy key $u$ from two
successive messages.
\end{question}  
\begin{question}\label{C4.18} 
Confident in the unbreakability of
RSA, I write the following. What mistakes have I made?
\begin{center}
0000001 0000000 0002048 0000001 1391142\\
0000000 0177147 1033288 1391142 1174371.
\end{center}
Advise me on how to increase the security of messages.
\end{question}
\begin{question}\label{E;maximum period}\label{C4.19}
Let $K$ be the finite field with $2^{d}$ elements
and primitive root $\alpha$. (Recall that $\alpha$ is
a generator of the cyclic group $K\setminus\{0\}$
under multiplication.) Let $T:K\rightarrow{\mathbb F}_{2}$
be a non-zero linear map. (Here we treat $K$ as a vector
space over ${\mathbb F}_{2}$.)

(i) Show that the map $S:K\times K \rightarrow{\mathbb F}_{2}$
given by $S(x,y)=T(xy)$ is a symmetric bilinear form.
Show further that $S$ is non-degenerate
(that is to say $S(x,y)=0$ for all $x$ implies $y=0$).

(ii) Show that the sequence $x_{n}=T(\alpha^{n})$
is the output from a linear feedback register of length 
at most $d$. (Part~(iii) shows that it must be exactly $d$.)

(iii) Show that the period of the system (that is to say the minimum
period of $T$) is $2^{d}-1$. Explain briefly why this
is best possible.
\end{question}
\end{document}
